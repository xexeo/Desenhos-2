\documentclass{book}
\usepackage[brazilian,english]{babel}
%\usepackage[utf8]{inputenc}
\usepackage[usenames,dvipsnames]{xcolor}
\definecolor{ibm1}{HTML}{648fff}
\definecolor{ibm2}{HTML}{785ef0}
\definecolor{ibm3}{HTML}{dc267f}
\definecolor{ibm4}{HTML}{fe6100}
\definecolor{ibm5}{HTML}{ffb000}
\usepackage[T1]{fontenc}
\usepackage{datetime2}
\usepackage[section]{placeins}
\usepackage{chessboard}
\usepackage[listings]{tcolorbox}

\usepackage{amsmath}
\usepackage{amsfonts}
\usepackage{amssymb}
\usepackage{amsthm}

\newcommand{\gxmat}{\mathbf}



\usepackage{pgf,tikz,tikz-qtree,pgfplots}
\pgfplotsset{compat=1.16}
\usepackage{mathrsfs}
\usepackage{pgf-pie}
\usetikzlibrary{arrows,chains}
\usetikzlibrary{arrows.meta}
\usetikzlibrary{mindmap}
\usetikzlibrary{datavisualization,3d }
\usepackage{forest}
\usetikzlibrary{datavisualization.formats.functions}
\usetikzlibrary{shapes.misc}
\usetikzlibrary{shapes.arrows}
\usetikzlibrary{calc}
\usetikzlibrary{math}
\usetikzlibrary{shapes.symbols}
\usetikzlibrary{shapes.geometric}

\usetikzlibrary{shadows}
\usetikzlibrary{automata} % state accepting
\usetikzlibrary{positioning}  % on grid
\usetikzlibrary{topaths,angles,quotes,babel}
\usetikzlibrary{intersections}
\usetikzlibrary{decorations}
\usetikzlibrary{decorations.shapes}
\usetikzlibrary{decorations.pathmorphing}
\usetikzlibrary{decorations.text}
\usepgflibrary{decorations.pathreplacing}
\usepgflibrary{decorations.markings}
\usepgflibrary{decorations.footprints}
\usepgflibrary{decorations.fractals}
\usetikzlibrary{matrix,arrows}
\usepackage{pgfplots}
\usetikzlibrary{graphs}
\usetikzlibrary{graphdrawing}
\usegdlibrary{force} % LUALATEX is NEEDED

\usetikzlibrary{backgrounds}
\pgfdeclarelayer{background}
\pgfdeclarelayer{foreground}
\pgfdeclarelayer{bm}
\pgfsetlayers{background,bm,main,foreground}
\usetikzlibrary{fit}
\usetikzlibrary{graphs,graphs.standard}
\usetikzlibrary{graphdrawing}
\usegdlibrary{trees,force,layered}

\usepackage{moeptikz}
\usepackage{tikzpeople}
\usetikzlibrary{patterns,patterns.meta}
\pgfplotsset{compat=1.14}
\makeatletter
\pgfdeclareshape{document}{
\inheritsavedanchors[from=rectangle] % this is nearly a rectangle
\inheritanchorborder[from=rectangle]
\inheritanchor[from=rectangle]{center}
\inheritanchor[from=rectangle]{north}
\inheritanchor[from=rectangle]{south}
\inheritanchor[from=rectangle]{west}
\inheritanchor[from=rectangle]{east}
% ... and possibly more
\backgroundpath{% this is new
% store lower right in xa/ya and upper right in xb/yb
\southwest \pgf@xa=\pgf@x \pgf@ya=\pgf@y
\northeast \pgf@xb=\pgf@x \pgf@yb=\pgf@y
% compute corner of ‘‘flipped page’’
\pgf@xc=\pgf@xb \advance\pgf@xc by-10pt % this should be a parameter
\pgf@yc=\pgf@yb \advance\pgf@yc by-10pt
% construct main path
\pgfpathmoveto{\pgfpoint{\pgf@xa}{\pgf@ya}}
\pgfpathlineto{\pgfpoint{\pgf@xa}{\pgf@yb}}
\pgfpathlineto{\pgfpoint{\pgf@xc}{\pgf@yb}}
\pgfpathlineto{\pgfpoint{\pgf@xb}{\pgf@yc}}
\pgfpathlineto{\pgfpoint{\pgf@xb}{\pgf@ya}}
\pgfpathclose
% add little corner
\pgfpathmoveto{\pgfpoint{\pgf@xc}{\pgf@yb}}
\pgfpathlineto{\pgfpoint{\pgf@xc}{\pgf@yc}}
\pgfpathlineto{\pgfpoint{\pgf@xb}{\pgf@yc}}
\pgfpathlineto{\pgfpoint{\pgf@xc}{\pgf@yc}}
}
}
\makeatother

\tikzstyle{doc}=[%
draw,
thick,
align=center,
color=black,
shape=document,
minimum width=20mm,
minimum height=28.2mm,
shape=document,
inner sep=2ex,
rounded corners=0
]
\tikzset{%
  cascaded/.style = {%
    general shadow = {%
      shadow scale = 1,
      shadow xshift = 1ex,
      shadow yshift = 1ex,
      draw,
      thick,
      fill = white},
    general shadow = {%
      shadow scale = 1,
      shadow xshift = .5ex,
      shadow yshift = .5ex,
      draw,
      thick,
      fill = white},
    fill = white,
    draw,
    thick,
    minimum width = 1.5cm,
    minimum height = 2cm}}
%\usepackage[all,cmtip]{xy}

\definecolor{blech}{rgb}{.78,.78.,.62}

\newcommand{\fnurl}[1]{\footnote{\url{#1}}}

\usepackage{hyperref}

\title{\LaTeX\ - A Book of Drawings \\ An ever evolving document}
\author{Geraldo Xexéo}
\date{\today \  \DTMcurrenttime }
\usepackage{chronosys}

\lstdefinestyle{myLateX}
{ language=[LaTeX]{TeX},
  texcsstyle=*\color{blue},
  basicstyle=\ttfamily,
  moretexcs={draw,path,node}, % user command highlight
  frame=single,
}


\begin{document}
\frontmatter
\maketitle

\tableofcontents

\listoffigures

\mainmatter

\chapter{Introduction}

This is a book of drawings made in \LaTeX\  with Tikz\footnote{Pronounced "tics"} and other useful packages. Some drawings are in Portuguese.

%Dois pacotes ou comandos dão conflito. Package xypdf Error: pdfTeX version 1.40.0 or higher is needed for the xypdf. Já para usar o comando \begin{verbatim}\usegdlibrary{force}\end{verbatim} é preciso usar o Lua\LaTeX.

Although most drawings in this book of examples use Tikz\fnurl{https://ctan.org/pkg/pgf?lang=en}, there are some easier solutions for some specific drawings. Moreover, Tikz has multiple libraries that must be included, and I don't kept control of it, I just added everyone.

For example, chessboard\fnurl{https://ctan.org/pkg/chessboard?lang=en} is a useful package for drawing chess boards. I enjoy that it uses a very practical notation that is known to chess players.

The following code generates the image in \autoref{fig:chess}.


\begin{lstlisting}[style=myLateX,caption=Code for a Chess board]
\chessboard[addfen={bnrbnkrq/%
pppppppp/%
8/8/8/8/%
PPPPPPPP/BNRBNKRQ},showmover=false]
\end{lstlisting}

\begin{figure}
\centering
\chessboard[addfen={bnrbnkrq/%
pppppppp/%
8/8/8/8/%
PPPPPPPP/BNRBNKRQ},showmover=false]
\caption{A position form Fischer's Random Chess}
\label{fig:chess}
\end{figure}


\begin{figure}[hbt]
\centering
\begin{tikzpicture}[every node/.style = {minimum width=2cm,minimum height=1em,draw,font=\tiny,node distance=0cm,align=center}]
\node[fill=ibm1] (c1) at (0,0) {\#648fff};
\node[fill=ibm2,right = of c1] (c2)  {\#785ef0};
\node[fill=ibm3,right = of c2] (c3)  {\#dc267f};
\node[fill=ibm4,right = of c3] (c4)  {\#fe6100};
\node[fill=ibm5,right = of c4] (c5)  {\#ffb000};
\node[fill=black,text=white,right = of c5] (c6) {\#000000};
\node[fill=white,right = of c6] (c7) {\#ffffff};
\foreach \i in {100,95,...,0} {
\node[fill=ibm1!\i,below = of c1] (c1)  {};
\node[fill=ibm2!\i,right = of c1] (c21)  {};
\node[fill=ibm3!\i,right = of c21] (c31)  {};
\node[fill=ibm4!\i,right = of c31] (c41)  {};
\node[fill=ibm5!\i,right = of c41] (c51)  {};
\node[fill=black!\i,text=white,right = of c51] (c61) {} ;
\node[right = of c61] (c71) {$\i$\%} ;

};
\end{tikzpicture}
\caption{Cores da paleta sugerida pela IBM para daltonismo, com atenuação na cor para branco.}
\end{figure}



\begin{figure}[hbt]
\centering
\begin{tikzpicture}[every node/.style = {minimum width=2cm,minimum height=1em,draw,font=\tiny,,draw,node distance=0cm,align=center}]
\node[fill=ibm1] (c1) at (0,0) {\#648fff};
\node[fill=ibm2,right = of c1] (c2)  {\#785ef0};
\node[fill=ibm3,right = of c2] (c3)  {\#dc267f};
\node[fill=ibm4,right = of c3] (c4)  {\#fe6100};
\node[fill=ibm5,right = of c4] (c5)  {\#ffb000};
\node[fill=black,text=white,right = of c5] (c6) {\#000000};
\node[fill=white,right = of c6] (c7) {\#ffffff};
\foreach \i in {100,95,...,0} {
\node[fill=ibm1!\i!black,below = of c1] (c1)  {};
\node[fill=ibm2!\i!black,right = of c1] (c21)  {};
\node[fill=ibm3!\i!black,right = of c21] (c31)  {};
\node[fill=ibm4!\i!black,right = of c31] (c41)  {};
\node[fill=ibm5!\i!black,right = of c41] (c51)  {};
\node[fill=black,text=white,right = of c51] (c61) {$\i$\%} ;
\node[fill=white!\i!black,right = of c61] (c71) {} ;
};

\end{tikzpicture}
\caption{Cores da paleta sugerida pela IBM para daltonismo, com atenuação na cor para preto.}
\end{figure}


\begin{figure}[hbt]
\centering
\begin{verbatim}
\begin{tikzpicture}
\node[draw,text width=3cm] at (0,0) {A
\begin{itemize}
    \item item um
    \item item dois
\end{itemize}
};
\end{tikzpicture}
\end{verbatim}
\begin{tikzpicture}
\node[draw,text width=3cm] at (0,0) {A
\begin{itemize}
    \item item um
    \item item dois
\end{itemize}
};
\end{tikzpicture}
\caption{Itemize no nó precisa transformar o nó de mbox para minipage, e o text width faz isso. \url{https://tex.stackexchange.com/questions/213662/enumerate-within-tikz-node}}
\end{figure}
\chapter{3D and Fake 3D}

Some times I had to build some 3D drawings based on boxes, for example, to describe a Data Warehouse Cube. There is an easy solution that is to develop a basic cube subroutine and use them to build more complex figures.

O próximo código construi uma caixa calculando os pontos, como uma projeção de 3D em 3D.

\begin{lstlisting}[style=myLateX,caption=Cubo azul em Fake 3D]
\newcommand{\drawbox}[5]{
    \pgfmathsetmacro \angle {30}
    \pgfmathsetmacro \xd {{2/3*cos(\angle)*#5}}
    \pgfmathsetmacro \yd {{2/3*sin(\angle)*#5}}
    \pgfmathsetmacro \x {{#1-#5+(#2-#5)*(\xd)*#5}}
    \pgfmathsetmacro \y {{#3-#5+(#2-#5)*(\yd)*#5}}

    \draw[fill=#4] (\x,\y) --
    (\x+#5,\y) -- (\x+#5,\y+#5) --
    (\x,\y+#5) -- cycle;

    \draw[fill=#4] (\x,\y+#5) --
    (\x+\xd,\y+#5+\yd) --
    (\x+#5+\xd,\y+#5+\yd) --
    (\x+#5,\y+#5) -- cycle;

    \draw[fill=#4] (\x+#5,\y+#5) --
    (\x+#5+\xd,\y+#5+\yd) --
    (\x+#5+\xd,\y+\yd) --
    (\x+#5,\y) -- cycle;

\drawbox{}{1}{1}{blue}{1}
}
\end{lstlisting}

\newcommand{\drawbox}[5]{
    \pgfmathsetmacro \angle {30}
    \pgfmathsetmacro \xd {{2/3*cos(\angle)*#5}}
    \pgfmathsetmacro \yd {{2/3*sin(\angle)*#5}}
    \pgfmathsetmacro \x {{#1-#5+(#2-#5)*(\xd)*#5}}
    \pgfmathsetmacro \y {{#3-#5+(#2-#5)*(\yd)*#5}}

    \draw[fill=#4] (\x,\y) -- (\x+#5,\y) -- (\x+#5,\y+#5) -- (\x,\y+#5) -- cycle;

    \draw[fill=#4] (\x,\y+#5) -- (\x+\xd,\y+#5+\yd) -- (\x+#5+\xd,\y+#5+\yd) -- (\x+#5,\y+#5) -- cycle;

    \draw[fill=#4] (\x+#5,\y+#5) -- (\x+#5+\xd,\y+#5+\yd) -- (\x+#5+\xd,\y+\yd) -- (\x+#5,\y) -- cycle;
}


A simple blue cube, \autoref{fig:box3dfake}, can be easily drawn with:



\begin{figure}[hbt]
    \centering
  \begin{tikzpicture}
  \drawbox{}{1}{1}{blue}{1}
  \end{tikzpicture}
    \caption{A fake 3D cube}
    \label{fig:box3dfake}
\end{figure}

And a composite figure can use the order of drawing to build a cube made of cubes, as in \autoref{fig:box3dfake1}.

\begin{lstlisting}[style=myLaTeX,caption=A cube made of cubes]

\pgfmathsetmacro{\profX}{5}

  \drawbox{1}{\profX}{1}{green}{.5}
  \drawbox{1.5}{\profX}{1}{green}{.5}
  \drawbox{2}{\profX}{1}{green}{.5}

  \drawbox{1}{\profX}{1.5}{green!50}{.5}
  \drawbox{1.5}{\profX}{1.5}{green!50}{.5}
  \drawbox{2}{\profX}{1.5}{green!50}{.5}

  \drawbox{1}{\profX}{2}{green!25}{.5}
  \drawbox{1.5}{\profX}{2}{green!25}{.5}
  \drawbox{2}{\profX}{2}{green!25}{.5}

  \pgfmathsetmacro{\profX}{3}

  \drawbox{1}{\profX}{1}{red}{.5}
  \drawbox{1.5}{\profX}{1}{red}{.5}
  \drawbox{2}{\profX}{1}{red}{.5}

  \drawbox{1}{\profX}{1.5}{red!50}{.5}
  \drawbox{1.5}{\profX}{1.5}{red!50}{.5}
  \drawbox{2}{\profX}{1.5}{red!50}{.5}

  \drawbox{1}{\profX}{2}{red!25}{.5}
  \drawbox{1.5}{\profX}{2}{red!25}{.5}
  \drawbox{2}{\profX}{2}{red!25}{.5}

  \drawbox{1}{1}{1}{blue}{.5}
  \drawbox{1.5}{1}{1}{blue}{.5}
  \drawbox{2}{1}{1}{blue}{.5}

  \drawbox{1}{1}{1.5}{blue!50}{.5}
  \drawbox{1.5}{1}{1.5}{blue!50}{.5}
  \drawbox{2}{1}{1.5}{blue!50}{.5}

  \drawbox{1}{1}{2}{blue!25}{.5}
  \drawbox{1.5}{1}{2}{blue!25}{.5}
  \drawbox{2}{1}{2}{blue!25}{.5}

\end{lstlisting}

\begin{figure}[hbt]
    \centering
\begin{tikzpicture}
\pgfmathsetmacro{\profX}{5}

  \drawbox{1}{\profX}{1}{green}{.5}
  \drawbox{1.5}{\profX}{1}{green}{.5}
  \drawbox{2}{\profX}{1}{green}{.5}

  \drawbox{1}{\profX}{1.5}{green!50}{.5}
  \drawbox{1.5}{\profX}{1.5}{green!50}{.5}
  \drawbox{2}{\profX}{1.5}{green!50}{.5}

  \drawbox{1}{\profX}{2}{green!25}{.5}
  \drawbox{1.5}{\profX}{2}{green!25}{.5}
  \drawbox{2}{\profX}{2}{green!25}{.5}

  \pgfmathsetmacro{\profX}{3}

  \drawbox{1}{\profX}{1}{red}{.5}
  \drawbox{1.5}{\profX}{1}{red}{.5}
  \drawbox{2}{\profX}{1}{red}{.5}

  \drawbox{1}{\profX}{1.5}{red!50}{.5}
  \drawbox{1.5}{\profX}{1.5}{red!50}{.5}
  \drawbox{2}{\profX}{1.5}{red!50}{.5}

  \drawbox{1}{\profX}{2}{red!25}{.5}
  \drawbox{1.5}{\profX}{2}{red!25}{.5}
  \drawbox{2}{\profX}{2}{red!25}{.5}

  \drawbox{1}{1}{1}{blue}{.5}
  \drawbox{1.5}{1}{1}{blue}{.5}
  \drawbox{2}{1}{1}{blue}{.5}

  \drawbox{1}{1}{1.5}{blue!50}{.5}
  \drawbox{1.5}{1}{1.5}{blue!50}{.5}
  \drawbox{2}{1}{1.5}{blue!50}{.5}

  \drawbox{1}{1}{2}{blue!25}{.5}
  \drawbox{1.5}{1}{2}{blue!25}{.5}
  \drawbox{2}{1}{2}{blue!25}{.5}

  \end{tikzpicture}
    \caption{A fake 3D cube of cubes}
    \label{fig:box3dfake1}
\end{figure}

It is also possible to use fixed coordinates, such as the following code, that result in \autoref{fig:2cubes}.

\begin{lstlisting}[style=myLaTeX,caption=A cube with fixed 2d Coordinates]


\begin{tikzpicture}[scale=0.33] %[x={10.0pt},y={10.0pt}]
\draw[line width=2pt] (0,0) -- (0,10);
\draw[line width=2pt] (0,0) -- (10,0);
\draw[line width=2pt] (0,10) -- (10,10);
\draw[line width=2pt] (0,10) -- (10,10);
\draw[line width=2pt] (10,0) -- (10,10);
\draw[line width=2pt] (0,0) -- (3,5);
\draw[line width=2pt] (10,0) -- (13,5);
\draw[line width=2pt] (0,10) -- (3,15);
\draw[line width=2pt] (10,10) -- (13,15);
\draw[line width=2pt] (3,5) -- (13,5);
\draw[line width=2pt] (3,5) -- (3,15);
\draw[line width=2pt] (13,5) -- (13,15);
\draw[line width=2pt] (3,15) -- (13,15);
\draw[line width=1pt] (5,0) -- (5,10);
\draw[line width=1pt] (5,10) -- (8,15);
\draw[line width=1pt] (5,0) -- (8,5);
\draw[line width=1pt] (8,5) -- (8,15);
\draw[black, fill=blue,fill opacity=0.5] (5,0) -- (5,10) -- (8,15) -- (8,5) -- cycle;
\end{tikzpicture}% pic 1
\qquad % <----------------- SPACE BETWEEN PICTURES
% ou
%\hspace{3cm}
\begin{tikzpicture}[scale=0.33] %[x={10.0pt},y={10.0pt}]
\draw[line width=2pt] (0,0) -- (0,10);
\draw[line width=2pt] (0,0) -- (10,0);
\draw[line width=2pt] (0,10) -- (10,10);
\draw[line width=2pt] (0,10) -- (10,10);
\draw[line width=2pt] (10,0) -- (10,10);
\draw[line width=2pt] (0,0) -- (3,5);
\draw[line width=2pt] (10,0) -- (13,5);
\draw[line width=2pt] (0,10) -- (3,15);
\draw[line width=2pt] (10,10) -- (13,15);
\draw[line width=2pt] (3,5) -- (13,5);
\draw[line width=2pt] (3,5) -- (3,15);
\draw[line width=2pt] (13,5) -- (13,15);
\draw[line width=2pt] (3,15) -- (13,15);
\draw[line  width=1pt] (0,10) -- (10,0);
\draw[line  width=1pt] (0,10) -- (3,5);
\draw[line  width=1pt] (3,5) -- (10,0);
\draw[black, fill=blue,fill opacity=0.5] (0,10) -- (10,0) -- (3,5) -- cycle;
\draw[line  width=1pt] (3,15) -- (10,10);
\draw[line  width=1pt] (3,15) -- (13,5);
\draw[line  width=1pt] (10,10) -- (13,5);
\draw[black, fill=blue,fill opacity=0.5] (3,15) -- (10,10) -- (13,5) -- cycle;
\end{tikzpicture}% pic 2
\end{lstlisting}


\begin{figure}[hbt]
\begin{center}
\begin{tikzpicture}[scale=0.33] %[x={10.0pt},y={10.0pt}]
\draw[line width=2pt] (0,0) -- (0,10);
\draw[line width=2pt] (0,0) -- (10,0);
\draw[line width=2pt] (0,10) -- (10,10);
\draw[line width=2pt] (0,10) -- (10,10);
\draw[line width=2pt] (10,0) -- (10,10);
\draw[line width=2pt] (0,0) -- (3,5);
\draw[line width=2pt] (10,0) -- (13,5);
\draw[line width=2pt] (0,10) -- (3,15);
\draw[line width=2pt] (10,10) -- (13,15);
\draw[line width=2pt] (3,5) -- (13,5);
\draw[line width=2pt] (3,5) -- (3,15);
\draw[line width=2pt] (13,5) -- (13,15);
\draw[line width=2pt] (3,15) -- (13,15);
\draw[line width=1pt] (5,0) -- (5,10);
\draw[line width=1pt] (5,10) -- (8,15);
\draw[line width=1pt] (5,0) -- (8,5);
\draw[line width=1pt] (8,5) -- (8,15);
\draw[black, fill=blue,fill opacity=0.5] (5,0) -- (5,10) -- (8,15) -- (8,5) -- cycle;
\end{tikzpicture}% pic 1
\qquad % <----------------- SPACE BETWEEN PICTURES
% ou
%\hspace{3cm}
\begin{tikzpicture}[scale=0.33] %[x={10.0pt},y={10.0pt}]
\draw[line width=2pt] (0,0) -- (0,10);
\draw[line width=2pt] (0,0) -- (10,0);
\draw[line width=2pt] (0,10) -- (10,10);
\draw[line width=2pt] (0,10) -- (10,10);
\draw[line width=2pt] (10,0) -- (10,10);
\draw[line width=2pt] (0,0) -- (3,5);
\draw[line width=2pt] (10,0) -- (13,5);
\draw[line width=2pt] (0,10) -- (3,15);
\draw[line width=2pt] (10,10) -- (13,15);
\draw[line width=2pt] (3,5) -- (13,5);
\draw[line width=2pt] (3,5) -- (3,15);
\draw[line width=2pt] (13,5) -- (13,15);
\draw[line width=2pt] (3,15) -- (13,15);
\draw[line  width=1pt] (0,10) -- (10,0);
\draw[line  width=1pt] (0,10) -- (3,5);
\draw[line  width=1pt] (3,5) -- (10,0);
\draw[black, fill=blue,fill opacity=0.5] (0,10) -- (10,0) -- (3,5) -- cycle;
\draw[line  width=1pt] (3,15) -- (10,10);
\draw[line  width=1pt] (3,15) -- (13,5);
\draw[line  width=1pt] (10,10) -- (13,5);
\draw[black, fill=blue,fill opacity=0.5] (3,15) -- (10,10) -- (13,5) -- cycle;
\end{tikzpicture}% pic 2
\end{center}
    \caption{Two cubes and 3 planes.}
    \label{fig:2cubes}
\end{figure}

\section{Trying to draw a Ellisoid}

\begin{figure}[htb]
\begin{tikzpicture}[x={(1cm,0cm)},y={(0cm,1cm)},z={(0.8cm,0.6cm)}]
% ellipsoid
\draw[gray] (0,0,0) arc (90:270:1 and 2);
\draw[dashed,gray] (0,0,0) arc (90:-90:1 and 2);
\draw[gray] (0,0,0) arc (270:450:1 and 2);
\draw[dashed,gray] (0,0,0) arc (270:90:1 and 2);
\draw[gray] (1,0,0) arc (0:360:1 and 0.5);
% axes
\draw[->] (0,0,0) -- (3,0,0) node[anchor=north east]{$x$};
\draw[->] (0,0,0) -- (0,2,0) node[anchor=north west]{$y$};
\draw[->] (0,0,0) -- (0,0,3) node[anchor=south]{$z$};
\end{tikzpicture}
\caption{Doido que eu devia arrumar}
\end{figure}

\begin{figure}[htb]
\centering
\begin{tikzpicture}[baseline=(current bounding box.center),x=0.5cm,y=0.5cm,z=0.3cm]
  \coordinate (O) at (0,0,0);
  \coordinate (A) at (6,0,0);
  \coordinate (B) at (0,3,0);
  \coordinate (C) at (0,0,4);

  \draw[-Latex] (O) -- (A) node[above] {$x$};
  \draw[-Latex] (O) -- (B) node[above] {$y$};
  \draw[-Latex] (O) -- (C) node[right] {$z$};

  \pgfmathsetmacro{\a}{4}
  \pgfmathsetmacro{\b}{2}
  \pgfmathsetmacro{\c}{1}

  \draw[ball color=blue!40,opacity=0.5] (O) ellipse ({\a} and {\b});

  \draw[thick,blue,-latex] (O) -- ({\a},0,0) node[anchor=north east]{$a$};
  \draw[thick,blue,-latex] (O) -- (0,{\b},0) node[anchor=north west]{$b$};
  \draw[thick,blue,-latex] (O) -- (0,0,{\c}) node[anchor=south]{$c$};

\begin{scope}[canvas is xz plane at y=0]
        \draw[dashed] (0,0) circle[x radius=\a, y radius=\c];
    \end{scope}

    \begin{scope}[canvas is xy plane at z=0]
        \draw[dashed] (0,0) circle[x radius=\a, y radius=\b];
    \end{scope}

    \begin{scope}[canvas is yz plane at x=0]
        \draw[dashed] (0,0) circle[x radius=\b, y radius=\c];
    \end{scope}

\end{tikzpicture}
\caption{Usa ball e fica bonitinho}
\end{figure}

\begin{figure}[htb]
\centering

\begin{tikzpicture}[scale=2, x={(1cm,0cm)}, y={(0cm,1cm)}, z={(0.5cm,0.5cm)}]
    \pgfmathsetmacro\a{1}
    \pgfmathsetmacro\b{0.7}
    \pgfmathsetmacro\c{0.5}

    \begin{scope}[canvas is xz plane at y=0]
        \draw[dashed] (0,0) circle[x radius=\a, y radius=\c];
    \end{scope}

    \begin{scope}[canvas is xy plane at z=0]
        \draw[dashed] (0,0) circle[x radius=\a, y radius=\b];
    \end{scope}

    \begin{scope}[canvas is yz plane at x=0]
        \draw[dashed] (0,0) circle[x radius=\b, y radius=\c];
    \end{scope}

    \draw[->] (-1.5,0,0) -- (1.5,0,0) node[right]{$x$};
    \draw[->] (0,-1.5,0) -- (0,1.5,0) node[above]{$y$};
    \draw[->] (0,0,-1.5) -- (0,0,1.5) node[below left]{$z$};

    \pgfmathsetmacro\anglex{30}
    \pgfmathsetmacro\angley{45}

    \begin{scope}[rotate around x=\anglex]
        \begin{scope}[rotate around y=\angley]
            \draw[fill=blue!50, opacity=0.5] (0,0,0) plot[variable=\t, domain=0:2*pi] ({\a*cos(deg(\t))},{\b*sin(deg(\t))},0) plot[variable=\t, domain=0:2*pi] ({\a*cos(deg(\t))},0,{\c*sin(deg(\t))}) plot[variable=\t, domain=0:2*pi] (0,{\b*cos(deg(\t))},{\c*sin(deg(\t))});
        \end{scope}
    \end{scope}
\end{tikzpicture}
\caption{Consertado por mim}
\end{figure}

\begin{figure}[htb]
\centering
\begin{tikzpicture}
\begin{axis}
[view={135}{20},%colormap/blackwhite,
axis lines=center, axis on top,ticks=none,
set layers=default,axis equal,
xlabel={$x$}, ylabel={$y$}, zlabel={$z$},
xlabel style={anchor=south east},
ylabel style={anchor=south west},
zlabel style={anchor=south west},
enlargelimits,
tick align=inside,
domain=0:2.00,
samples=20,
z buffer=sort,
]
\addplot3 [surf,opacity=0.4,domain=-1:0,
domain y=0:360] ({sin(y)*sqrt(1-x^2)},{2*cos(y)*sqrt(1-x^2)},{x});
\addplot3 [surf,opacity=0.4,domain=0:1,
domain y=0:360,on layer=axis foreground] ({sin(y)*sqrt(1-x^2)},{2*cos(y)*sqrt(1-x^2)},{x});
\end{axis}
\end{tikzpicture}
\caption{Quente que eu achei em um site}
% https://tex.stackexchange.com/questions/417491/draw-an-ellipsoid}
\end{figure}

\begin{figure}[htb]
\centering
\begin{tikzpicture}[scale=2]
  \pgfmathsetmacro\a{2}
  \pgfmathsetmacro\b{1.5}
  \pgfmathsetmacro\c{1}
  \pgfmathsetmacro\x{3} % x-axis scale factor

  % ellipsoid coordinates
  \foreach \theta in {0,10,...,170}
    \foreach \phi in {0,10,...,360}
    {
      \pgfmathsetmacro\x{\a*sin(\theta)*cos(\phi)}
      \pgfmathsetmacro\y{\b*sin(\theta)*sin(\phi)}
      \pgfmathsetmacro\z{\c*cos(\theta)}
      \coordinate (e) at (\x,\y,\z);
      \pgfmathsetmacro\rotX{30} % rotation angle in x
      \pgfmathsetmacro\rotY{45} % rotation angle in y
      \path (e) [rotate around x=\rotX, rotate around y=\rotY] coordinate (e');
      \draw[black!50] (e')--++(0,0.02,0) (e')--++(0,-0.02,0) (e')--++(0,0,0.02) (e')--++(0,0,-0.02);
    }

  % x-axis
  \draw[red, very thick, -latex] (0,0,0) -- (\a*\x,0,0);
  \node[red, above] at (\a*\x/2,0,0) {$x$};

  % y-axis
  \draw[green!70!black, very thick, -latex] (0,0,0) -- (0,\b,0);
  \node[green!70!black, right] at (0,\b/2,0) {$y$};

  % z-axis
  \draw[blue, very thick, -latex] (0,0,0) -- (0,0,\c);
  \node[blue, above] at (0,0,\c/2) {$z$};
\end{tikzpicture}
\caption{Não sei como o Chat GPT gerou isso}
\end{figure}

\begin{figure}[htb]
\centering
\begin{tikzpicture}
\begin{axis}
[view={135}{20},colormap={blue}{
            color=(blue) color=(blue)
        },
axis lines=center,axis on top,
ticks=none,set layers=default,axis equal,
xlabel={$x$}, ylabel={$y$}, zlabel={$z$},
xlabel style={anchor=south east},
ylabel style={anchor=south west},
zlabel style={anchor=south west},
enlargelimits,
tick align=inside,
domain=0:2.00,
samples=20,
z buffer=sort,
]
\addplot3 [surf,draw=blue!30,fill=white,domain=-1:1,samples=20,
domain y=00:180] ({x},{-2*cos(y)*sqrt(1-x^2)},{-sin(y)*sqrt(1-x^2)});
\addplot3 [surf,draw=blue!30,fill=white,domain=-1:1,
domain y=0:180,on layer=axis foreground] ({x},{2*cos(y)*sqrt(1-x^2)},{sin(y)*sqrt(1-x^2)});
\end{axis}
\end{tikzpicture}
\caption{Bem parecido com que eu quero}
\end{figure}

\begin{figure}[htb]
\centering
\begin{tikzpicture}[remember picture]
%\pgfsetlayers{pre main,main,axis foregound}
\begin{axis}
[view={135}{20},colormap={blue}{
            color=(cyan) color=(cyan)
        },axis lines=none,axis equal,set layers=standard,
enlargelimits,domain=0:2,samples=20, z buffer=sort,
]

\pgfonlayer{axis background}
\draw[-] (axis cs:0,0,-1.5)--(axis cs:0,0,-1);
\draw[-] (axis cs:0,-3,0)--(axis cs:0,-2,0);
\draw[-] (axis cs:-2,0,0)--(axis cs:-1,0,0);
\addplot3[draw=none] coordinates{(0,0,-2) (0,0,2)};
\addplot3[draw=none] coordinates{(0,-2.5,0) (0,2.8,0)};
\endpgfonlayer
\pgfonlayer{main}
\addplot3 [surf,draw=cyan,fill=white,domain=0:1,samples=20,
domain y=00:180] ({x},{-2*cos(y)*sqrt(1-x^2)},{-sin(y)*sqrt(1-x^2)});
\addplot3 [surf,draw=cyan,fill=white,domain=0:1,
domain y=0:180,on layer=axis foreground] ({x},{2*cos(y)*sqrt(1-x^2)},{sin(y)*sqrt(1-x^2)});
\coordinate(x1) at (axis cs:1,0,0);
\coordinate(x2) at (axis cs:1.5,0,0);
\coordinate(y1) at (axis cs:0,1.9,0);
\coordinate(y2) at (axis cs:0,2.5,0);
\coordinate(z1) at (axis cs:0,0,0.9);
\coordinate(z2) at (axis cs:0,0,1.5);
\endpgfonlayer
\end{axis}
\end{tikzpicture}
\begin{tikzpicture}[remember picture,overlay]
\draw[->] (x1)--(x2)node[left]{$x$};
\draw[->] (y1)--(y2)node[right]{$y$};
\draw[->] (z1)--(z2)node[right]{$z$};
\end{tikzpicture}
\caption{Novamente parecido com que eu quero}
\end{figure}

\begin{figure}

\begin{center}
\begin{tikzpicture}[baseline=(current bounding box.center),x=0.5cm,y=0.5cm,z=0.3cm]
  \coordinate (O) at (0,0,0);
  \coordinate (A) at (6,0,0);
  \coordinate (B) at (0,3,0);
  \coordinate (C) at (0,0,4);

  \draw[-Latex] (O) -- (A) node[above] {$x$};
  \draw[-Latex] (O) -- (B) node[above] {$y$};
  \draw[-Latex] (O) -- (C) node[right] {$z$};

\begin{scope}[rotate around y=30,rotate around z=-30]

  \pgfmathsetmacro{\a}{4}
  \pgfmathsetmacro{\b}{2}
  \pgfmathsetmacro{\c}{1}



  %ellipse ({\a} and {\b});

  \draw[thick,blue,-latex] (O) -- ({\a},0,0) node[anchor=north east]{$a$};
  \draw[thick,blue,-latex] (O) -- (0,{\b},0) node[anchor=north west]{$b$};
  \draw[thick,blue,-latex] (O) -- (0,0,{\c}) node[anchor=south]{$c$};

    \begin{scope}[canvas is xz plane at y=0]
    \draw[dashed] (O) circle[x radius=\a, y radius=\c];
    \end{scope}
    \begin{scope}[canvas is xy plane at z=0]
      \draw[ball color=blue!40,opacity=0.5] (O)  circle[x radius=\a, y radius=\b];
        \draw[dashed] (O) circle[x radius=\a, y radius=\b];
    \end{scope}
    \begin{scope}[canvas is yz plane at x=0]
        \draw[dashed] (O) circle[x radius=\b, y radius=\c];
    \end{scope}

\end{scope}

\end{tikzpicture}
\end{center}

\caption{Rotação por escopo!}
\end{figure}


\chapter{Desenhos de Negócios}


You can draw a triangle by trial and error, as I did to find the best point for putting the top edge in \autoref{fig:3angulo}, with single command \verb|\draw|, as shown in the following code.

\begin{verbatim}
\begin{tikzpicture}
  \draw (-2,0) node[anchor=north]{\Large Prazo}   -- (2,0) node[anchor=north]{\Large Custo}  -- (0,3.46) node[anchor=south] {\Large Escopo}   -- cycle;
\end{tikzpicture}    
\end{verbatim}

\begin{figure}[hbt]
    \centering
\begin{tikzpicture}
  \draw (-2,0) node[anchor=north]{\Large Prazo}   -- (2,0) node[anchor=north]{\Large Custo}  -- (0,3.46) node[anchor=south] {\Large Escopo}   -- cycle;
\end{tikzpicture}
    \caption{Triangle}
    \label{fig:3angulo}
\end{figure}

Another figure done by deciding where the edges should be \textit{a priori}, resulting in \autoref{fig:pyramid}. In this code, there is a good example of using \verb|foreach| to achieve a result. Also, the use of \verb|intersections| to calculate where a point is.

\begin{verbatim}
\begin{tikzpicture}

\coordinate (A) at (-3.5,0) {};
\coordinate (B) at ( 3.5,0) {};
\coordinate (C) at (0,6) {};

\path[name path=AC,draw=none] (A) -- (C);
\path[name path=BC,draw=none] (B) -- (C);

\filldraw[draw=black, ultra thick,fill=white] 
(A) -- (B) -- (C) -- cycle ;

\foreach \y/\A in {0/Despejo Controlado,
                   1/Aterro ou Incineração,
                   2/Reciclagem,
                   3/Reuso,
                   4/\parbox{3cm}{\centering
                   Redução}} 
    {
    \path[draw=none, very thick, dashed, name
    path=horiz] (A|-0,\y) -- (B|-0,\y);
    \draw[draw=black, very thick, dashed, 
          name intersections={of=AC and horiz,by=P},
          name intersections={of=BC and horiz,by=Q}]
          (P) -- (Q)
          node[midway,above,font=\bfseries\scshape,
          color=black] {\A};
}

\node[single arrow,rotate=90,draw=black,minimum
height=6cm] at (-4,3) {melhor opção};

\end{tikzpicture}    
\end{verbatim}


\begin{figure}[hbt]
\centering
\begin{tikzpicture}

\coordinate (A) at (-3.5,0) {};
\coordinate (B) at ( 3.5,0) {};
\coordinate (C) at (0,6) {};

\path[name path=AC,draw=none] (A) -- (C);
\path[name path=BC,draw=none] (B) -- (C);

\filldraw[draw=black, ultra thick,fill=white] (A) -- (B) -- (C) -- cycle ;

\foreach \y/\A in {0/Despejo Controlado,
                   1/Aterro ou Incineração,
                   2/Reciclagem,
                   3/Reuso,
                   4/\parbox{3cm}{\centering Redução}} 
    {
    \path[draw=none, very thick, dashed, name path=horiz] (A|-0,\y) -- (B|-0,\y);
    \draw[draw=black, very thick, dashed, 
          name intersections={of=AC and horiz,by=P},
          name intersections={of=BC and horiz,by=Q}] (P) -- (Q)
          node[midway,above,font=\bfseries\scshape,color=black] {\A};
}

\node[single arrow,rotate=90,draw=black,minimum height=6cm] at (-4,3) {melhor opção};

\end{tikzpicture}
\caption{A tipical pyramid of concepts.}
\label{fig:pyramid}
\end{figure}

\chapter{Transparences, Shades}

\tikzfading[name=fade out,
            inner color = transparent!0,
            outer color = transparent!70]
\begin{figure}[hbt]
\centering
\def\firstcircle{(0,0) circle (1.5cm)}
\def\secondcircle{(0:2cm) circle (1.5cm)}
\begin{tikzpicture}[opacity=0.5]
 \fill [path fading=fade out,
        fading transform={xshift=0.3cm,
        yshift=0.5cm}] \firstcircle;
\end{tikzpicture}
    \caption{Shading, Fdding and Transform}
\end{figure}




\begin{figure}[hbt]
\centering
\def\firstcircle{(0,0) circle (1.5cm)}
\def\secondcircle{(0:2cm) circle (1.5cm)}
\begin{tikzpicture}[blend group=multiply]
 \fill [red,path fading=fade out] \firstcircle;
 \fill [blue,path fading=fade out] \secondcircle;
    \draw \firstcircle node[below] {$A$}
          \secondcircle node[below=1.5cm] {$B$};
    \node[anchor=south] at (current bounding box.north) {$A \cup B$};

\end{tikzpicture}
    \caption{Multiplicando cores fading}
\end{figure}

\chapter{Graphs and Trees}

In this chapter there are many exemples of using nodes and paths to draw diagrams.

\begin{figure}[hbt]
    \centering
   \begin{tikzpicture}
\node (shape) at (0,2) [draw] {class Shape};
\node (rect) at (-2,0) [draw] {class Rectangle};
\node (circle) at (2,0) [draw] {class Circle};
\node (ellipse) at (6,0) [draw] {class Ellipse};

\draw (node cs:name=ellipse,anchor=north) |- (0,1);
\draw (node cs:name=circle,anchor=north)
  |- (0,1);

\draw [arrows = -{Triangle[open, angle=60:3mm]}]
(node cs:name=rect,anchor=north) |- (0,1) -| (node cs:name=shape,anchor=south);
\end{tikzpicture}
    \caption{Caption}
    \label{fig:my_1label}
\end{figure}


\begin{figure}
    \centering
\begin{tikzpicture}[shorten >=1pt,node distance=2cm,on grid]
\node[state,initial] (q_0) {$q_0$};
\node[state] (q_1) [right=of q_0] {$q_1$};
\node[state](q_2) [right=of q_1] {$q_2$};
\node[state,accepting](q_3) [right=of q_2] {$q_3$};
\path[->] (q_0) edge node [above] {0} (q_1)
edge [loop above] node {1} ()
edge [bend left] node [above] {2} (q_2)
edge [bend right] node [below] {3} (q_3)
(q_1) edge node [above] {4} (q_2);
\path[->] (q_2) edge node [above] {5} (q_3);
\end{tikzpicture}
    \caption{Based on the   topathas manual.}

\end{figure}


\begin{figure}
    \centering
    \begin{tikzpicture}
[every node/.style={node distance=1cm and 0.3cm}]
\node (St1) at (0,0) {\{stock, broth\}};
\node (St2) [right  =  of St1] {\{stock, breed\}};
\node (St3) [right = of St2] {\{stock, share\}};
\node (D) [above = of St1] {\{soup\}};
\node (V) [above = of St2] {\{variety\}};
\node (A) [above = of St3] {\{asset\}};
\node (B) [below = of St1] {\{beef broth, beef stock\}};

\draw[-Latex] (St1) -- node [midway,right] {\tiny is-a} (D);
\draw[-Latex] (St2) -- node [midway,right] {\tiny is-a} (V);
\draw[-Latex] (St3) -- node [midway,right] {\tiny is-a} (A);
\draw[-Latex] (St1) -- node [midway,right] {\tiny like-a} (B);
\draw[-Latex] ([xshift=2em]B.north west) -- node [midway,right] {\tiny is-a} ([xshift=-2em]St1.south);
\end{tikzpicture}
    \caption{Relations from Wordnet}
    \label{fig:wordnet1}
\end{figure}

\begin{figure}
    \centering
    \begin{tikzpicture}
    \useasboundingbox (1,-3) rectangle (5,4);
    \scope[transform canvas={scale=.6}]
         % Your actual drawing

    \begin{scope}
        [
            grow=right,
            level 1/.style={sibling distance=18em,level distance=2.5cm},
            level 2/.style={sibling distance=14em,level distance=4.5cm}
            ,
            every node/.style =
            {   shape=rectangle, rounded corners,
                draw, align=center,
                fill=white,
                font=\Large,
                inner sep=5pt % espaço entre texto e borda
            },
            every annotation/.style={rectangle,
            rounded corners=0mm,
            font=\normalsize}
        ]
  \node (inc) {Incerteza} % root
    child { node {Nebulosidade}
        child { node[annotation]
                {Faltam distinções precisas ou definidas
                \begin{itemize}\setlength\itemsep{0em}
                    \item Vago
                    \item Confuso
                    \item Falta de Clareza
                    \item Indistinto
                    \item Impreciso
                \end{itemize}
                }
        }
    }
    child { node (amb) {Ambiguidade}
    child { node {Discordância}
         child { node[annotation]
                {Desacordo ao escolher uma alternativa
                 \begin{itemize}\setlength\itemsep{0em}
                    \item Dissonância
                    \item Incongruência
                    \item Discrepância
                    \item Conflito
                \end{itemize}
                }
        }
    }
    child { node {Não-especificidade}
    child { node[annotation]
                {Alternativas não especificadas
                 \begin{itemize}\setlength\itemsep{0em}
                    \item Variedade
                    \item Generalidade
                    \item Diversividade
                    \item Equívoco
                    \item Imprecisão
                \end{itemize}
                }
        }
    }
    };
    \end{scope}

      \begin{scope}[every annotation/.style={fill=gray!20,text=black}]
      \node [annotation,above left] at (amb.north)
{relacionamentos 1 para muitos};
\end{scope}
    \endscope
\end{tikzpicture}
    \caption{Exemplo de uso de Transform Canvas, mas que não deu certo no documento em que foi usado, pois gerou outras mudanças}

\end{figure}


\begin{figure}[hbt]
\centering
\begin{tikzpicture}
    \tikzmath{  \x1 = 0 ;
                \x5 = 8 ;
                \y1 = 0 ;
                \y2 = 1 ;
                \y3 = 2 ;
                \x3 = \x5 /2 ;
                \x2 = \x3 /2;
                \x4 = (\x3+\x5)/2;
                }

\node (R) at (\x1,\y2) {$R$};
\node (Q1) at (\x2,\y3) {$\underline{Q}$};
\node (D1) at (\x2,\y1) {$\underline{D}$};
\node (Q2) at (\x3,\y3) {$Q$};
\node (D2) at (\x3,\y1) {$D$};
\node (Q3) at (\x4,\y3) {$Q^\prime$};
\node (D3) at (\x4,\y1) {$D^\prime$};
\node (VSR) at (\x5,\y2) {$\Re$};

\draw[-latex] (Q1) -- (R) ;
\draw[-latex] (D1) -- (R) ;
\draw[-latex] (Q1) -- (Q2) node[midway,above] {$\alpha_Q$};
\draw[-latex] (D1) -- (D2)
node[midway,above] {$\alpha_D$};
\draw[-latex] (Q2) -- (Q3) node[midway,above] {$\beta_Q$};
\draw[-latex] (D2) -- (D3)
node[midway,above] {$\beta_Q$};
\draw[-latex] (Q3) -- (VSR);
\draw[-latex] (D3) -- (VSR);
\end{tikzpicture}
\caption{Figure from IR, first try, using calculated absolute positions}
\label{fig:ir}
\end{figure}

\begin{figure}[hbt]
\centering
\begin{tikzpicture}
\node (R) at (0,0) {$R$};
\node (Q1) [above right = of R] {$\underline{Q}$};
\node (D1) [below right = of R] {$\underline{D}$};
\node (Q2) [right = of Q1] {$Q$};
\node (D2) [right = of D1]  {$D$};
\node (Q3) [right = of Q2]  {$Q^\prime$};
\node (D3) [right = of D2]  {$D^\prime$};
\node (VSR) [below right = of Q3] {$\Re$};

\draw[-latex] (Q1) -- (R) ;
\draw[-latex] (D1) -- (R) ;
\draw[-latex] (Q1) -- (Q2) node[midway,above] {$\alpha_Q$};
\draw[-latex] (D1) -- (D2)
node[midway,above] {$\alpha_D$};
\draw[-latex] (Q2) -- (Q3) node[midway,above] {$\beta_Q$};
\draw[-latex] (D2) -- (D3)
node[midway,above] {$\beta_Q$};
\draw[-latex] (Q3) -- (VSR);
\draw[-latex] (D3) -- (VSR);
\end{tikzpicture}
\caption{Figure from IR, second try, using relative positions.}
\label{fig:ir2}
\end{figure}




\begin{figure}
    \centering
\begin{tikzpicture}[
  tlabel/.style={pos=0.4,right=-1pt,font=\footnotesize\color{red!70!black}},
]
\node{S}
child {node {a}}
child {node {S}
  child {node {a}}
  child {node {S}
    child {node {$\varepsilon$}
      edge from parent node[tlabel,pos=0.2] {2}
    }
    edge from parent node[tlabel] {1}
  }
  child {node {B}
    child {node {B}
      child {node {b}
        edge from parent node[tlabel,pos=0.2] {5}
     }
      edge from parent node[tlabel] {4}
    }
    child {node {b}}
  }
    edge from parent node[tlabel] {1}
  }
child {node {B}
  child[missing] {}
  child[missing] {}
  child {node {b}
    edge from parent node[tlabel,pos=0.15,right=2pt] {5}
  }
};
\end{tikzpicture}    \caption{This tree is described in https://tex.stackexchange.com/questions/85112/drawing-a-syntax-tree-in-tikz}
    \label{fig:my_2label}
\end{figure}


\begin{figure}
centering
\resizebox{0.8\textwidth}{!}{
\begin{tikzpicture}[
  ->,
  >=stealth',
  edge from parent/.style={thick,draw=black!70,-latex},
  level 1/.style={sibling distance=8.5cm},
  level 2/.style={child anchor=north},
  level 3/.style={child anchor=north,sibling distance=2cm},
  level distance=3.5cm,
  auto,
  tree nodes/.style={
    shape=rectangle,
    rounded corners,
    thick,
    draw=black!75,
    minimum size=+5mm
  },
  and/.style     ={tree nodes, top color=blue!10, bottom color=white},
  or/.style      ={tree nodes, top color=red!10,  bottom color=white},
  terminal/.style={tree nodes, top color=green!5, bottom color=white},
  relation/.style={tree nodes, inner sep=2pt, fill=gray!20, font=\scriptsize},
  font=\footnotesize,
  every node/.append style={align=center}
]
\node              [and]                     {1 \\ 2 \\ 3}
  child {     node [and]      (appr)         {1 \\ 2 \\ 3}
    [sibling distance = 2cm]
    child {   node [terminal] (car)          {2}          }
    child {   node [relation] (rel)          {1 \\ 2 \\ 3}}
    child {   node [terminal] (zebra)        {1 \\ 2 \\ 3}}
  }
  child {     node [or]       (obstacle)     {1 \\ 2 \\ 3}
    [sibling distance = 6cm]
    child {   node [and]      (ped_on_zebra) {1 \\ 2 \\ 3}
      child { node [terminal] (ped)          {1}          }
      child { node [relation] (rel_1)        {1 \\ 2 \\ 3}}
      child { node [terminal] (zebra_2)      {1 \\ 2 \\ 3}}
    }
    child   { node [and]      (dog_on_zebra) {1 \\ 2 \\ 3}
      child { node [terminal] (dog)          {3}          }
      child { node [relation] (rel_2)        {1 \\ 2 \\ 3}}
      child { node [terminal] (zebra_1)      {1 \\ 2 \\ 3}}
    }
  };

\path [dashed,-]
  (rel) edge (zebra)
        edge (car)

  (rel_1) edge (ped)
          edge (zebra_2)

  (rel_2) edge (dog)
          edge (zebra_1);

\begin{scope}[shift={(7,1)}, node distance=+1cm, on grid, label position=right, start chain=ch going above]
  \foreach \sStyle/\tText in {relation/Relation, terminal/Terminal-Node, and/And-Node, or/Or-Node}
    \node[
     on chain=ch,
     \sStyle,
     label={= \tText}
    ] {};
\end{scope}
\end{tikzpicture}}
\caption{This tree is described in \url{https://tex.stackexchange.com/questions/123212/tikz-tree-some-childs-without-arrows}}
    \label{fig:my_3label}
\end{figure}

\begin{figure}[hbt]
\centering
\begin{tikzpicture}
\node {S}
    child {node {SN}
        child {node {Det$_m$}
            child {node {\textit{O}}}
        }
        child {node {N$_M$}
            child {node {\textit{menino}}}
        }
    }
    child {node[xshift=2cm] {SV}
    %child {node {}}
        child {node {SV}
            child {node {V$_t$}
                child {node {viu}}
            }
            child {node {SN}
                child {node {Det$_f$}
                    child {node {\textit{a}}}
                }
                child {node {N$_f$}
                    child {node {\textit{mulher}}}
                }
            }
        }
        child {node {SP}
            child {node[xshift=2cm] {P}
                child {node{com}
            }
            child {node {SN}
                child {node {Det$_m$}
                    child {node {o}
                    }
                }
                child {node {N$_m$}
                    child {node {binóculo}
                    }
                }
            }
        }
    }
};
\end{tikzpicture}
\caption{Primeira forma de interpretar sintaticamente a sentença ``O menino viu a mulher de binóculo'', onde o menino tem o binóculo, baseado em ...}
\label{fig:a1}
\end{figure}

\begin{figure}[hbt]
\centering
\begin{tikzpicture}
\node {S}
    child {node {SN}
        child {node {Det$_m$}
            child {node {\textit{O}}}
        }
        child {node {N$_M$}
            child {node {\textit{menino}}}
        }
    }
    child {node[xshift=1cm] {SV}
        child {node {V$_t$}
            child {node {viu}
            }
        }
        child {node {SN}
            child {node {Det$_f$}
                child {node {\textit{a}}
                }
            }
            child {node {N'$_f$}
                child {node {N$_f$}
                    child {node {\textit{mulher}}
                    }
                child {node {SP}
                    child {node {P}
                        child {node{com}
                        }
                    }
                    child {node[xshift=.5cm] {SN}
                        child {node {Det$_m$}
                            child {node {o}
                            }
                        }
                        child {node {N$_m$}
                            child {node {binóculo}
                            }
                        }
                    }
                }
            }
        }
    }
};
\end{tikzpicture}
\caption{Segunda forma de interpretar sintaticamente a sentença ``O menino viu a mulher de binóculo'', onde a mulher tem o binóculo, baseado em ...}
\label{fig:a2}
\end{figure}

\begin{figure}
    \centering
\begin{tikzpicture}
\begin{scope}
  \tikzset{edge from parent/.append style={draw=none},
  every tree node/.style={draw=none},level distance=2cm
  }
  \Tree [[.{Level 1} [.{Level 2} ]]]
\end{scope}
  \begin{scope}[xshift=2in]
\tikzset{every tree node/.style={draw,circle, minimum size=2em,fill=blech},
    level distance=2cm,sibling distance=1cm}
    \Tree[.\node (Root) {};
       [.4 11 {} ]
       [.5 6  {} ]
       ]
   \node [above=.25cm of Root] {Root};
\end{scope}
\end{tikzpicture}    \caption{https://tex.stackexchange.com/questions/153598/how-to-draw-empty-nodes-in-tikz-qtree}

\end{figure}


\begin{figure}
    \centering
\begin{tikzpicture}
[every node/.style={node distance=1cm and 0.3cm}]
\node (C) at (0,0) {\{Communicate\}};
\node (T) [below = of C] {\{talk\}};
\node (S) [below right = of T] {\{stammer\}};
\node (W) [below left = of T] {\{whisper\}};

\draw[-Latex] (C) -- (T);
\draw[-Latex] (T) -- (S);
\draw[-Latex] (T) -- (W);
\end{tikzpicture}

\begin{tikzpicture}
[every node/.style={node distance=1cm and 0.3cm}]
\node (C) at (0,0) {\{car\}};
\node (T) [below = of C] {\{engine\}};
\node (S) [below right = of T] {\{spark plug\}};
\node (W) [below left = of T] {\{cylinder\}};

\draw[Latex-] (C) -- node [midway,right] {\tiny is-part-of} (T);
\draw[Latex-] (T) -- node [midway,right] {\tiny is-part-of}(S);
\draw[Latex-] (T) -- node [midway,right] {\tiny is-part-of} (W);
\end{tikzpicture}
    \caption{Wordnet}
    \label{fig:wordnet2}
\end{figure}

\begin{figure}
    \centering
    \begin{tikzpicture}
\begin{scope}[
every node/.style = {node distance={10mm},
shape=rectangle, rounded corners,
                draw, align=center,
                fill=white,
                font=\Large,
                inner sep=8pt }
]
\node (NB) {\textbf{Nota Baixa}};
\node (PD) [below left = of NB] {\textbf{Prova Difícil}};
\node (NFLE) [right = of PD] {\textbf{Não Fiz Lista de Exercícios}};
\draw[-{Latex[length=3mm]}] (PD) -- (NB.south west);
\draw[-{Latex[length=3mm]}] (NFLE) -- (NB.south east);
\node (NE) [below = of PD] {\textbf{Não Estudei}} ;
\draw[-{Latex[length=3mm]}] (NE) -- (PD);
\node (NTT)  [below right = of NE] {\textbf{Não Tive Tempo}};
\draw[-{Latex[length=3mm]}] (NTT) -- (NE.south east);
\draw[-{Latex[length=3mm]}] (NTT) -- (NFLE);
\node (NMO)  [below  = of NTT,fill = yellow] {\textbf{Não Me Organizei}};
\draw[-{Latex[length=3mm]}] (NMO) -- (NTT);
\end{scope}
 \end{tikzpicture}
    \caption{Grafo - Diagrama de Causas Raiz Vertical}
    \label{fig:my_label12312}
\end{figure}

\begin{figure}
    \centering
  \begin{tikzpicture}
    \graph[spring layout] {
      A -> ["1"] B,
      A -> {C, D},
      C -> {B, D},
    };
  \end{tikzpicture}
    \caption{Auto layout}
    \label{fig:my_label5492}
\end{figure}


\section{Nodes With Lists Inside}

\begin{figure}
    \centering

    \begin{tikzpicture}
    \begin{scope}
        [->,>=stealth',
            grow=down,
            level 1/.style={sibling distance=12em,level distance=2cm},
            level 2/.style={sibling distance=12em,level distance=2cm,shorten >=0cm},
            level 3/.style={sibling distance=10em,level distance=2cm,shorten >=0cm},
            every node/.style =
            {   shape=rectangle,
                rounded corners,
                draw,
                align=center,
                fill=white,
                font=\large,
                inner sep=5pt % espaço entre texto e borda
            },
            every annotation/.style={shape=rectangle,
            rounded corners=0mm,
            font=\footnotesize}
        ]
  \node (inc) {\textbf{Incerteza}} % root
    child { node {Nebulosidade}
        child { node[annotation,align=left]
                {Faltam distinções precisas ou definidas
                \begin{itemize}\setlength\itemsep{-.5em}
                    \item Vago
                    \item Confuso
                    \item Falta de Clareza
                    \item Indistinto
                    \item Impreciso
                \end{itemize}
                }
        }
    }
    child { node (amb) {Ambiguidade}
    child { node[yshift=-2cm] {Discordância}
         child { node[annotation,align=left]
                {Desacordo ao escolher uma alternativa
                 \begin{itemize}\setlength\itemsep{-.5em}
                    \item Dissonância
                    \item Incongruência
                    \item Discrepância
                    \item Conflito
                \end{itemize}
                }
        }
    }
    child { node {Não-especificidade}
    child { node[annotation,align=left]
                {Alternativas não especificadas
                 \begin{itemize}\setlength\itemsep{-.5em}
                    \item Variedade
                    \item Generalidade
                    \item Diversividade
                    \item Equívoco
                    \item Imprecisão
                \end{itemize}
                }
        }
    }
    };
    \end{scope}

      \begin{scope}[every annotation/.style={fill=gray!20,text=black}]
      \node [annotation,above right ,align=center,text width=2cm] at (amb.north)
{relacionamentos \\ 1 para muitos};
\end{scope}

\end{tikzpicture}




    \caption{Graph Tree}



\end{figure}

\begin{figure}
    \centering
    \begin{tikzpicture}
    \begin{scope}
        [   ->,
            grow=right,
            level 1/.style={sibling distance=12em,level distance=.7cm},
            level 2/.style={sibling distance=10em,level distance=4cm,}
            ,
            every node/.style =
            {   shape=rectangle,
                rounded corners,
                draw,
                align=center,
                fill=white,
                font=\large,
                inner sep=5pt % espaço entre texto e borda
            },
            every annotation/.style={rectangle,
            rounded corners=0mm,
            font=\footnotesize}
        ]
  \node (inc) {\textbf{Incerteza}} % root
    child { node {Nebulosidade}
        child { node[annotation,align=left]
                {Faltam distinções precisas ou definidas
                \begin{itemize}\setlength\itemsep{-.5em}
                    \item Vago
                    \item Confuso
                    \item Falta de Clareza
                    \item Indistinto
                    \item Impreciso
                \end{itemize}
                }
        }
    }
    child { node (amb) {Ambiguidade}
    child { node {Discordância}
         child { node[annotation,align=left]
                {Desacordo ao escolher uma alternativa
                 \begin{itemize}\setlength\itemsep{-.5em}
                    \item Dissonância
                    \item Incongruência
                    \item Discrepância
                    \item Conflito
                \end{itemize}
                }
        }
    }
    child { node {Não-especificidade}
    child { node[annotation,align=left]
                {Alternativas não especificadas
                 \begin{itemize}\setlength\itemsep{-.5em}
                    \item Variedade
                    \item Generalidade
                    \item Diversividade
                    \item Equívoco
                    \item Imprecisão
                \end{itemize}
                }
        }
    }
    };
    \end{scope}

      \begin{scope}[every annotation/.style={fill=gray!20,text=black}]
      \node [annotation,above left ,align=center,text width=2cm] at (amb.north)
{relacionamentos \\ 1 para muitos};
\end{scope}

\end{tikzpicture}

    \caption{Caption}

\end{figure}


\section{Positioning and TikzMath}


\section{Using arrays and foreach}

The next figure describes a vector where each cell points to a linked list. It is first necessary to define the styles of cells and links:
\begin{verbatim}
\tikzset{
node of list/.style = {
             draw,
             fill=orange!20,
             minimum height=6mm,
             minimum width=6mm,
             node distance=6mm
   },
link/.style = {
     -stealth,
     shorten >=1pt
     },
array element/.style = {
    draw, fill=white,
    minimum width = 6mm,
    minimum height = 10mm
  }
}
\end{verbatim}

\tikzset{
node of list/.style = {
             draw,
             fill=orange!20,
             minimum height=6mm,
             minimum width=6mm,
             node distance=6mm
   },
link/.style = {
     -stealth,
     shorten >=1pt
     },
array element/.style = {
    draw, fill=white,
    minimum width = 6mm,
    minimum height = 10mm
  }
}

Then, we will use a command that builds a linked list using \verb|foreach|.
\begin{verbatim}
\def\LinkedList#1{%
  \foreach \element in \list {
     \node[node of list, right = of aux, name=ele] {\element};
     \node[node of list, name=aux2, anchor=west] at ([xshift=-.4pt] ele.east) {};
     \draw[link] (aux) -- (ele);
     \coordinate (aux) at (aux2);
   }
   \fill (aux) circle(2pt);
}
\end{verbatim}

\def\LinkedList#1{%
  \foreach \element in \list {
     \node[node of list, right = of aux, name=ele] {\element};
     \node[node of list, name=aux2, anchor=west] at ([xshift=-.4pt] ele.east) {};
     \draw[link] (aux) -- (ele);
     \coordinate (aux) at (aux2);
   }
   \fill (aux) circle(2pt);
}

Finally, the following code results in \autoref{fig:linkedlist}.

\begin{verbatim}
\begin{tikzpicture}
\foreach \index/\list in {
.2/{(3,11),(16,24),null},
.4/{(4,10),(17,23),null},
.6/{(5,9),(18,22),null},
.8/{(6,8),(19,21),null},
1/{7,20,null}} {
   \node[array element] (aux) at (0,-\index*5) {\index};
   \LinkedList{\list}
}
\end{tikzpicture}
\end{verbatim}

\begin{figure}[h]
\centering
\begin{tikzpicture}
\foreach \index/\list in {
.2/{(3,11),(16,24),null},
.4/{(4,10),(17,23),null},
.6/{(5,9),(18,22),null},
.8/{(6,8),(19,21),null},
1/{7,20,null}} {
   \node[array element] (aux) at (0,-\index*5) {\index};
   \LinkedList{\list}
}
\end{tikzpicture}
    \caption{This figure uses a pre-defined commando to draw a linked list.}
\label{fig:linkedlist}
\end{figure}




\begin{figure}
    \centering
\begin{tikzpicture}
 \draw[fill] (0,0) circle (2pt) coordinate (a) node {A};
 \draw[fill] (5,0) circle (2pt) coordinate (b);
 \draw (a)  -- (b) ;
 \draw ($(a)+(0,1)$) node {X} -- ($(b)+(0,1)$) node {Y};
\end{tikzpicture}
    \caption{Exemplo de somar posições}
\end{figure}


\begin{figure}
    \centering
    \begin{tikzpicture}[every node/.style={circle,draw,fill=white},
    every path/.style={-{LaTeX},draw}]

\node (n2) at (0,0) {2};
\node[above left = of n2] (n1) {1};
\node[above right = of n2](n3) {3};
\node[below right = of n2] (n4) {4};
\node[below left = of n2]  (n5) {5};
\path (n1)  edge [bend right] (n2);
\path (n1) edge [bend left] (n3);
\path (n2) -- (n3);
\path (n2) -- (n4);
\path (n2) -- (n5);
\path (n3) edge [bend left] (n1);
\path (n4) -- (n3);
\path (n5) -- (n1);
\path (n5) -- (n4);



    \end{tikzpicture}
    \caption{Grafo}
    \label{fig:my_labeldasd1}
\end{figure}


\begin{figure}
    \centering
    \begin{tikzpicture}[every node/.style={circle,draw,fill=white},
    every path/.style={-,draw}]

\node at (0,0) (n4) {4};
\node[above left = of n4] (n1) {1};
\node[above right = of n4](n3) {3};

\node[right= of n4] (n2)  {2};
\node[above= of n4]  (n5) {5};
\path (n2) -- (n3);
\path (n1) -- (n4);
\path (n3) -- (n4);
\path (n4) -- (n5);
\path (n3) -- (n5);

    \end{tikzpicture}
    \caption{Grafo Co Citação}
    \label{fig:my_labeldasd22}
\end{figure}

\begin{figure}
\centering
\begin{verbatim}
 \begin{figure}
\centering
\shorthandoff{"}
\tikz \graph {
a ->["x"] b ->["y"'] c ->["z" red] d;
};
    \caption{Bug na interação do BABEL com o TIKZ precisa desligar o aspas com shorthandoff}
    \label{fig:my_labeldasd}
\end{figure}
\end{verbatim}
\shorthandoff{"}
\begin{tikzpicture}
 \graph {
a ->["x"] b ->["y"'] c ->["z" red] d;
};
\end{tikzpicture}

    \caption{Bug na interação do BABEL com o TIKZ precisa desligar o aspas com shorthandoff}
    \label{fig:my_labeldasd}
\end{figure}

\begin{figure}
    \centering
\begin{tikzpicture}
[every node/.style={circle,draw=ibm2,thick},
every edge/.style={->],draw=ibm3,thick}
]
    \graph[spring electrical  layout, coarsen = false,
    node distance=4mm] {d[fill=ibm3!50,electric charge=50] -> {a[fill=ibm3!35,electric charge=35],b[fill=ibm3,electric charge=100]};
    b -> {c[fill=ibm3!85,electric charge=85]};
    c -> b;
    f[fill=ibm3!30,electric charge=30] -> {b,e[fill=ibm3!70,electric charge=70]};
    e -> {d, b, f};
    g[fill=ibm3!10,electric charge=10]  -> {b,e};
    h[fill=ibm3!10,electric charge=10]  -> {b,e};
    i[fill=ibm3!10,electric charge=10]  -> {b,e};
    j[fill=ibm3!10,electric charge=10] -> {b,e};
    k[fill=ibm3!10,electric charge=10] -> e;
    m[fill=ibm3!10,electric charge=10] -> e;
    };
\end{tikzpicture}

    \caption{Grafo com força spring e variações elétricas}
    \label{fig:pgr123}
\end{figure}




\newcommand{\boxberthier}[5]{
\nodepart{one} #1
\nodepart{two} #2
\nodepart{three} #3
\nodepart{four} #4
\nodepart{five} #5
}
\begin{figure}
    \centering


\resizebox{0.8\textwidth}{!}{
\begin{tikzpicture}[
    box/.style =
    {   shape=rectangle split,
        draw, align=center,
        minimum width = 8em,
        minimum height = 26pt,
        rectangle split,
        rectangle split parts=5,
        rectangle split draw splits=false,
        /pgf/rectangle split ignore empty parts=true,
        rectangle split part align={center,left},
        rectangle split part fill={black!20,white}
     },
     every path/.style={-Latex}]

\node[box] (raiz) at (0,0) {
\boxberthier{Propriedade \\ do Documento}{Texto}{Links}{Multimída}{}};

\node[box,above right = of raiz] (classic)  {
\boxberthier{Modelos \\ Clássicos}{Booleano}{Vetors}{Probabilístico}{}};

\node[box,above right = of classic] (setthor)  {
\boxberthier{Baseados em \\ Teoria dos Conjuntos}{\textit{Fuzzy}}{Booleno Extendido}{Baseado em Conjutnos}{}};

\node[box,right = of classic] (algeb)  {
\boxberthier{Algebraicos}{Vetor Generalizado}{LSI}{Redes Neurais}{}};

\node[box,below right = of classic] (prob)  {
\boxberthier{Probabilísticos}{BM25}{Modelos de Linguagem}{Redes Bayesianas}{Divergência da Aleatoriedade}};

\node[box,below = of classic] (semi)  {
\boxberthier{Texto Semi \\ Estruturado}{Nós Próximos}{Baseados  em XML}{}{}};

\node[box,below = of semi] (web)  {
\boxberthier{Web}{\textit{Pagerank}}{\textit{HITS}}{}{}};

\node[box,below = of web] (multi)  {
\boxberthier{Multimídia}{Imagem}{Aúdio}{Música}{Vídeo}};

\path[draw] ([xshift=-5pt]raiz.two east) node{\bullet} -- (classic.west);
\path[draw] ([xshift=-5pt]raiz.two east) node{\bullet} -- (semi.west);
\path[draw] ([xshift=-5pt]raiz.three east) node{\bullet}-- (web.west);
\path[draw] ([xshift=-5pt]raiz.four east)node{\bullet} -- (multi.west);
\path[draw] ([xshift=-5pt]classic.two east) node{\bullet}-- (setthor.west);
\path[draw] ([xshift=-5pt]classic.three east)node{\bullet} -- (algeb.west);
\path[draw] ([xshift=-5pt]classic.four east) node{\bullet}-- (prob.west);
\end{tikzpicture}
}

    \caption{Taxonomia de Modelos de IR com posicionamento relativo}
    \label{fig:my_irtaxon}
    \end{figure}


\begin{figure}
\centering
\resizebox{.8\textwidth}{!}{
\begin{tikzpicture}[grow=right,
   level 1/.style={sibling distance=4em,level distance=5.3cm},
   level 2/.style={sibling distance=5em,level distance=5cm},
    every node/.style =
    {   shape=rectangle, rounded corners,
        draw, align=center,
        fill=white,
        minimum width = 8em,
        minimum height = 26pt,
     }  ]
\node {Propriedade \\do Documento} [grow=right]
    child {node {Texto \\ Estruturado \\ ou Clássico}
        child {node {Baseado em \\ Teoria \\ dos Conjuntos}}
        child {node {Albraicos}}
        child {node {Probabilísticos}}}
    child {node {Texto \\Semi-Estruturado}}
    child {node {Web}}
    child {node {Multimídia}};
\end{tikzpicture}
}
\caption{Taxonomia do Modelos de IR com child nodes}
\label{fig:mIRkss}
\end{figure}
\chapter{Cartesian Coordinates}



\section{Grids}

The following code allows for the creation of \autoref{fig:grid1} and \autoref{fig:grid2}

\begin{verbatim}
\makeatletter
\def\grd@save@target#1{%
  \def\grd@target{#1}}
\def\grd@save@start#1{%
  \def\grd@start{#1}}
\tikzset{
  grid with coordinates/.style={
    to path={%
      \pgfextra{%

        \edef\grd@@target{(\tikztotarget)}%
        \tikz@scan@one@point\grd@save@target\grd@@target\relax
        \edef\grd@@start{(\tikztostart)}%
        \tikz@scan@one@point\grd@save@start\grd@@start\relax

       \draw[minor help lines] (\tikztostart) grid (\tikztotarget);

        \draw[major help lines] (\tikztostart) grid (\tikztotarget);

        \grd@start

        \pgfmathsetmacro{\grd@xa}{\the\pgf@x/1cm}
        \pgfmathsetmacro{\grd@ya}{\the\pgf@y/1cm}

        \grd@target

        \pgfmathsetmacro{\grd@xb}{\the\pgf@x/1cm}
        \pgfmathsetmacro{\grd@yb}{\the\pgf@y/1cm}

        \pgfmathsetmacro{\grd@xc}{\grd@xa + \pgfkeysvalueof{/tikz/grid with coordinates/major step}}

        \pgfmathsetmacro{\grd@yc}{\grd@ya + \pgfkeysvalueof{/tikz/grid with coordinates/major step}}

        \foreach \x in {\grd@xa,\grd@xc,...,\grd@xb}
        \node[anchor=north] at (\x,\grd@ya) {\pgfmathprintnumber{\x}};

        \foreach \y in {\grd@ya,\grd@yc,...,\grd@yb}
        \node[anchor=east] at (\grd@xa,\y) {\pgfmathprintnumber{\y}};
      }
    }
  },
  minor help lines/.style={
    help lines,
    step=\pgfkeysvalueof{/tikz/grid with coordinates/minor step}
  },
  major help lines/.style={
    help lines,
    line width=\pgfkeysvalueof{/tikz/grid with coordinates/major line width},
    step=\pgfkeysvalueof{/tikz/grid with coordinates/major step}
  },
  grid with coordinates/.cd,
  minor step/.initial=.2,
  major step/.initial=1,
  major line width/.initial=2pt,
}
\makeatother
\end{verbatim}

\makeatletter
\def\grd@save@target#1{%
  \def\grd@target{#1}}
\def\grd@save@start#1{%
  \def\grd@start{#1}}
\tikzset{
  grid with coordinates/.style={
    to path={%
      \pgfextra{%

        \edef\grd@@target{(\tikztotarget)}%
        \tikz@scan@one@point\grd@save@target\grd@@target\relax
        \edef\grd@@start{(\tikztostart)}%
        \tikz@scan@one@point\grd@save@start\grd@@start\relax

       \draw[minor help lines] (\tikztostart) grid (\tikztotarget);

        \draw[major help lines] (\tikztostart) grid (\tikztotarget);

        \grd@start

        \pgfmathsetmacro{\grd@xa}{\the\pgf@x/1cm}
        \pgfmathsetmacro{\grd@ya}{\the\pgf@y/1cm}

        \grd@target

        \pgfmathsetmacro{\grd@xb}{\the\pgf@x/1cm}
        \pgfmathsetmacro{\grd@yb}{\the\pgf@y/1cm}

        \pgfmathsetmacro{\grd@xc}{\grd@xa + \pgfkeysvalueof{/tikz/grid with coordinates/major step}}

        \pgfmathsetmacro{\grd@yc}{\grd@ya + \pgfkeysvalueof{/tikz/grid with coordinates/major step}}

        \foreach \x in {\grd@xa,\grd@xc,...,\grd@xb}
        \node[anchor=north] at (\x,\grd@ya) {\pgfmathprintnumber{\x}};

        \foreach \y in {\grd@ya,\grd@yc,...,\grd@yb}
        \node[anchor=east] at (\grd@xa,\y) {\pgfmathprintnumber{\y}};
      }
    }
  },
  minor help lines/.style={
    help lines,
    step=\pgfkeysvalueof{/tikz/grid with coordinates/minor step}
  },
  major help lines/.style={
    help lines,
    line width=\pgfkeysvalueof{/tikz/grid with coordinates/major line width},
    step=\pgfkeysvalueof{/tikz/grid with coordinates/major step}
  },
  grid with coordinates/.cd,
  minor step/.initial=.2,
  major step/.initial=1,
  major line width/.initial=2pt,
}
\makeatother

The first example of grid construction generates \autoref{fig:grid1}.
\begin{verbatim}
\begin{tikzpicture}

  \draw(-1,-1) to[grid with coordinates,grid with coordinates/major line width=1pt] (3,3);

\end{tikzpicture}
\end{verbatim}

\begin{figure}[hbt]
\centering
\begin{tikzpicture}

  \draw(-1,-1) to[grid with coordinates,grid with
  coordinates/major line width=1pt] (3,3);

\end{tikzpicture}
\caption{Grid with Coordinates}
\label{fig:grid1}
\end{figure}

The second exame uses the same macro defined before, changing line widths, resulting in \autoref{fig:grid2}

\begin{verbatim}
\begin{tikzpicture}

  \draw(-2,-2) to[grid with coordinates,grid with
  coordinates/major line width=2pt,grid with
  coordinates/major step=.5,grid with
  coordinates/minor step=0.1] (3,3);

\end{tikzpicture}
\end{verbatim}

\begin{figure}[hbt]
\centering
\begin{tikzpicture}

  \draw(-2,-2) to[grid with coordinates,grid with coordinates/major line width=2pt,grid with coordinates/major step=.5,grid with coordinates/minor step=0.1] (3,3);

\end{tikzpicture}
\caption{Grid with Coordinates, another use}
\label{fig:grid2}
\end{figure}

\section{Simple Figures}

\begin{figure}[hbt]
    \centering
    \begin{tikzpicture}
\draw  (0,0)-- (0,5) -- (5,5) -- (5,0) -- (0,0);
\draw[fill=black]  (0,0) circle (1.5pt);
\draw[fill=black]  (5,0) circle (1.5pt);
\draw[fill=black]  (0,5) circle (1.5pt);
\draw[fill=black]  (5,5) circle (1.5pt);
\node[left] at (0,0) {$\{\}=(0,0)$ };
\node[left] at (0,5) {$\{\beta\}=(0,1)$ };
\node[right] at (5,0) {$\{\alpha\}=(1,0)$};
\node[right] at (5,5) {$\{\alpha,\beta\}= (1,1)$};
\draw[color=red,style=thick] (0,5) -- (5,0) node [pin={[pin edge={solid,red}]60:{$p(x)+p(\Bar{x})=p(y)+p(\Bar{y})=1$}},midway] {};

\end{tikzpicture}
    \caption{ERRADA!!!!! }

\end{figure}






\section{Axis and Vectors}



\begin{figure}[hbt]
\centering
\begin{tikzpicture}

% eixos
\draw[very thick,<->] (0,10) -- (0,0) -- (10,0);
\node[rotate=90]
at (-.3,2.5) {$termo_m$};
\node[below] at (2.6,0) {$termo_p$};

% retas verticais
%\draw[dotted] (1,0) -- (1,5);
%\draw[dotted] (2,0) -- (2,5);
%\draw[dotted] (3,0) -- (3,5);
%\draw[dotted] (4,0) -- (4,5);
%\draw[dotted] (5,0) -- (5,5);

% retas horizontais
%\draw[dotted] (0,1) -- (5,1);
%\draw[dotted] (0,2) -- (5,2);
%\draw[dotted] (0,3) -- (5,3);
%\draw[dotted] (0,4) -- (5,4);
%\draw[dotted] (0,5) -- (5,5);



\draw[,[->] (0,0) -- (10,10)  node[above] {$d_1$}  ;

\draw[,red,->] (0,0) -- (5,4)  node[above] {$q$}  ;

\draw[,[->] (0,0) -- (5,1)  node[above] {$d_2$}  ;

\draw[,[->] (0,0) -- (1,5)  node[above] {$d_3$}  ;

\end{tikzpicture}
\caption{Regras fuzzy funcionam como especificação de pedaços das funções sendo agregadas }

\end{figure}
\begin{figure}[hbt]
\centering
\begin{tikzpicture}

% eixos
\draw[very thick,<->] (0,5) -- (0,0) -- (5,0);
\node[rotate=90]
at (-.3,2.5) {$termo_m$};
\node[below] at (2.6,0) {$termo_p$};

\coordinate (origin) at (0,0);
\coordinate (q) at (5,5);
\coordinate (d1) at (5,4);
\coordinate (d2) at (1,4);
\coordinate (d3) at (4,1);


\draw[->] (origin) -- (q)  node[above] {$q$}  ;

\draw[red,->] (origin) -- (d1)  node[above] {$d_1$}  ;

\draw[->,blue] (origin) -- (d2)  node[above] {$d_2$}  ;

\draw[->,orange] (origin) -- (d3)  node[above] {$d_3$}  ;

     \pic [draw,->,"$\alpha$" font=\small,
     angle radius=20mm,orange,
     angle eccentricity=1.1,
     ] {angle=d3--origin--q};

     \pic [draw,->,red,
     angle radius=40mm,
     angle eccentricity=1.1,
     "$\beta$"] {angle=d1--origin--q};

          \pic [draw,->,"$\theta$",
     angle radius=10mm,blue,
     angle eccentricity=1.2,
     ] {angle=q--origin--d2};

\end{tikzpicture}
\caption{Usando o coseno dos vetores. }
\label{fig:cosvet}
\end{figure}

\begin{figure}[hbt]
\centering
\begin{tikzpicture}

% eixos
\draw[very thick,<->] (0,10) -- (0,0) -- (10,0);
\node[rotate=90]
at (-.3,2.5) {$termo_m$};
\node[below] at (2.6,0) {$termo_p$};

\coordinate (origin) at (0,0);
\coordinate (q) at (5,4);
\coordinate (d1) at (10,10);
\coordinate (d2) at (1,4);
\coordinate (d3) at (4,1);


% retas verticais
%\draw[dotted] (1,0) -- (1,5);
%\draw[dotted] (2,0) -- (2,5);
%\draw[dotted] (3,0) -- (3,5);
%\draw[dotted] (4,0) -- (4,5);
%\draw[dotted] (5,0) -- (5,5);

% retas horizontais
%\draw[dotted] (0,1) -- (5,1);
%\draw[dotted] (0,2) -- (5,2);
%\draw[dotted] (0,3) -- (5,3);
%\draw[dotted] (0,4) -- (5,4);
%\draw[dotted] (0,5) -- (5,5);

\draw[->,blue] (origin) -- (d1)  node[above] {$d_1$}  ;
\draw[->] (origin) -- (q)  node[above] {$q$}  ;
\draw[red,->] (origin) -- (d2)  node[above] {$d_2$}  ;
\draw[orange,->] (origin) -- (d3)  node[right] {$d_3$}  ;

\draw[-latex,dashed,blue] (d1) -- (q)  node[label={[label distance=2cm]45:$q-d_1$}] {}  ;
\draw[dashed,red,-latex] (d2) -- (q)  node[label={[label distance=2cm]177:$q-d_2$}] {}  ;
\draw[dashed,orange,-latex] (d3) -- (q) node[label={[label distance=2cm]270:$q-d_3$}] {}  ;



\end{tikzpicture}
\caption{Exemplo de problema com o uso do tamanho dos vetores }
\label{fig:vetdist1}
\end{figure}


\section{Real 3D}

\begin{figure}
\begin{center}
\begin{tikzpicture}[scale=0.6]
  \begin{axis}[grid=both,
  colormap/blackwhite,
  title={Usando $\max(\mu_{\Tilde{A}}(x),\mu_{\Tilde{B}}(x))$},
  xlabel={$\mu_{\Tilde{A}}(x)$},
  ylabel={$\mu_{\Tilde{B}}(x)$},
  zlabel={$\mu_{\Tilde{A}\cup\Tilde{B}}(x)$}
  ]
\addplot3[surf,domain=0:1]
      {max(x,y)};
\end{axis}
\end{tikzpicture}
\qquad
\begin{tikzpicture}[scale=0.6]
  \begin{axis}[grid=both,
    colormap/blackwhite,colorbar,
    title={Usando $\mu_{\Tilde{A}}(x) + \mu_{\Tilde{B}}(y) -
  \mu_{\Tilde{A}}(x),\mu_{\Tilde{B}}(x)$},
  xlabel={$\mu_{\Tilde{A}}(x)$},
  ylabel={$\mu_{\Tilde{B}}(x)$},
  zlabel={$\mu_{\Tilde{A}\cup\Tilde{B}}(x)$}
  ]
    \addplot3[surf,domain=0:1]
      {x+y-x*y)};
  \end{axis}
\end{tikzpicture}
\end{center}
    \caption{Desenho de 3D  (real) a partir de funções.}

\end{figure}

\section{More difficult drawing mixing nodes and curves}



\begin{figure}[hbt]
\centering
\begin{tikzpicture}

% eixos
\draw[<->] (0,5) -- (0,0) -- (5,0);
\node
at (-.2,1.8) {$y$};
\node[below] at (2.6,0) {$x$};

% triangulos horizontais
\draw (0,0) -- (1,-2) node[below] {X1}-- (2,0) --
        (3,-2) node[below] {X3 }-- (4,0) -- (5,-2) node[below] {X5};
\draw[dashed] (0,-2) node[below] {X0}-- (1,0) -- (2,-2) node[below] {X2} --
        (3,0)-- (4,-2) node[below] {X4} -- (5,0);

% triangulos verticais
\draw (-2,0) node[rotate=90,above] {Y0} -- (0,1) -- (-2,2) node[rotate=90,above] {Y2} --
        (0,3)-- (-2,4) node[rotate=90,above] {Y4} -- (0,5);
\draw[dashed] (0,0) -- (-2,1) node[rotate=90,above] {Y1} -- (0,2) --
        (-2,3) node[rotate=90,above] {Y3} -- (0,4) -- (-2,5) node[rotate=90,above] {Y5};

% retas verticais
\draw[dotted] (1,0) -- (1,5);
\draw[dotted] (2,0) -- (2,5);
\draw[dotted] (3,0) -- (3,5);
\draw[dotted] (4,0) -- (4,5);
\draw[dotted] (5,0) -- (5,5);

% retas horizontais
\draw[dotted] (0,1) -- (5,1);
\draw[dotted] (0,2) -- (5,2);
\draw[dotted] (0,3) -- (5,3);
\draw[dotted] (0,4) -- (5,4);
\draw[dotted] (0,5) -- (5,5);



\draw[very thick] (0,0) coordinate (a1) .. controls  (1,2) and (4,1) .. (5,5) coordinate (a2) node[pin=below left:{$y=f(x)$}] { }  ;

\draw[blue!50,rotate around={45:(0.5,0.5)}] (0.5,0.5) ellipse [x radius=0.7, y radius=0.2];

\draw[blue!50,rotate around={35:(1,1)}] (1,1) ellipse [x radius=0.7, y radius=0.2];

\draw[blue!50,rotate around={35:(1.5,1.3)}] (1.5,1.3) ellipse [x radius=0.7, y radius=0.2];

\draw[blue!50,rotate around={30:(2,1.5)}] (2,1.5) ellipse [x radius=0.7, y radius=0.2];

\draw[blue!50,rotate around={30:(2.5,1.8)}] (2.5,1.8) ellipse [x radius=0.7, y radius=0.2];

\draw[blue!50,rotate around={40:(3,2)}] (3,2) ellipse [x radius=0.7, y radius=0.2];

\draw[blue!50,rotate around={45:(3.5,2.5)}] (3.5,2.5) ellipse [x radius=0.7, y radius=0.2];

\draw[blue!50,rotate around={45:(4,3)}] (4,3) ellipse [x radius=0.7, y radius=0.2];

\draw[blue!50,rotate around={60:(4.5,3.5)}] (4.5,3.5) ellipse [x radius=0.7, y radius=0.2];

\draw[blue!50,rotate around={70:(4.8,4.5)}] (4.8,4.5) ellipse [x radius=0.7, y radius=0.2];

\draw[thick,red,dashed] (2.6,0) -- (2.6,1.8);
\draw[thick,red,dashed] (0,1.8) -- (2.6,1.8);
\fill[red] (2.6,1.8) circle [radius=0.1];

\node[font=\tiny] at (2,3.5) {IF X=X2 THEN Y=Y1};
\node[font=\tiny] at (2,3.2) {IF X=X2 THEN Y=Y2};
\node[font=\tiny] at (2,2.9) {IF X=X3 THEN Y=Y1};
\node[font=\tiny] at (2,2.6) {IF X=X3 THEN Y=Y2};

\end{tikzpicture}
\caption{Regras fuzzy funcionam como especificação de pedaços das funções sendo agregadas }

\end{figure}

\begin{figure}[hbt]
\centering
\begin{tikzpicture}[
ponto/.style={rectangle,draw,very thick,color=red,fill=red,
minimum width=5pt,
minimum height=5pt},
query/.style={circle,draw,very thick,color=blue,
minimum width=5pt,fill=blue,
minimum height=5pt}
]

% eixos
\draw[very thick,<->] (0,10) -- (0,0) -- (10,0);
\node[rotate=90]
at (-.3,2.5) {$\textrm{Significado}_2$};
\node[below] at (2.6,0) {$\textrm{Significado}_1$};
\filldraw[even odd rule,inner color=blue!50!white,outer color=white,blue] (5,4) circle (2.2);
\node[ponto,label =above:{Doc D}] at (2,2) {};
\node[ponto,label =above:{Doc A}] at (4,8) {};
%\node[ponto,label =above:{Doc D}] at (6,6) {};
\node[ponto,label =above:{Doc B}] at (5,5) {};
\node[ponto,label =above:{Doc C}] at (3,4) {};
\node[query,label =above:{Query A}] at (5,4) {};
\node[query,label =above:{Query B}] at (8,2) {};


\end{tikzpicture}
\caption{Ideia do Modelo Vetorial}
\label{fig:asdash}
\end{figure}


\begin{figure}[hbt]
\centering
\begin{tikzpicture}

% eixos
\draw[<->] (0,5) -- (0,0) -- (5,0);

% triangulos horizontais
\draw (0,0)
        -- (1,-2) node[below] {X1}
        -- (2,0) ;
\draw[red] (2,0)
        -- (3,-2) node[red,below] {X3 }
        -- (4,0);
\draw   (4,0)
        -- (5,-2) node[below] {X5};

\draw[dashed] (0,-2) node[below] {X0}
        -- (1,0) ;
\draw[blue,dashed] (1,0)
        -- (2,-2) node[below,blue] {X2}
        -- (3,0);
\draw[dashed] (3,0)
        -- (4,-2) node[below] {X4}
        -- (5,0);

% triangulos verticais
\draw (-2,0) node[rotate=90,above] {Y0}
        -- (0,1)
        -- (-2,2) node[red,rotate=90,above] {Y2}
        -- (0,3)
        -- (-2,4) node[rotate=90,above] {Y4}
        -- (0,5);
\draw[dashed] (0,0)
        -- (-2,1) node[rotate=90,above] {Y1}
        -- (0,2)
        -- (-2,3) node[blue,rotate=90,above] {Y3}
        -- (0,4)
        -- (-2,5) node[rotate=90,above] {Y5};

% retas verticais
\draw[dotted] (1,0) -- (1,5);
\draw[dotted] (2,0) -- (2,5);
\draw[dotted] (3,0) -- (3,5);
\draw[dotted] (4,0) -- (4,5);
\draw[dotted] (5,0) -- (5,5);

% retas horizontais
\draw[dotted] (0,1) -- (5,1);
\draw[dotted] (0,2) -- (5,2);
\draw[dotted] (0,3) -- (5,3);
\draw[dotted] (0,4) -- (5,4);
\draw[dotted] (0,5) -- (5,5);



\fill[black] (2.5,0) circle [radius=0.1] node[above] {$x$};
\draw[dotted,->] (2.5,0) -- (2.5,-0.9);
\fill[black] (2.5,-1) circle [radius=0.1] node[right,font=\tiny] {$\mu_{X3}(x)=0.5$};
\node[left,font=\tiny] at (2.5,-1) {$\mu_{X2}(x)=0.5$};

\fill[pattern color=red,pattern=vertical lines] (2,1) rectangle (4,3) ;
\fill[pattern color=blue,pattern=horizontal lines] (1,2) rectangle (3,4) ;

\node[font=\tiny] at (2,4) [above] {IF X=X2 THEN Y=Y3};
\node[font=\tiny] at (3,1) [below] {IF X=X3 THEN Y=Y2};

\fill[pattern color=red,pattern=vertical lines] (0,1) -- (-1,1.5) -- (-1,2.5) -- (0,3);
\fill[pattern color=blue,pattern=horizontal lines] (0,2) -- (-1,2.5) -- (-1,3.5) -- (0,4);

\fill[black] (0,2.5) circle [radius=0.1] node[right] {$y$};


\end{tikzpicture}
    \caption{Duas regras ativadas simultaneamente de um conjunto de regras, a partir de uma entrada $x$, são agregadas e uma função de defuzzificação, como o centróide, é usada para determinar $y$}

\end{figure}



\section{1D}


\begin{figure}[hbt]
\begin{tikzpicture}
\draw[|->] node[below] {0}  (0,0) -- (10,0)node[below] {D};
\fill[black] (2,0) circle [radius=0.1] node[below] {$x$};
\fill[black] (7,0) circle [radius=0.1] node[below] {$y$};
\fill[black] (9,0) circle [radius=0.1] node[below] {$z$};
\draw[|->,blue,dotted] (2,-0.6) -- (4,-0.6) node[below] {$\epsilon$} -- (6,-0.6);
\fill[red] (6,0)  circle [radius=0.1] node[above] {$(y+z+x)/3$};
\end{tikzpicture}
\caption{Visão gráfica da medida simples de concordância para três pontos, x basicamente em desacordo, considerando um contra a média de todos}

\end{figure}

\begin{figure}[hbt]
\begin{tikzpicture}
\draw[|->] node[below] {0}  (0,0) -- (10,0)node[below] {D};
\fill[black] (7.7,0) circle [radius=0.1] node[below] {$x$};
\fill[black] (7,0) circle [radius=0.1] node[below] {$y$};
\fill[black] (9,0) circle [radius=0.1] node[below] {$z$};
\fill[red] (8,0)  circle [radius=0.1] node[above] {$(y+z)/2$};
\end{tikzpicture}
\caption{Visão gráfica da medida simples de concordância para três pontos, x basicamente em acordo}

\end{figure}

\section{Desenhos de Fuzzy}



\begin{figure}[hbt]
    \centering
    \begin{tikzpicture}[scale=0.7]
\draw[<->] (0,5) -- (0,0) -- (12,0);
\node[left] at (0,5) {$100\%$};
\node[left] at (0,4) {$80\%$};
\node[left] at (0,3) {$60\%$};
\node[left] at (0,2) {$40\%$};
\node[left] at (0,1) {$20\%$};
\node[left] at (0,0) {$0\%$};
\draw[style=dashed,color=gray!50] (0,5) -- (12,5);
\draw[style=dashed,color=gray!50] (0,4) -- (12,4);
\draw[style=dashed,color=gray!50] (0,3) -- (12,3);
\draw[style=dashed,color=gray!50] (0,2) -- (12,2);
\draw[style=dashed,color=gray!50] (0,1) -- (12,1);
\node[below] at (12,0) {$120$};
\node[below] at (10,0) {$100$};
\node[below] at (8,0) {$80$};
\node[below] at (6,0) {$60$};
\node[below] at (4,0) {$40$};
\node[below] at (2,0) {$20$};
\node[below] at (0,0) {$0$};
\draw[color=blue,style=thick] (0,0) -- (3,0) -- (3,2.5) -- (6,5) -- (9,2.5)-- (9,0) -- (12,0);
\node[rotate=90] at (-2,2.5) {$\mu_{\tilde{V}}(x)$};
\node at (6,-1) {V(x) - (km/h)};
\node[above,color=blue]  at (6,5) {médio$^{0,5}$};
\end{tikzpicture}
    \caption{O conjunto referente ao corte-alfa de médio com $\alpha=0,5$, ou seja médio$^{0,5}$.}

\end{figure}

\begin{figure}[hbt]
    \centering
    \begin{tikzpicture}[scale=0.7]
\draw[<->] (0,5) -- (0,0) -- (12,0);
\node[left] at (0,5) {$100\%$};
\node[left] at (0,4) {$80\%$};
\node[left] at (0,3) {$60\%$};
\node[left] at (0,2) {$40\%$};
\node[left] at (0,1) {$20\%$};
\node[left] at (0,0) {$0\%$};
\draw[style=dashed,color=gray!50] (0,5) -- (12,5);
\draw[style=dashed,color=gray!50] (0,4) -- (12,4);
\draw[style=dashed,color=gray!50] (0,3) -- (12,3);
\draw[style=dashed,color=gray!50] (0,2) -- (12,2);
\draw[style=dashed,color=gray!50] (0,1) -- (12,1);
\node[below] at (12,0) {$120$};
\node[below] at (10,0) {$100$};
\node[below] at (8,0) {$80$};
\node[below] at (6,0) {$60$};
\node[below] at (4,0) {$40$};
\node[below] at (2,0) {$20$};
\node[below] at (0,0) {$0$};
\node[rotate=90] at (-2,2.5) {$\mu_{\tilde{V}}(x)$};
\node at (6,-1) {V(x) - (km/h)};
\draw[color=blue,style=thick] (0,0) -- (3,0) -- (4,2.5) -- (5,0) -- (8,0)-- (10,5) -- (12,0);
\draw[color=black,style=thick] (3.5,1.25) -- (4.5,1.25);
\draw[color=black,style=thick] (8.5,1.25) -- (11.5,1.25);
\end{tikzpicture}
    \caption{Exemplo de cortes-$\alpha$}

\end{figure}

\begin{figure}[hbt]
    \centering
    \begin{tikzpicture}[yscale=0.05,xscale=.3]
\draw[<->] (0,100) -- (0,0) -- (30,0);
\node[left] at (0,100) {$100\%$};
\node[left] at (0,80) {$80\%$};
\node[left] at (0,60) {$60\%$};
\node[left] at (0,40) {$40\%$};
\node[left] at (0,20) {$20\%$};
\node[left] at (0,0) {$0\%$};
\draw[style=dashed,color=gray!50] (0,100) -- (30,100);
\draw[style=dashed,color=gray!50] (0,80) -- (30,80);
\draw[style=dashed,color=gray!50] (0,60) -- (30,60);
\draw[style=dashed,color=gray!50] (0,40) -- (30,40);
\draw[style=dashed,color=gray!50] (0,20) -- (30,20);
\node[below] at (30,0) {$30$};
\node[below] at (25,0) {$25$};
\node[below] at (20,0) {$20$};
\node[below] at (15,0) {$15$};
\node[below] at (10,0) {$10$};
\node[below] at (5,0) {$5$};
\node[below] at (0,0) {$0$};

\draw[color=black,style=thick] (3,20) -- (11,20);
\draw[color=black,style=thick] (16,20) -- (24,20);

\draw[color=black,style=thick] (4,40) -- (10,40);
\draw[color=black,style=thick] (17,40) -- (23,40);

\draw[color=black,style=thick] (5,60) -- (9,60);
\draw[color=black,style=thick] (18,60) -- (22,60);

\draw[color=black,style=thick] (6,80) -- (8,80);
\draw[color=black,style=thick] (19,80) -- (21,80);

\node[rotate=90] at (-5,50) {$\mu_{\tilde{A}}(x)$};

\node at (15,-15) {x};

\end{tikzpicture}
    \caption{Representação dos cortes-$\alpha$ do conjunto $\tilde{A}$ dos números perto de 7 ou 20}

\end{figure}



\begin{figure}[hbt]
\centering
\begin{tikzpicture}
\draw[<->] (0,5) -- (0,0) -- (8,0);
\node[rotate=90] at (-.3,2.5) {$y$};
\node[below] at (4,0) {$x$};

\draw[very thick] (0,0) .. controls (4,6) and (7,4).. (8,5) node[pin=below:{$y=f(x)$}] { } ;
\draw[rotate around={45:(1,1)}] (0.7,1)ellipse [x radius=1, y radius=0.5];
\draw[rotate around={45:(1.5,2)}] (1.5,2)ellipse [x radius=1, y radius=0.5];
\draw[rotate around={30:(3,3)}] (3,3)ellipse [x radius=1, y radius=0.5];
\draw[rotate around={10:(4.5,4)}] (4.5,4)ellipse [x radius=1.2, y radius=0.3];
\draw[rotate around={10:(5.5,4.2)}] (5.5,4.2)ellipse [x radius=1.2, y radius=0.5];
\draw[rotate around={10:(6.5,4.5)}] (6.5,4.5)ellipse [x radius=1.2, y radius=0.3];
 \end{tikzpicture}
 \caption{Grafico de $y=f(x)$}
\end{figure}





\begin{figure}
\centering

\begin{tikzpicture}[scale=0.7]

\def\pttocm#1{\pgfmathparse{#1 / 19.93333
    }\pgfmathprintnumber{\pgfmathresult}}

    \newcommand{\printcoords}[1]{
      \newdimen\posx
      \pgfextractx{\posx}{\pgfpointanchor{#1}{center}}
      \newdimen\posy
      \pgfextracty{\posy}{\pgfpointanchor{#1}{center}}
      (\pttocm\posx ; \pttocm\posy )
    }


\draw[<->] (0,5) -- (0,0) -- (12,0);
\node[left] at (0,5) {$100\%$};
\node[left] at (0,4) {$80\%$};
\node[left] at (0,3) {$60\%$};
\node[left] at (0,2) {$40\%$};
\node[left] at (0,1) {$20\%$};
\node[left] at (0,0) {$0\%$};
\draw[style=dashed,color=gray!50] (0,5) -- (12,5);
\draw[style=dashed,color=gray!50] (0,4) -- (12,4);
\draw[style=dashed,color=gray!50] (0,3) -- (12,3);
\draw[style=dashed,color=gray!50] (0,2) -- (12,2);
\draw[style=dashed,color=gray!50] (0,1) -- (12,1);
\node[below] at (12,0) {$120$};
\node[below] at (10,0) {$100$};
\node[below] at (8,0) {$80$};
\node[below] at (6,0) {$60$};
\node[below] at (4,0) {$40$};
\node[below] at (2,0) {$20$};
\node[below] at (0,0) {$0$};
\draw[color=blue,style=thick,style=dotted,name path=vava] (0,0) -- (3,0) -- (4,2.5) -- (5,0) -- (8,0)-- (10,5) node[above] {$\Tilde{A}$} -- (12,0);
\path[name path=xixi,draw=none] (0,1) -- (12,1);
\path[name intersections={of=vava and xixi }];
\coordinate (A) at (intersection-1);
\coordinate (B) at (intersection-2);
\coordinate (C) at (intersection-3);
\coordinate (D) at (intersection-4);
\draw[color=black,style=thick] (A) node[left] {\printcoords{A}} -- (B);
\draw[color=black,style=thick] (C) --  (D) node[right] {$\tilde{A}^{20\%}$}  ;
\path[name path=xuxu,draw=none] (0,2) -- (12,2);
\path[name intersections={of=vava and xuxu }];
\coordinate (A) at (intersection-1);
\coordinate (B) at (intersection-2);
\coordinate (C) at (intersection-3);
\coordinate (D) at (intersection-4);
\draw[color=red,style=thick] (A) -- (B);
\draw[color=red,style=thick] (C) -- (D) node[right] {$\Tilde{A}^{40\%}$};
\path[name path=xexe,draw=none] (0,3) -- (12,3);
\path[name intersections={of=vava and xexe }];
\coordinate (A) at (intersection-1);
\coordinate (B) at (intersection-2);
\draw[color=green,style=thick] (A) -- (B) node[right] {$\Tilde{A}^{60\%}$};
\path[name path=xaxa,draw=none] (0,4) -- (12,4);
\path[name intersections={of=vava and xaxa }];
\coordinate (A) at (intersection-1);
\coordinate (B) at (intersection-2);
\draw[color=orange,style=thick] (A) -- (B)node[right] {$\Tilde{A}^{80\%}$};
\end{tikzpicture}
    \caption{Exemplo de cortes-$\alpha$, onde as linhas horizontais indicam os valores do eixo das abcissas que pertencem ao conjunto nítido correspondente}

\end{figure}


\section{Gráficos a partir de números}
\begin{tikzpicture}
\begin{axis}[
    title={Temperature dependence of CuSO\(_4\cdot\)5H\(_2\)O solubility},
    xlabel={Temperature [\textcelsius]},
    ylabel={Solubility [g per 100 g water]},
    xmin=0, xmax=100,
    ymin=0, ymax=120,
    xtick={0,20,40,60,80,100},
    ytick={0,20,40,60,80,100,120},
    legend pos=north west,
    ymajorgrids=true,
    grid style=dashed,
]

\addplot[
    color=blue,
    mark=square,
    ]
    coordinates {
    (0,23.1)(10,27.5)(20,32)(30,37.8)(40,44.6)(60,61.8)(80,83.8)(100,114)
    };
    \legend{CuSO\(_4\cdot\)5H\(_2\)O}

\end{axis}
\end{tikzpicture}

\begin{tikzpicture}
\draw[<->] (0,11) -- (0,0) -- (11,0);

\foreach \i in {0,1,...,10} \draw (\i,0)--(\i,-.1);
\foreach \i in {0,.2,...,1} \node[below] at (10*\i,0) {\pgfmathprintnumber
[fixed,fixed zerofill,precision=1,use comma]
{\i}};
\node[below] at (5.5,-.5) {Revocação};

\foreach \i in {0,1,...,10} \draw (0,\i)--(-.1,\i);
\foreach \i in {0,.2,...,1} \node[left] at (0,10*\i) {\pgfmathprintnumber
[fixed,fixed zerofill,precision=1,use comma]
{\i}};
\node[rotate=90] at (-.8,5.5) {Precisão};

\draw[very thick] (1,10) ..
controls (8,9)  and    (3,0)
.. (10,1) ;

\node at (4,10) {Consultas específicas};
\node at (9,2) {Consultas genéricas};


\end{tikzpicture}


\begin{figure}
\centering
\begin{tikzpicture}
\begin{axis}[
axis lines = left,
xlabel = $f_{i,j}$,
ylabel = $\mathrm{TF}_K$
]
\addplot [domain=0:20, samples=40,thick, color=ibm2]
{x/(x+1)} node [pos=0.3, above left] {$K=1$}  ;
\addplot [dotted,domain=0:20, samples=40,thick, color=ibm3]
{x/(x+5)} node [pos=0.9, above left] {$K=5$}  ;
\addplot [dashed,domain=0:20, samples=40,thick, color=ibm4]
{x/(x+10)} node [pos=0.9, above left] {$K=10$}  ;
\end{axis}
\end{tikzpicture}
\caption{Várias funções (usa pgfplot) no eixo 2D}
\end{figure}
\chapter{Circle Magic}

Pie charts are very easy!

\begin{verbatim}
\begin{tikzpicture}
\pie{24/SAP, 12/Oracle,
6/Sage, 6/Infor, 5/Microsoft,
47/Outros}
\end{tikzpicture}
\end{verbatim}

\begin{figure}[hbt]
    \centering
    \begin{tikzpicture}
\pie{24/SAP, 12/Oracle,
6/Sage, 6/Infor, 5/Microsoft,
47/Outros}
\end{tikzpicture}
    \caption{ERP Market in 2013 }
    \label{fig:ERPMarket}
\end{figure}

Using polar coordinates it is easy to draw a circle. The following code results in \autoref{fig:circle1}.

\begin{verbatim}
\begin{tikzpicture}
  \coordinate (center) at (1,2);
  \def\radius{2.5cm}
  % a circle
  \draw[dotted] (center) circle[radius=\radius];

  \fill[black] (center) ++(0:\radius)
  circle[radius=4pt] node[black,right] {1} ;
  
  \fill[red] (center) ++(36:\radius)
  circle[radius=2pt] node[right] {2 Malala's Call};
  
  \fill[red] (center) ++(2*36:\radius)
  circle[radius=2pt] node[above] {3};
  
  \fill[blue] (center) ++(3*36:\radius)
  circle[radius=2pt] node[above] {4};
  
  \fill[red] (center) ++(4*36:\radius)
  circle[radius=2pt] node[left] {5};
  
  \fill[red] (center) ++(5*36:\radius)
  circle[radius=2pt] node[left] {6};
  
  \fill[blue] (center) ++(6*36:\radius)
  circle[radius=2pt] node[left] {7};
  
  \fill[red] (center) ++(7*36:\radius)
  circle[radius=2pt] node[below] {8};
  
  \fill[red] (center) ++(8*36:\radius)
  circle[radius=2pt] node[below] {9};
  
  \fill[blue] (center) ++(9*36:\radius)
  circle[radius=2pt] node[right] {10};
  
   \draw[-{>[scale=2.5,
          length=2,
          width=3]}]  (center)+(4*36:\radius) --   
          +(9*36:\radius) ;
   \draw[-{>[scale=2.5,
          length=2,
          width=3]}]  (center)+(2*36:\radius) --   
          +(9*36:\radius) ;
   \draw[-{>[scale=2.5,
          length=2,
          width=3]}]  (center)+(0:\radius) --   
          +(9*36:\radius) node [midway, right] {$w$};
 
   \draw[-{>[scale=2.5,
          length=2,
          width=3]}]  (center)+(7*36:\radius) --   
          +(3*36:\radius) ;
   \draw[-{>[scale=2.5,
          length=2,
          width=3]}]  (center)+(5*36:\radius) --   
          +(3*36:\radius) ;
   \draw[-{>[scale=2.5,
          length=2,
          width=3]}]  (center)+(0:\radius) --   
          +(3*36:\radius) node [midway, below] {$w$};
          
  \draw[-{>[scale=2.5,
          length=2,
          width=3]}]  (center)+(1*36:\radius) --   
          +(6*36:\radius) ;
   \draw[-{>[scale=2.5,
          length=2,
          width=3]}]  (center)+(8*36:\radius) --   
          +(6*36:\radius) ;
   \draw[-{>[scale=2.5,
          length=2,
          width=3]}]  (center)+(0:\radius) --   
          +(6*36:\radius) node [midway, below] {$w$};  
 
                 
\end{tikzpicture}
\end{verbatim}


\begin{figure}[hbt]
\begin{tikzpicture}
  \coordinate (center) at (1,2);
  \def\radius{2.5cm}
  % a circle
  \draw[dotted] (center) circle[radius=\radius];

  
  \fill[black] (center) ++(0:\radius)
  circle[radius=4pt] node[black,right] {1} ;
  
  \fill[red] (center) ++(36:\radius)
  circle[radius=2pt] node[right] {2 Malala's Call};
  
  \fill[red] (center) ++(2*36:\radius)
  circle[radius=2pt] node[above] {3};
  
  \fill[blue] (center) ++(3*36:\radius)
  circle[radius=2pt] node[above] {4};
  
  \fill[red] (center) ++(4*36:\radius)
  circle[radius=2pt] node[left] {5};
  
  \fill[red] (center) ++(5*36:\radius)
  circle[radius=2pt] node[left] {6};
  
  \fill[blue] (center) ++(6*36:\radius)
  circle[radius=2pt] node[left] {7};
  
  \fill[red] (center) ++(7*36:\radius)
  circle[radius=2pt] node[below] {8};
  
  \fill[red] (center) ++(8*36:\radius)
  circle[radius=2pt] node[below] {9};
  
  \fill[blue] (center) ++(9*36:\radius)
  circle[radius=2pt] node[right] {10};
  
   \draw[-{>[scale=2.5,
          length=2,
          width=3]}]  (center)+(4*36:\radius) --   
          +(9*36:\radius) ;
   \draw[-{>[scale=2.5,
          length=2,
          width=3]}]  (center)+(2*36:\radius) --   
          +(9*36:\radius) ;
   \draw[-{>[scale=2.5,
          length=2,
          width=3]}]  (center)+(0:\radius) --   
          +(9*36:\radius) node [midway, right] {$w$};
 
   \draw[-{>[scale=2.5,
          length=2,
          width=3]}]  (center)+(7*36:\radius) --   
          +(3*36:\radius) ;
   \draw[-{>[scale=2.5,
          length=2,
          width=3]}]  (center)+(5*36:\radius) --   
          +(3*36:\radius) ;
   \draw[-{>[scale=2.5,
          length=2,
          width=3]}]  (center)+(0:\radius) --   
          +(3*36:\radius) node [midway, below] {$w$};
          
  \draw[-{>[scale=2.5,
          length=2,
          width=3]}]  (center)+(1*36:\radius) --   
          +(6*36:\radius) ;
   \draw[-{>[scale=2.5,
          length=2,
          width=3]}]  (center)+(8*36:\radius) --   
          +(6*36:\radius) ;
   \draw[-{>[scale=2.5,
          length=2,
          width=3]}]  (center)+(0:\radius) --   
          +(6*36:\radius) node [midway, below] {$w$};  
 
                 
\end{tikzpicture}
\caption{Points and chords in a circle.}
\label{fig:circle1}
\end{figure}

Another example of using polar coordinates to draw a circle results in \autoref{fig:malala}.

\begin{verbatim}
\begin{tikzpicture}
% posicao central do circulo
  \coordinate (center) at (1,2);
% coloca o nome aqui
\def\nome{Campbell's Hero Journey}
% nome fica no centro
  \node[align=center,text width=4cm,anchor=center] at (center) {\baselineskip=16pt \Huge{\nome}\par};
% raio do circulo
  \def\radius{4cm}
% numero de pontos
  \def\passos{10}
% tamanho em angulo graus do passo
  \def\passo{360/\passos}

% em vez de círculo, podemos usar um arco
% aqui tem um truque, que é usar o shift para
% o primeiro valor que você usar no nosso
% "loop aberto", no caso o 2*\passo
% 
  
\draw[black] ([shift=(2*\passo:\radius)]center) arc (2*\passo:-7*\passo:\radius);

% cada ponto é um fill 

% tem que acertar para cada ponto o multiplicador do passo
% isso deveria ser um for, mas é realmente melhor
% fazer na mão para controlar tudo

  \fill (center) ++(2*\passo:\radius)
   node[above,yshift=1em,xshift=2em] {\textbf{Primeiro Ato}};
  

  \fill[black] (center) ++(2*\passo:\radius)
  %circle[radius=4pt] 
  node[regular polygon, regular polygon sides=3, fill,regular polygon rotate=-90,minimum width = 11pt,inner sep =0] {} 
  node[left,yshift=-.7em ] {1} node[black, right,xshift=.5em,yshift=.3em] {Call to Action } ;
  
  \fill[black] (center) ++(1*\passo:\radius)
  circle[radius=2pt] node[left] {2} node[right] {Malala's Call};
  
  \fill[black] (center) ++(0*\passo:\radius)
  circle[radius=2pt] node[left] {3} node[right] {Malala's Call};
  
% SEGUNDO ATO   
  
  \fill (center) ++(-1*\passo:\radius)
   node[above,yshift=.5em,xshift=4.3em] {\textbf{Segundo Ato}};
  \fill[black] (center) ++(-1*\passo:\radius)
  circle[radius=2pt] node[below,right] {Malala's Call}
  node[left] {4};
 
  \fill[black] (center) ++(-2*\passo:\radius)
  circle[radius=2pt]  node[above] {5} node[below,xshift=3.5em] { Malala's Call};
  
% TERCEIRO ATO  
  
    \fill (center) ++(-3*\passo:\radius)
   node[below,yshift=-.5em,xshift=-3em] {\textbf{Terceiro Ato}};
  
  \fill[black] (center) ++(-3*\passo:\radius)
  circle[radius=2pt]  node[above] {6} node[left,xshift=-1em] {Malala's Call};
  


  \fill[black] (center) ++(-4*\passo:\radius)
  circle[radius=2pt] node[right] {7} node[left] {Malala's Call};
  
  \fill[black] (center) ++(-5*\passo:\radius)
  circle[radius=2pt] node[right] {8} node[left] {Malala's Call};
  
  \fill[black] (center) ++(-6*\passo:\radius)
  circle[radius=2pt] node[right] {9} node[left]  {Malala's Call};
  
  \fill[black] (center) ++(-7*\passo:\radius)
  node[shape=rectangle,fill] {} node[right,yshift=-.5em] {10}  node[above,xshift=-3.5em] {Malala's Call };
 
 \end{tikzpicture}
\end{verbatim}


\begin{figure}[hbt]
\begin{tikzpicture}
% posicao central do circulo
  \coordinate (center) at (1,2);
% coloca o nome aqui
\def\nome{Campbell's Hero Journey}
% nome fica no centro
  \node[align=center,text width=4cm,anchor=center] at (center) {\baselineskip=16pt \Huge{\nome}\par};
% raio do circulo
  \def\radius{4cm}
% numero de pontos
  \def\passos{10}
% tamanho em angulo graus do passo
  \def\passo{360/\passos}

% em vez de círculo, podemos usar um arco
% aqui tem um truque, que é usar o shift para
% o primeiro valor que você usar no nosso
% "loop aberto", no caso o 2*\passo
% 
  
\draw[black] ([shift=(2*\passo:\radius)]center) arc (2*\passo:-7*\passo:\radius);

% cada ponto é um fill 

% tem que acertar para cada ponto o multiplicador do passo
% isso deveria ser um for, mas é realmente melhor
% fazer na mão para controlar tudo

  \fill (center) ++(2*\passo:\radius)
   node[above,yshift=1em,xshift=2em] {\textbf{Primeiro Ato}};
  

  \fill[black] (center) ++(2*\passo:\radius)
  %circle[radius=4pt] 
  node[regular polygon, regular polygon sides=3, fill,regular polygon rotate=-90,minimum width = 11pt,inner sep =0] {} 
  node[left,yshift=-.7em ] {1} node[black, right,xshift=.5em,yshift=.3em] {Call to Action } ;
  
  \fill[black] (center) ++(1*\passo:\radius)
  circle[radius=2pt] node[left] {2} node[right] {Malala's Call};
  
  \fill[black] (center) ++(0*\passo:\radius)
  circle[radius=2pt] node[left] {3} node[right] {Malala's Call};
  
% SEGUNDO ATO   
  
  \fill (center) ++(-1*\passo:\radius)
   node[above,yshift=.5em,xshift=4.3em] {\textbf{Segundo Ato}};
  \fill[black] (center) ++(-1*\passo:\radius)
  circle[radius=2pt] node[below,right] {Malala's Call}
  node[left] {4};
 
  \fill[black] (center) ++(-2*\passo:\radius)
  circle[radius=2pt]  node[above] {5} node[below,xshift=3.5em] { Malala's Call};
  
% TERCEIRO ATO  
  
    \fill (center) ++(-3*\passo:\radius)
   node[below,yshift=-.5em,xshift=-3em] {\textbf{Terceiro Ato}};
  
  \fill[black] (center) ++(-3*\passo:\radius)
  circle[radius=2pt]  node[above] {6} node[left,xshift=-1em] {Malala's Call};
  


  \fill[black] (center) ++(-4*\passo:\radius)
  circle[radius=2pt] node[right] {7} node[left] {Malala's Call};
  
  \fill[black] (center) ++(-5*\passo:\radius)
  circle[radius=2pt] node[right] {8} node[left] {Malala's Call};
  
  \fill[black] (center) ++(-6*\passo:\radius)
  circle[radius=2pt] node[right] {9} node[left]  {Malala's Call};
  
  \fill[black] (center) ++(-7*\passo:\radius)
  node[shape=rectangle,fill] {} node[right,yshift=-.5em] {10}  node[above,xshift=-3.5em] {Malala's Call };
 
 \end{tikzpicture}
\caption{The Heroes Journey}
\label{fig:malala}
\end{figure}






\chapter{Measurements}

\begin{figure}
\begin{tikzpicture}

% ruler
 
            %% lower divisions
            \foreach \x in {0,1,...,10}{
            \draw (\x,0) -- (\x,0.2)node[above,scale=0.4]{\x};
            }
            \foreach \x in {0.1,0.2,...,9.9}{
            \draw (\x,0) -- (\x,0.075);
            }
            \foreach \x in {0.5,1,...,9.5}{
            \draw (\x,0) -- (\x,0.15);
            }


\node(a) [cylinder,draw, rotate=90, minimum height=2cm,aspect=0.25,minimum width=5cm,text width=1cm,align=center] at (0,0) {\rotatebox{-90}{ERP}} ;

\node(a) [cylinder,draw, rotate=90, minimum height=2cm,aspect=0.25,minimum width=5cm,text width=1cm,align=center] at (10,0)  {\rotatebox{-90}{Sistemas de Vendas Legado}} ;
\end{tikzpicture}
    \caption{Tentando usar cilindros, mas varia com o texto dentro dele }
    \label{fig:tentacili}
\end{figure}


\begin{figure}
\begin{tikzpicture}

% ruler
 
            %% lower divisions
            \foreach \x in {0,1,...,10}{
            \draw (\x,0) -- (\x,0.2)node[above,scale=0.4]{\x};
            }
            \foreach \x in {0.1,0.2,...,9.9}{
            \draw (\x,0) -- (\x,0.075);
            }
            \foreach \x in {0.5,1,...,9.5}{
            \draw (\x,0) -- (\x,0.15);
            }


\node(a) [cylinder,draw, rotate=90, minimum height=2cm,minimum width=3cm,align=center] at (0,0) {};
\node (d) [rectangle, minimum height=2cm,aspect=0.25,minimum width=5cm,text width=2cm,align=center] at (0,0) { ERP};

\node(b) [cylinder,draw, rotate=90, minimum height=2cm,minimum width=3cm,align=center] at (5,0) {} ;
\node(c) [rectangle, minimum height=2cm,aspect=0.25,minimum width=5cm,text width=2cm,align=center] at (5,0) { Sistemas Legados};
%{\rotatebox{-90}{Sistemas de Vendas Legado}} ;
\end{tikzpicture}

    \caption{Tentando usar cilindros,texto de fora }
    \label{fig:tentacili1}
\end{figure}


\chapter{Other}


\begin{figure}
    \centering
    \begin{tikzpicture}
    [ bola/.style={circle,draw=white, fill=white,font=\footnotesize,text=black} ]
    
    \node (n1) at (0,10) {$0$} ;
    \node (n2) at (1.5,10) {$\lambda$};
    \node (n3) at (3,10) {$\infty$} ;  

    \draw[->] (n1.east) -- (n2.west);
    \draw[->] (n2.east) -- (n3.west);

    \node (n6) at (12,10) {$0$} ;
    \node (n5) at (10.5,10) {$\lambda$};
    \node (n4) at (9,10) {$\infty$} ;  
    
    \draw[<-] (n4.east) -- (n5.west);
    \draw[<-] (n5.east) -- (n6.west);
        
    \node (nimn) at  (0,3) {$i_{min}$};
    \node (numax) at (12,3) {$u_{max}$};
    \node (min) at (3,3) {$\min$};
    \node (max) at (9,3) {$\max$};
    \node at (1.5,10.5)  {interseções};
    \node at (1.5,9.5) {t-normas};
    \node at (10.5,10.5)  {uniões};
    \node at (10.5,9.5) {t-conormas};

    \draw[<->] (nimn.south) -- (0,1) -- (12,1)--(numax.south);
    \draw[<->] (min.south) -- (3,2) -- (9,2) --(max.south) ;

    \draw[->] (n1.east) -- (n2.west);
    \draw[->] (n2.east) -- (n3.west);


    \draw[thick] (0,5) -- (12,5);
    \node at (1.5,5) [above] {associativa};
    \node at (6,5) [above]   {idempotente};
    \node at (10.5,5) [above] {associativa};
    \node at (6,2) [below] {operações padrão};
    \node at (6,1) [below] {operações drásticas};
        
    \node (m1) at (3,7) [bola] {$-\infty$};
    \node (m2) at (6,7) [bola] {$\lambda$};
    \node (m3) at (9,7) [bola] {$\infty$};
    \draw[->] (m1.east) -- (m2.west);    
    \draw[->] (m2.east) -- (m3.west);    
    \node at (6,7.5) {médias};
    
    \draw[thick] (nimn.north) -- (n1.south);
    \draw[thick] (min.north) -- (m1.south) ;
    \draw[thick] (m1.north) -- (n3.south);
    \draw[thick] (max.north) -- (m3.south);
    \draw[thick] (m3.north) -- (n4.south);
    \draw[thick] (numax.north) -- (n6.south);

    \end{tikzpicture}
    \caption{Variáveis e posições}
  
\end{figure}



\begin{figure}[hbt]
\centering
\begin{tikzpicture}
 \draw[decorate,decoration={coil, aspect=0}] (0,0) circle (1.5cm);
\end{tikzpicture}
    \caption{Decorations need many libraries (at least)}
    
\end{figure}



\begin{figure}[hbt]
\centering
\begin{tikzpicture}
\fill[red,nearly transparent] (2,0) rectangle (3,2.5);
\fill[blue,nearly transparent] (0,1) rectangle (7,1.5);

\draw [decorate,decoration={brace}]
(2,2.5) -- (3,2.5)node [black,midway,above] {\footnotesize
 atributo };

\draw [decorate,decoration={brace}]
(0,1) -- (0,1.5)node [black,midway,left] {\footnotesize tupla};

\foreach \x in {0,1,2,3,4,5,6,7}
    \draw (\x,0) -- (\x,2.5);
\foreach \y in {0,.5,1,1.5,2,2.5}    
    \draw (0,\y) -- (7,\y);

\draw [decorate,decoration={brace,mirror,amplitude=7pt}]
(0,0) -- (7,0)node [black,midway,below=4pt] {\footnotesize{relação}};


\end{tikzpicture}


    \caption{Modelo abstrato de uma relação como uma tabela - usando decorations de chaves}

\end{figure}



\begin{figure}
\centering
\begin{tikzpicture}[description/.style={fill=white,inner sep=2pt},show background grid,show background rectangle]
    \useasboundingbox (-6.5,-5) rectangle (6.5,4);
    \scope[transform canvas={scale=.7}]
      \matrix (m) [matrix of math nodes, row sep=31pt,
    column sep=40pt, text height=1.5ex, text depth=0.25ex]
    { \\ \\ \\ \underset{v_0}{\bullet} & \underset{v_1}{\bullet} & \underset{v_2}{\bullet} & \underset{v_3}{\bullet} & \cdots & \underset{v_{\lambda - 3}}{\bullet} & \underset{v_{\lambda - 2}}{\bullet} & \underset{v_{\lambda - 1}}{\bullet} & \underset{v_\lambda}{\bullet} \\ \\ \\ \\ \\ \\ \\ \\ \underset{\hat v_0}{\bullet} & \underset{\hat v_1}{\bullet} & \underset{\hat v_2}{\bullet} & \underset{\hat v_3}{\bullet} & \cdots & \underset{\hat v_{\lambda - 3}}{\bullet} & \underset{\hat v_{\lambda - 2}}{\bullet} & \underset{\hat v_{\lambda - 1}}{\bullet} & \underset{\hat v_\lambda}{\bullet} \\};
    \path[->,font=\scriptsize]
    (m-4-1) edge [bend left=20] node[auto] {$1$} (m-4-2)
    (m-4-2) edge [bend left=20] node[auto] {$\lambda$} (m-4-1)
            edge [bend left=20] node[auto] {$2$} (m-4-3)
    (m-4-3) edge [bend left=20] node[auto] {$\lambda - 1$} (m-4-2)
            edge [bend left=20] node[auto] {$3$} (m-4-4)
    (m-4-4) edge [bend left=20] node[auto] {$\lambda - 2$} (m-4-3)
            edge [bend left=20] node[auto] {$4$} (m-4-5)
    (m-4-5) edge [bend left=20] node[auto] {$\lambda - 3$} (m-4-4)
            edge [bend left=20] node[auto] {$\lambda - 3$} (m-4-6)
    (m-4-6) edge [bend left=20] node[auto] {$4$} (m-4-5)
            edge [bend left=20] node[auto] {$\lambda - 2$} (m-4-7)
    (m-4-7) edge [bend left=20] node[auto] {$3$} (m-4-6)
            edge [bend left=20] node[auto] {$\lambda - 1$} (m-4-8)
    (m-4-8) edge [bend left=20] node[auto] {$2$} (m-4-7)
            edge [bend left=20] node[auto] {$\lambda$} (m-4-9)
    (m-4-9) edge [bend left=20] node[auto] {$1$} (m-4-8);
    \draw[<-] (m-4-1) .. controls +(70:50pt) and +(110:50pt) .. node[pos=.5, above]{\scriptsize $\lambda$} (m-4-1);
    \draw[<-] (m-4-2) .. controls +(70:50pt) and +(110:50pt) .. node[pos=.5, above]{\scriptsize $\lambda - 2$} (m-4-2);
    \draw[<-] (m-4-3) .. controls +(70:50pt) and +(110:50pt) .. node[pos=.5, above]{\scriptsize $\lambda - 4$} (m-4-3);
    \draw[<-] (m-4-4) .. controls +(70:50pt) and +(110:50pt) .. node[pos=.5, above]{\scriptsize $\lambda - 6$} (m-4-4);
    \draw[<-] (m-4-6) .. controls +(70:50pt) and +(110:50pt) .. node[pos=.5, above]{\scriptsize $6 - \lambda$} (m-4-6);
    \draw[<-] (m-4-7) .. controls +(70:50pt) and +(110:50pt) .. node[pos=.5, above]{\scriptsize $4 - \lambda$} (m-4-7);
    \draw[<-] (m-4-8) .. controls +(70:50pt) and +(110:50pt) .. node[pos=.5, above]{\scriptsize $2 - \lambda$} (m-4-8);
    \draw[<-] (m-4-9) .. controls +(70:50pt) and +(110:50pt) .. node[pos=.5, above]{\scriptsize $-\lambda$} (m-4-9);
    \path[draw] (-4.1, -2) rectangle (3.7, 0);
    \draw (-3.5, -1) node {$e$:};
    \draw[<-] (-3.2, -1) .. controls +(-20:18pt) and +(200:18pt) .. (-1.7, -1);
    \draw (-.5, -1) node {$f$:};
    \draw[->] (-.2, -1) .. controls +(20:18pt) and +(160:18pt) .. (1.3, -1);
    \draw (2.5, -1) node {$h$:};
    \draw[<-] (3.1, -1.5) .. controls +(70:40pt) and +(110:40pt) ..  (2.9, -1.5);
    \draw (-.5, 5) node {$V(\lambda)$};
    \path[->,font=\scriptsize]
    (m-12-1) edge [bend left=20] node[auto] {$\lambda$} (m-12-2)
    (m-12-2) edge [bend left=20] node[auto] {$1$} (m-12-1)
            edge [bend left=20] node[auto] {$\lambda - 1$} (m-12-3)
    (m-12-3) edge [bend left=20] node[auto] {$2$} (m-12-2)
            edge [bend left=20] node[auto] {$\lambda - 2$} (m-12-4)
    (m-12-4) edge [bend left=20] node[auto] {$3$} (m-12-3)
            edge [bend left=20] node[auto] {$\lambda - 3$} (m-12-5)
    (m-12-5) edge [bend left=20] node[auto] {$4$} (m-12-4)
            edge [bend left=20] node[auto] {$4$} (m-12-6)
    (m-12-6) edge [bend left=20] node[auto] {$\lambda - 3$} (m-12-5)
            edge [bend left=20] node[auto] {$3$} (m-12-7)
    (m-12-7) edge [bend left=20] node[auto] {$\lambda - 2$} (m-12-6)
            edge [bend left=20] node[auto] {$2$} (m-12-8)
    (m-12-8) edge [bend left=20] node[auto] {$\lambda - 1$} (m-12-7)
            edge [bend left=20] node[auto] {$1$} (m-12-9)
    (m-12-9) edge [bend left=20] node[auto] {$\lambda$} (m-12-8);
    \draw[<-] (m-12-1) .. controls +(70:50pt) and +(110:50pt) .. node[pos=.5, above]{\scriptsize $\lambda$} (m-12-1);
    \draw[<-] (m-12-2) .. controls +(70:50pt) and +(110:50pt) .. node[pos=.5, above]{\scriptsize $\lambda - 2$} (m-12-2);
    \draw[<-] (m-12-3) .. controls +(70:50pt) and +(110:50pt) .. node[pos=.5, above]{\scriptsize $\lambda - 4$} (m-12-3);
    \draw[<-] (m-12-4) .. controls +(70:50pt) and +(110:50pt) .. node[pos=.5, above]{\scriptsize $\lambda - 6$} (m-12-4);
    \draw[<-] (m-12-6) .. controls +(70:50pt) and +(110:50pt) .. node[pos=.5, above]{\scriptsize $6 - \lambda$} (m-12-6);
    \draw[<-] (m-12-7) .. controls +(70:50pt) and +(110:50pt) .. node[pos=.5, above]{\scriptsize $4 - \lambda$} (m-12-7);
    \draw[<-] (m-12-8) .. controls +(70:50pt) and +(110:50pt) .. node[pos=.5, above]{\scriptsize $2 - \lambda$} (m-12-8);
    \draw[<-] (m-12-9) .. controls +(70:50pt) and +(110:50pt) .. node[pos=.5, above]{\scriptsize $-\lambda$} (m-12-9);
    \draw (-.5, -4) node {$V(\lambda)^\ast$};
    \endscope
\end{tikzpicture}
\caption{
The manual states: “Tracking of the picture size is (locally) switched off …”
This means that the bounding box is lost, which needs to be specified manually via the useasboundingbox path (= path[use as bounding box]) which also needs to be outside of the scope that has transform canvas applied to.
You might consider the necessarity to transform your whole picture (this also affects font-sizes!).
} 
\end{figure}




\begin{figure}[hbt]
\centering
\begin{tikzpicture}

\draw  (0,0) rectangle (10,10) ;
\draw  (1,1) rectangle (6,6);
\draw  (4,4) rectangle (9,9);
\draw (3,1) -- (9,7) ;

\draw[dashed] (5,3) -- (6,3) ;
\draw[dashed] (7,5) -- (7,4) ;

\node at (0.1,9.5) [right] {Medidas Fuzzy};
\node at (0.1,9) [right,align=left,font=\fontsize{5pt}{5pt}\selectfont]
{monotônica e contínua \\ ou semicontínua};

\node at (4,5.7) [right,align=left,font=\fontsize{8pt}{8pt}\selectfont] 
{ Medidas de \\ Probabilidade};
\node at (4,5.2) [right,align=left,font=\fontsize{5pt}{5pt}\selectfont]
{aditiva};

\node at (1,5.7)
[right,align=left,font=\fontsize{8pt}{8pt}\selectfont] 
{Medidas de Crença};
\node at (1,5.2)
[right,align=left,font=\fontsize{5pt}{5pt}\selectfont]
{superaditiva e contínua \\ por cima };


\node at (4,8.7)
[right,align=left,font=\fontsize{8pt}{8pt}\selectfont]  
{Medidas de Plausabilidade};
\node at (4,8.2)
[right,align=left,font=\fontsize{5pt}{5pt}\selectfont]
{subaditiva e contínua \\ por baixo };

\node at (7,4.5)
[above right,align=left,font=\fontsize{8pt}{8pt}\selectfont]  
{Medidas de \\ Necessidade};

\node at (4,1.5)
[above right,align=left,font=\fontsize{8pt}{8pt}\selectfont]  
{Medidas de \\ Possibilidade};

\node at (7,3)
[right,align=left,font=\fontsize{8pt}{8pt}\selectfont] 
{Nítido};

\draw[->] (7,3) -- (6.5,4.3);
\draw[->] (7,3) -- (5.7,3.5);




\end{tikzpicture}


    \caption{Klir e os tipos de medida}
\end{figure}





















\begin{figure}
    \centering
\begin{tikzpicture}
\node[black,draw,rectangle,
    minimum width = 3cm, 
    minimum height = 4cm] (a) at (0,0) {$\gxmat{A'}$};
\node at (2cm,0) {$=$};
\node at (a.south) [below] {$t \times d$};
\node at (a.west) [left] {termos};
\node at (a.north) [above] {documentos};

\node[black,draw,rectangle,
    minimum width = 1cm, 
    minimum height = 4cm] (uu) at (3cm,0) {$\gxmat{U'}$};
\node[black,draw,rectangle,dashed,
    minimum width = 2cm, 
    minimum height = 4cm] (u) at (3.5cm,0) {};
\node at (uu.south) [below] {$d \times r$};    
\node at (5cm,0) {$\times$};


\node[black,draw,rectangle,
    minimum width = 1cm, 
    minimum height = 1cm] (ss) at (6cm,0.5) {};
\node[black,draw,rectangle,dashed,
    minimum width = 2cm, 
    minimum height = 2cm] (s) at (6.5cm,0) {};
\node at (ss.south) [below] {$r \times r$};
\node (sl) at (ss) {$\Sigma'$};
\draw[-] ($(ss.north west)+(0.1,-0.1)$) -- (sl.north west);
\draw[-] ($(ss.south east)+(-0.1,0.1)$) -- (sl.south east);

\node at (8cm,0) {$\times$}    ;


\node[black,draw,rectangle,
    minimum width = 3cm, 
    minimum height = 1cm] (vv) at (10cm,0.5) {};
\node[black,draw,rectangle,dashed,
    minimum width = 3cm, 
    minimum height = 2cm] (v) at (10cm,0) {};
\node (vl) at (vv) {$\gxmat{V}^T$};   
\node at (vv.south) [below] {$r \times d$};  
 \end{tikzpicture}
    \caption{LSI}
    \label{fig:lsi}
\end{figure}




\begin{figure}
    \centering
\begin{tikzpicture}
\tikzstyle{textbf} = [text centered,font=\bfseries]
\pgfmathsetmacro \s {6};
	\draw[red,fill = red!80] (0,0) rectangle (\s,\s);
	\draw[red,fill = red!60] (0,0) rectangle (\s*.75,\s*.75);
	\draw[red,fill = red!40] (0,0) rectangle (\s/2,\s/2);
	\draw[red,fill = red!20] (0,0) rectangle (\s/3,\s/3);
	\node[textbf] at (\s/6,\s/6) {Descriçao};
    \node[textbf] at (\s*.25,\s*.4) {Diagnóstico};
    \node[textbf] at (\s/2,\s*.6) {Predição};
    \node[textbf] at (\s*.75,\s/8*7)  {Prescrição};    
	\draw[very thick,-latex] (0,0) -> (\s*1.2,0)
	node [midway, below] {Complexidade};
	\draw[very thick,-latex] (0,0) -> (0,\s*1.2)
	node [midway, above ,rotate=90] {Valor};
\end{tikzpicture}
    \caption{Caption}
    \label{fig:my_labely}
\end{figure}

\begin{figure}
\centering
\begin{tikzpicture}

\filldraw[fill=green!40, draw=black,thick] (0,0.5) rectangle (4,6.5);
\filldraw[fill=red!40, draw=black,thick] (4,0.5) rectangle  (8,6.5);

\draw[fill=red,thick] (4,1.5) arc (-90:90:2cm);
\draw[fill=green,thick] (4,1.5) arc (-90:-270:2cm) -- (4,1.5);

\path (4,1.5) arc  (-90:-275:2cm) coordinate (A) ;

\node at (4,3) [above=.5cm,rotate=90,align=center] {$VP=A\cap B$ \\ relevantes \\ recuperados};
\node at (4,3) [above=.5cm,rotate=270,align=center] {$FP=B-A$ \\ não relevantes \\ recuperados};

\node at (2,6) [below,align=center] {$FN = A-B$} ;
\node at (6,6) [below,align=center] {$VN = C-A-B$};

\node at (2,0.5) [below,align=center] {relevantes \\ $A$ };
\node at (6,0.5) [below,align=center] {não relevantes \\ $C-A$};
\draw[-Latex] (4.3,6.6) node [above,align=center] {recuperados \\  $B$} -- (A) ;
\end{tikzpicture}
    \caption{Information Retrieval}
    \label{fig:my_label1}
\end{figure}






\begin{figure}
\centering
\begin{tikzpicture}[%
term/.style = {rounded corners = 6,%
minimum width = 6ex,%
minimum height = 4ex},%
every node/.style = {draw},%
every path/.style = {-latex},%
nd/.style = {minimum width = 10pt,%
minimum height = 36pt}]
\node[term] (t1) at (0,0) {T1};
\node[term,below = of t1] (t2) {T2};
\node[term,below = of t2] (t3) {T3};
\node[term,below = of t3] (t4) {T4};
\node[doc,nd, right = 3cm of t1] (d1) {D1};
\node[doc,nd, right = 3cm of t4] (d3) {D3};
\node[doc,nd] at ($(d1)!.5!(d3)$) (d2) {D2};
\draw (t1) -- (d1);
\draw (t2) -- (d1);
\draw (t3) -- (d1);
\draw (t3) -- (d2);
\draw (t3) -- (d3);
\draw (t4) -- (d3);
\end{tikzpicture}
\caption{Representação de um índice}
\end{figure}

\begin{figure}
\centering
\begin{tikzpicture}[%
term/.style = {rounded corners = 6,%
minimum width = 6ex,%
minimum height = 4ex},%
every node/.style = {draw},%
every path/.style = {-latex},%
nd/.style = {minimum width = 10pt,%
minimum height = 36pt}]
\node[term] (t1) at (0,0) {T1};
\node[term,below = of t1] (t2) {T2};
\node[term,below = of t2] (t3) {T3};
\node[term,below = of t3] (t4) {T4};
\node[cloud,right = of t2] (c1) {C1};
\node[cloud,right = of t3] (c2) {C2};
\node[doc,nd, right = 3cm of t1] (d1) {D1};
\node[doc,nd, right = 3cm of t4] (d3) {D3};
\node[doc,nd] at ($(d1)!.5!(d3)$) (d2) {D2};
\draw (t1) -- (c1);
\draw (t2) -- (c1);
\draw (t3) -- (c1);
\draw (t3) -- (c2);
\draw (t4) -- (c2);
\draw (c1) -- (d1);
\draw (c1) -- (d2);
\draw (c2) -- (d2);
\draw (c2) -- (d3);
\end{tikzpicture}
\caption{Representação do LSI/LSA}
\end{figure}



\begin{figure}
    \centering

\startchronology[startyear=1986,stopyear=2008,color=NavyBlue,dates=false]
\chronoperiode[color=lightgray,dates=false]{1986}{1991}{}
\setupchronoevent{colorbox=white}
\chronoevent{1986}{SGML}
\chronoevent{1991}{HTML}


\chronoevent[markdepth=-2cm]{1995}{JS}

\chronoevent[markdepth=1.5cm]{1996}{CSS}
\chronoevent[markdepth=2.5cm]{1999}{CSS 3}
\chronoevent[markdepth=1.5cm]{1998}{CSS 2}

\chronoevent{2005}{Ajax}
\chronoevent{2008}{HTML 5}

\chronoevent[markdepth=-1cm]{1996}{XML}

\chronoevent{1994}{HTML 2}
\chronoevent{1997}{HTML 4}
\chronoevent[markdepth=-1cm]{1991}{WWW}
\stopchronology

    \caption{Cronologia}
    \label{fig:my_labelx}
\end{figure}







\begin{figure}
\resizebox{\textwidth}{!}{
\begin{forest}
every path/.style={-latex}
[Métodos \\ de fusão \\ de termos,
for tree={align=center},for tree=draw,for tree=rounded corners=4,fill = yellow,for tree={minimum height=48pt,minimum width=6em,edge=-latex,anchor=center}
    [Manual]
    [Automático, fill = yellow
        [Termo\\Único,fill=yellow
            [Linguístico 
                [\textit{Finite-State}\\\textit{Transducer}]
            ]
            [Não \\ Linguístico, fill=yellow
                [Remoção \\de Afixos,fill=yellow
                    [Remoção \\ de Sufixos,fill=yellow
                        [Casamento \\ mais longo,fill=yellow]
                        [Remoção \\ Simples]
                    ]
                ]
                [Estatísticos
                    [n-gramas]
                    [HMM]
                    [Minial \\ Description \\ Length]
                    [Variedade de \\ Sucessores]
                ]
                [Busca em \\ Tabela]
                [Misto
                    [Flexões e \\ Derivaçoes]
                    [Baseados \\ em Corpus]
                ]
            ]
        ]
        [Termos\\com várias\\palavras
            [Não \\ Linguísticos \\ (Similaridade)
                [Co-ocorrência\\ de formas \\ de palavras]
                [Co-ocorrência \\ de n-gramas]
            ]
            [Linguísticos\\(Padrões \\\ Sintáticos)
                [\textit{Finite-State}\\\textit{Transducer}]
            ]
        ]
    ]
]
\end{forest}
}
\caption{Stemmers}

\end{figure}
\chapter{Mais!}
\begin{figure}
    \centering
 \begin{tikzpicture}
\draw[<->] (0,11) -- (0,0) -- (11,0);

\foreach \i in {0,1,...,10} \draw (\i,0)--(\i,-.1);
\foreach \i in {0,.2,...,1} \node[below] at (10*\i,0) {\pgfmathprintnumber
[fixed,fixed zerofill,precision=1,use comma]
{\i}};
\node[below] at (5.5,-.5) {Revocação};

\foreach \i in {0,1,...,10} \draw (0,\i)--(-.1,\i);
\foreach \i in {0,.2,...,1} \node[left] at (0,10*\i) {\pgfmathprintnumber
[fixed,fixed zerofill,precision=1,use comma]
{\i}};
\node[rotate=90] at (-.8,5.5) {Precisão};

\draw[very thick] (1,10) .. 
controls (8,9)  and    (3,0) 
.. (10,1) ;

\node at (4,10) {Consultas específicas};
\node at (9,2) {Consultas genéricas};


\end{tikzpicture}   
    \caption{Caption}
    \label{fig:my_label23}
\end{figure}






\chapter{Arquiteturas}


\begin{figure}
    \centering
\begin{tikzpicture}
[proc/.style = {rectangle,rounded corners=10,draw=blue,font=\sffamily,fill=blue!10,align=center,thick,minimum height = 32pt, minimum width = 6em},
app/.style = { rectangle,rounded corners=10,dashed,very thick,draw,inner sep=12pt,fill=blue!30},
seta/.style = {-{LaTeX},draw,thick,blue}]

\node[proc] (RH) at (0,0) {\textbf{Request} \\ \textbf{Handler}};

\node[right = of RH] (fake) {};

\node[proc,right = of fake] (RW)  
{\textbf{Response} \\ \textbf{Writer}};    

\node[proc,below = of fake,yshift=-12pt] (In)  
{ \textbf{Index}};    
\node[proc,above = of RH] (Q)  
{ \textbf{Query}};    
\node[proc,above = of RW] (R)  
{ \textbf{Response}};    
\node[proc,below = of In] (UH)  
{ \textbf{Update Handler}};  



\begin{pgfonlayer}{bm}
\node[app,fill=blue!20,draw,fit=(In)] (Lucene)  {};
\end{pgfonlayer}
\begin{pgfonlayer}{background}
\node[app,fit=(RH)(RW)(Lucene)(UH)] (Solr) {};
\end{pgfonlayer}

\node[text=white,fill=purple,anchor=west] (Name1) at (Solr.north west) {\textbf{Solr}};
\node[text=white,fill=purple,anchor=west] (Name2) at (Lucene.north west) {\textbf{Lucene}};


\node[proc, right = of UH,xshift=5em] (DS) {\textbf{Data Source}};

\path[seta](Q) -- (RH);
\path[seta](RW) -- (R);
\path[seta](RH) |- (In);
\path[seta](In) -| (RW);
\path[seta](UH) -- (In);
\path[seta](DS) -- (UH);
\end{tikzpicture}



    
    \caption{Solr Architecture}
    \label{fig:my_solrarh}
\end{figure}

\begin{figure}
    \centering
    \label{fig:qrocha}
    \begin{tikzpicture}
    \tikzmath{  \yd = 1.3 ; 
                \xd = 1.75 ;
                \x1 = 0 ; 
                \y1 = 0 ;
                \x2 = \x1 + \xd*1.5 ;
                \x3 = \x2 + 2*\xd ;
                \x4 = \x3 + \xd ;
                \x6 = \x4 + \xd ;
                \x5 = \x4 + 1.5*\xd ;
                \y2 = \y1 + \yd ;
                \y3 = \y2 + \yd ;
                \y4 = \y3 + \yd ;
                \y5 = \y4 + \yd ;
                \y6 = \y5 + \yd ;
                \y7 = \y6 + \yd ;         
                }
    \node[cloud,draw,aspect=2] (software) at (\x2,\y1) {Software};
    \node[ellipse,align=center,draw] (avaliacao) at (\x3,\y2) {Avaliação};
    \node[draw=black,rectangle] (medidas) at (\x2,\y3) {Medidas};
    \node[draw=black,rectangle] (crit) at (\x4,\y3) {Critérios};
    \node[rectangle,align=center,draw] (proc) at (\x5,\y2) {Processo de\\ Avaliação};
    \node[draw=black,ellipse,draw] (agre) at (\x2,\y4) {Agregar};
    \node[draw=black,rectangle] (medagre) at (\x2,\y5) {Medidas Agregadas};
    \node[draw=black,rectangle] (atrib) at (\x4,\y5)  {Atributos} edge [in=30,out=60,loop right] node[below=0.3]{são  compostos por} ();
    \node[draw=black,rectangle] (obj) at (\x4,\y6) {Objetivos};
    \node[align=center] (rq) at (\x2,\y7) {\textbf{Relações} \\  \textbf{Quantitativas}};
    \node[align=center] (rl) at (\x4,\y7) {\textbf{Relações} \\ \textbf{Lógicas}};
    \node[rectangle,draw,rotate=90] (fz) at (\x1,\y4) {Funções Fuzzy};
    
    \draw[thick,dashed] (avaliacao.north) -- (\x3,\y7) ;
    \draw[->] (software) -- (avaliacao);
    \draw[->] (avaliacao) -- (medidas.south east);
    \draw[->] (medidas) -- (agre);
    \draw[->] (agre) -- (medagre);
    \draw[->] (medagre) -- node [midway,above] {quantificam} (atrib);
    \draw[->] (medidas) --node [midway,above] {quantificam}  (crit);
    \draw[->] (crit) -- node [midway, above right] {determinam} (proc);
    \draw[->] (atrib) --node [midway,right] {são compostos por}  (crit);
\draw[->] (obj) -- node [midway,right] {são atingidos por}  (atrib);
    
    \draw[->] (fz.east) |- node [midway,above] {interpretam} (medagre.west);
    \draw[->] (fz.west) |- node [midway,below] {interpretam} (medidas.west);
\end{tikzpicture} 
    \caption{Modelo Fuzzy de Qualidade Rocha }
    
\end{figure}


\begin{figure}
    \centering
    \begin{tikzpicture}%
[every node/.style={%
draw,%
black,%
align=center,%
node distance=1cm and 1cm,%
minimum height = 1.5cm,%
minimum width = 3cm,
},
every path/.style={%
black,
Latex-Latex,
thick
}%
]

\node (CPU) at (0,0) {CPU};
\node (MP) [left = of CPU]  {Memória \\ Princial};
\node (MA) [right = of CPU]  {Memória \\ Auxiliar};
\node (ES) [above = of CPU]  {Dispositivo \\ de Entrada \\ e Saída};

\draw (CPU.east) -- (MA.west);
\draw (CPU.west) -- (MP.east);
\draw (CPU.north) -- (ES.south);

\end{tikzpicture}
    \caption{Computador Simples, usa estilos genéricos}
    \label{fig:comp1s}
\end{figure}



\autoref{fig:comp12s} uses different interesting commands, such as calculating point position by the intersections of a vertical and a a horizontal reference (using \verb!(h-|v)! or \verb!(v|-h)!, general and specific styles, style overwriting, y and x shifts in points, etc.

\begin{verbatim}
\begin{tikzpicture}%
[every node/.style={%
draw,%
black,%
align=center,%
node distance=1cm and 3cm,%
minimum height = 1.5cm,%
minimum width = 3cm,
},
registro/.style={%
minimum height = 18pt,%
minimum width = 4cm,
node distance=.5cm and 3cm,%
},
every path/.style={%
black,
Latex-Latex,
thick
}%
]

\node (UdC) at (0,0) {Unidade \\ de Controle};
\node (ALU) [above = of UdC]  {ALU};
\node (B) [above = of ALU]  {Buffer};

\node[registro] (Pilha) [right = of UdC] {Registro de Pilha};
\node[registro] (End) [above = of Pilha] {Registro de Endereço};
\node[registro] (Regn) [above = of End]  {Registro Geral N};
\node[registro,draw=none] (Regp) [above = of Regn]  {...};
\node[registro] (Reg2) [above = of Regp]  {Registro Geral 2};
\node[registro] (Reg1) [above = of Reg2]  {Registro Geral 1};
 
\node[draw=none] (ini) at  (2,-1) {};
\node[draw=none]  at (2,7) {};
 %($(UdC.south west)!.5!(Pilha.east)$)
\draw[-,ultra thick] ([yshift=-10pt]3,0|-Pilha.west) -- ([yshift=10pt]3,0|-Reg1.west);
\draw (Pilha.west) -- (3,0|-Pilha.west);
\draw (Reg1.west) -- (3,0|-Reg1.west); 
\draw (Reg2.west) -- (3,0|-Reg2.west);
\draw (Regn.west) -- (3,0|-Regn.west);
\draw (End.west) -- (3,0|-End.west);  
\draw (ALU.east) -- (3,0|-ALU.east);  
\draw (UdC.east) -- (3,0|-UdC.east);  
\draw (B.east) -- (3,0|-B.east);  

\draw[-,ultra thick] ([xshift=15pt,yshift=15pt]Reg1.north)
-- node [pos=0.5,draw=none,above] {Barramento 
de Dados} ([xshift=-15pt,yshift=15pt]Reg1.north-|B.north);
\draw ([yshift=15pt]Reg1.north)
-- (Reg1.north) ;
\draw ([yshift=15pt]Reg1.north-|B.north)
-- (B.north) ;

\node[draw=none] (T1) at ([yshift=-10pt]Pilha.south) {Barramento Interno de Dados};
\draw[Latex-,dotted] ([yshift=-10pt]3,0|-Pilha.west) -- (T1.west) ;

\draw[ultra thick,-] ([yshift=15pt,xshift=-15pt]B.west) --
([yshift=-15pt,xshift=-15pt]UdC.south west) --
node [draw=none,below,pos=0.5] {Barramento de Controle}
([yshift=-15pt,xshift=15pt]Pilha.south east |- 
UdC.south west) --
([yshift=15pt,xshift=15pt] Pilha.south east |- B.west) 
;

\draw (Pilha.east) -- ([xshift=15pt]Pilha.east -| Pilha.south east);
\draw (Pilha.east) -- ([xshift=15pt]Pilha.east -| Pilha.south east);
\draw (Pilha.east) -- ([xshift=15pt]Pilha.east -| Pilha.south east);
\draw (End.east) -- ([xshift=15pt]End.east -| Pilha.south east);

\draw (UdC.west) -- ([xshift=-15pt]UdC.west -| UdC.south west);
\draw (ALU.west) -- ([xshift=-15pt]ALU.west -| UdC.south west);
\draw (B.west) -- ([xshift=-15pt]B.west -| UdC.south west);

\end{tikzpicture}
\end{verbatim}




\begin{figure}
    \centering
\begin{tikzpicture}%
[every node/.style={%
draw,%
black,%
align=center,%
node distance=1cm and 3cm,%
minimum height = 1.5cm,%
minimum width = 3cm,
},
registro/.style={%
minimum height = 18pt,%
minimum width = 4cm,
node distance=.5cm and 3cm,%
},
every path/.style={%
black,
Latex-Latex,
thick
}%
]

\node (UdC) at (0,0) {Unidade \\ de Controle};
\node (ALU) [above = of UdC]  {ALU};
\node (B) [above = of ALU]  {Buffer};

\node[registro] (Pilha) [right = of UdC] {Registro de Pilha};
\node[registro] (End) [above = of Pilha] {Registro de Endereço};
\node[registro] (Regn) [above = of End]  {Registro Geral N};
\node[registro,draw=none] (Regp) [above = of Regn]  {...};
\node[registro] (Reg2) [above = of Regp]  {Registro Geral 2};
\node[registro] (Reg1) [above = of Reg2]  {Registro Geral 1};
 
\node[draw=none] (ini) at  (2,-1) {};
\node[draw=none]  at (2,7) {};
 %($(UdC.south west)!.5!(Pilha.east)$)
\draw[-,ultra thick] ([yshift=-10pt]3,0|-Pilha.west) -- ([yshift=10pt]3,0|-Reg1.west);
\draw (Pilha.west) -- (3,0|-Pilha.west);
\draw (Reg1.west) -- (3,0|-Reg1.west); 
\draw (Reg2.west) -- (3,0|-Reg2.west);
\draw (Regn.west) -- (3,0|-Regn.west);
\draw (End.west) -- (3,0|-End.west);  
\draw (ALU.east) -- (3,0|-ALU.east);  
\draw (UdC.east) -- (3,0|-UdC.east);  
\draw (B.east) -- (3,0|-B.east);  

\draw[-,ultra thick] ([xshift=15pt,yshift=15pt]Reg1.north)
-- node [pos=0.5,draw=none,above] {Barramento 
de Dados} ([xshift=-15pt,yshift=15pt]Reg1.north-|B.north);
\draw ([yshift=15pt]Reg1.north)
-- (Reg1.north) ;
\draw ([yshift=15pt]Reg1.north-|B.north)
-- (B.north) ;

\node[draw=none] (T1) at ([yshift=-10pt]Pilha.south) {Barramento Interno de Dados};
\draw[Latex-,dotted] ([yshift=-10pt]3,0|-Pilha.west) -- (T1.west) ;

\draw[ultra thick,-] ([yshift=15pt,xshift=-15pt]B.west) --
([yshift=-15pt,xshift=-15pt]UdC.south west) --
node [draw=none,below,pos=0.5] {Barramento de Controle}
([yshift=-15pt,xshift=15pt]Pilha.south east |- 
UdC.south west) --
([yshift=15pt,xshift=15pt] Pilha.south east |- B.west) 
;

\draw (Pilha.east) -- ([xshift=15pt]Pilha.east -| Pilha.south east);
\draw (Pilha.east) -- ([xshift=15pt]Pilha.east -| Pilha.south east);
\draw (Pilha.east) -- ([xshift=15pt]Pilha.east -| Pilha.south east);
\draw (End.east) -- ([xshift=15pt]End.east -| Pilha.south east);

\draw (UdC.west) -- ([xshift=-15pt]UdC.west -| UdC.south west);
\draw (ALU.west) -- ([xshift=-15pt]ALU.west -| UdC.south west);
\draw (B.west) -- ([xshift=-15pt]B.west -| UdC.south west);

\end{tikzpicture}    
    \caption{Abstração da CPU, usa referências com shift e cálculo de pontos por interseção de uma referência vertical com uma horizontal, estilos genéricos e específicos}
    \label{fig:comp12s}
\end{figure}


\begin{figure}[hbt]
    \centering
\begin{tikzpicture}[seta/.style = {-latex},
adf/.style = {minimum width = 2cm}]
  \node[doc,cascaded] (fet) at (0,0) {Fetch list};
  \node[draw=none,right = of fet] (int) {};
  \node[doc,cascaded, right = of int] (u) {Updates};
  \node[doc,cascaded, right = of u] (c) {Content};
  \node[doc,cascaded, right = of c] (i) {Indexes};

\node[server,adf,above = 2.0cm of int,label=below:{web db}] (wd) {};
\node[server,adf,below = 2.0cm of int,label=below:{fetchers}] (fs) {};
\node[server,adf,above = of i,label=above:{indexers}] (is) {};
\node[server,adf,below = of i,label=right:{searchers}] (ss) {};
\node[server,adf,below = of ss,label=below:{web servers}] (ws) {};

\draw[seta] (wd.west) to[bend right] (fet) ;
\draw[seta] (u) to[bend right] (wd.east);
\draw[seta] (fet) to[bend right] (fs.west);
\draw[seta] (fs.east) to[bend right] (u);
\draw[seta] (u) -- (c);
\draw[seta] (is) -- (i);
\draw[seta] (i) -- (ss);
\draw[seta] (c) to[bend left] (is.west);
\draw[seta] (c) to[bend right] (ss.west);
\draw[latex-latex] (ss) -- (ws);
\end{tikzpicture}
\caption{Lucene?}
\end{figure}

\section{Mecanismos de Busca}

\begin{figure}
    \centering
\resizebox{\textwidth}{!}{%
\begin{tikzpicture}[every node/.style = {rectangle,draw,%
minimum width = 15ex,
rounded corners=4,
minimum height = 24pt,
node distance= 18pt and 24pt,
align = center
},
every path/.style = {-latex},
db/.style = {cylinder, 
aspect=0.25,
rounded corners=0,
shape border rotate=90}]
\node[cloud] (web) at (0,0) {Web};
\node[right = of web] (crawler) {Crawler};
\node[db, right = of crawler] (paginas) {Páginas};
\node[right = of paginas] (prepro) {Pré-Processamento};
\node[ellipse,fill = yellow,rounded corners=0,, below = of web] (anu) {Anunciante};
\node[right = of anu] (sisanu) {Sistema de \\ Anúncios};
\node[below = of prepro] (outras) {Outras \\ Análises};
\node[right = of outras] (indexador) {Indexador};
\node[right = of indexador] (arede) {Análise \\ da Rede};
\node[below = of sisanu] (apre) {Aprendizado};
\node[db,right = of apre] (muser) {Modelo \\ Usuário};
\node[db, right = of muser] (outdad) {Outros \\ Dados};
\node[db, below = of indexador] (indice) {Índice};
\node[db, below = of arede] (mrede) {Modelo \\ da Rede};
\node[below left = of apre] (obs) {Observador};
\node[ellipse,fill = yellow,below = of obs] (user) {Usuário};
\node[right = of user] (interf) {Interface};
\node[right = of interf] (mod) {Modificador \\ de Consultas};
\node[right = of mod] (busca) {Buscador};

\draw (web) -- (crawler);
\draw (crawler) -- (paginas);
\draw (paginas) -- (prepro);
\draw[latex-latex] (anu) -- (sisanu);
\draw (sisanu) -- (outras);
\draw (prepro) -- (outras);
\draw ($(prepro.east)!.5!(prepro.south east)$) -| (indexador);
\draw (prepro.east) -| (arede);
\draw (arede) -- (mrede);
\draw (indexador) -- (indice);
\draw[latex-latex] (outras) -- (outdad);
\draw (mrede) |- ($(busca.east)!.5!(busca.south east)$);
\draw (indice) |- ($(busca.east)!.5!(busca.north east)$);
\draw (outdad) -- (busca);
\draw (outdad) -- (mod);
\draw (muser) -- (mod);
\draw (muser) -- (busca);
\draw[latex-latex] (mod) -- (busca);
\draw[latex-latex] (interf) -- (mod);
\draw[latex-latex] (user) -- (interf);
\draw (user) -- (obs);
\draw (obs) -- (apre);
\draw (apre) -- (muser);
\draw (obs) -- (sisanu);
\draw (obs.west) -| ($(anu.west)+
(-.5,0)$) |- ($(outras.west)!0.5!(outras.north west)$);
\draw (prepro.north) |- ($(paginas.north)+(0,.3)$) -| (crawler.north);
\draw (obs) -| ($(outdad.south west)!.2!(outdad.south)$);
\draw[latex-latex] (interf) -- (obs);
\end{tikzpicture}
}
    \caption{Modelo genérico de um mecanismo de busca moderno}
    \label{fig:mbm}
\end{figure}

\begin{figure}
    \centering
\resizebox{\textwidth}{!}{
\begin{tikzpicture}[every node/.style = {rectangle,draw,%
minimum width = 15ex,
rounded corners=4,
minimum height = 48pt,
node distance= 36pt and 32pt,
align = center
},
n/.style ={draw=none,minimum width = 0, minimum height = 0},
every path/.style = {-latex},
ruido/.style = {dashed},
ss/.style = {latex-latex},
db/.style = {cylinder, 
aspect=0.25,
rounded corners=0,
shape border rotate=90},
c1/.style={
%minimum height = 1cm, minimum width=4cm , 
aspect=1.5},
d1/.style = {minimum width = 1cm,
minimum height = 48pt}]

\node[cloud,c1] (io) at (0,0) {Ideia \\ Original};
\node[alice, minimum width=1cm,
below right = of io] (em) {Emissor};
\node[doc,d1,right = of em] (re) {Representação \\ Escrita};
\node[right = of re] (ir) {Indexador};
\node[db, right = of ir] (rp) {Repositório};
\node[below = of ir] (mb) {Mecanismo \\ de Busca};
\node[doc,d1,cascaded,left = of mb] (od) {Outros \\ Documentos};

\node[cloud,c1,below left = of em] (ni) {Necessidade \\ da \\ Informação};
\node[bob, minimum width=1cm,below right = of ni] (rc) {Receptor};
\node[cloud,c1, left = of rc,aspect=1.7 ] (ii) {Interpretação};
\node[doc,d1,right = of rc] (rn) {Representação \\ da Necessidade};

\draw[ruido,ss] (io) to[bend left]node[n,midway,above,yshift=4pt] {1} (em);
\draw[ruido,ss] (rc) to[bend left] node[n,midway,above] {g}  (ii);
\draw[ruido,ss] (rc) to[bend right] node[n,midway,above,xshift=4pt] {a} (ni);
\draw[ruido] (em) --node[n,midway,above] {2}   (re);
\draw (re) --node[n,midway,above] {3} (ir);
\draw[ruido] (ir) --node[n,midway,above] {4} (rp);

\draw[ruido] (rp) --node[n,midway,above] {d} (mb);
\draw (rn) --node[n,midway,above] {c} (mb);
\draw[ruido] (mb) --node[n,midway,above] {e} (re);
\draw[ruido] (re) to[bend right=10]node[n,midway,above] {f} (rc);
\draw[ruido] (rc) --node[n,midway,above] {b} (rn);
\draw[ruido] (mb) --node[n,midway,above] {e} (od);
\draw[ruido] (od) --node[n,midway,above] {f} (rc);
\end{tikzpicture}}

    \caption{Caption Identificada 1}
    \label{fig:my_labelsxg2}

    
\end{figure}

\begin{figure}
    \centering
    
\begin{tikzpicture}[every node/.style = {rectangle,draw,%
minimum width = 15ex,
rounded corners=4,
minimum height = 24pt,
node distance= 36pt and 32pt,
align = center
},
every path/.style = {-latex},
db/.style = {cylinder, 
aspect=0.25,
rounded corners=0,
shape border rotate=90}]
\node[doc,cascaded] (h) at (0,0) {HTML};
\node[doc,cascaded, right = of h] (p) {PDF};
\node[doc,cascaded, right = of p] (w) {Word};

\node[below = of h] (et1) {Extrair Texto};
\node[below = of p] (et2) {Extrair Texto};
\node[below = of w] (et3) {Extrair Texto};

\node[below = of et2] (a) {Análise};
\node[db, below = of a] (i) {Índice};

\draw (h) -- (et1);
\draw (p) -- (et2);
\draw (w) -- (et3);
\draw (et1) -- (a);
\draw (et2) -- (a);
\draw (et3) -- (a);
\draw (a) -- (i);
\end{tikzpicture}
    \caption{Indexar}
    \label{fig:my_label2sd312}
\end{figure}



\chapter{Neural Networks}

\newcommand{\nnnode}[5]{
\node[#1,k,#2](#3) {#4};
\node[k,right = of #3]{#5};
}

\begin{figure}[hbt]
\begin{tikzpicture}[%
k/.style = {node distance = 10pt and 1em},
g/.style = {fill=green,draw,circle,minimum width = 24pt,minimum height = 24pt},
r/.style = {fill =red,draw,circle,circle,minimum width = 24pt,minimum height = 24pt},
b/.style = {fill =cyan,draw,circle,minimum width = 24pt,minimum height = 24pt},
%p/.style = {purple,draw,circle},
j/.style={fill=yellow,circle,draw,minimum width = 24pt,minimum height = 24pt},
]
\node[j,k] (ic) at (0,0)  { };
\node[k,right = of ic] {Input Cell};

%\nnnode{j}{below = of ic}{BFIC}{\circ}{Backfed Input Cell}
%\nnnode{j}{below = of BFIC}{nic} {\triangleright}{Noisy Input Cell};
\nnnode{g}{below = of ic}{hc}{}{Hidden Cell}
\nnnode{b}{below = of hc}{rc}{}{Recurrent Cell}
\nnnode{b}{below = of rc}{mc}{\circ}{Memory Cell}
\nnnode{r}{below = of mc}{oc}{}{Output Cell}

\node[j,k,below=of oc] (ff11)  { };
\node[j,k,below=of ff11] (ff12)  { };
\node[g,k,right=of ff11] (ff21)  { };
\node[g,k,below=of ff21] (ff22)  { };
\node[r,k,right=of ff21,yshift=-16pt] (ffo)  { };
\draw (ff11) -- (ff21);
\draw (ff11) -- (ff22);
\draw (ff12) -- (ff21);
\draw (ff12) -- (ff22);
\draw (ff21) -- (ffo);
\draw (ff22) -- (ffo);
\end{tikzpicture}
\caption{Tipo de NN}
\end{figure}
\begin{figure}[hbt]
    \centering
\begin{tikzpicture}[i/.style={%
node distance = .4ex and 12ex,minimum width=3.5em,
minimum height=3.5em
},
j/.style={node distance = .4ex and 12ex},
r/.style={fill=white}
]
\node[j] (C1) at (0,0){$i_1$};
\node[j,below = of C1] (C2) {$i_2$};
\node[j,below = of C2] (C3) {$\vdots$};
\node[j,below = of C3] (C4) {$i_{k-1}$};
\node[j,below = of C4] (C5) {$i_{k}$};
\node[j,below = of C5] (C6) {$i_{k+1}$};
\node[j,below = of C6] (C7) {$\vdots$};
\node[j,below = of C7] (C8) {$i_{n-1}$};
\node[j,below = of C8] (C9) {$i_{n}$};

\node[i,draw,circle,right = of C5] (cs) {$\sum$};
\node[i,draw,circle,right = of cs] (C11) {$g$};
\draw[->] (C1) to node[swap,pos=.5,r] {$w_1$} (cs);
\draw[->] (C2) to node[midway,pos=.5,r] {$w_2$}(cs);
\draw[->] (C4) to node[midway,pos=.5,r] {$w_{k-1}$}(cs);
\draw[->] (C5) to node[midway,pos=.5,r] {$w_k$}(cs);
\draw[->] (C6) to node[midway,pos=.5,r] {$w_{k+1}$}(cs);
\draw[->] (C8) to node[midway,pos=.5,r] {$w_{n-1}$}(cs);
\draw[->] (C9) to node[midway,pos=.5,r] {$w_n$}(cs);
\node[i,right = of C1] (cb) {$bias$};
\draw[->] (cb) to node[midway,pos=.5,r] {$w_{0}$}(cs);
\draw[->] (cs) to node[midway,pos=.5,r] {$s$}(C11);
\node[right = of C11] (C12) {};
\draw[->] (C11) to node[midway,pos=.5,r] {$o$}(C12);
\end{tikzpicture}
    \caption{A Single Neuron in a Neural Network}
    \label{fig:my_label21312}
\end{figure}

\begin{figure}
\begin{tikzpicture}[
every node/.style={align=center},
every path/.style={-{Latex},draw=black},
lay/.style={draw=black}
]

\node[lay] (tp2) at (0,0) {$t_{k+2}$};
\node[lay] (tp1) at (0,1) {$t_{k+1}$};
\node[lay] (tm1) at (0,3) {$t_{k-1}$};
\node[lay] (tm2) at (0,4) {$t_{k-2}$};

\node (r1) at (0,5) {Entrada};
\node (r2) at (2,5) {Projeção};
\node (r3) at (4,5) {Saída};

\node[lay] (soma) at (2,2) {Soma};
\node[lay](t0) at (4,2) {$t_k$};

\path (tp2) -> (soma);
\path (tp1) -> (soma);
\path (tm1) -> (soma);
\path (tm2) -> (soma);
\path (soma) -> (t0);


\node (cbow) at (2,-1) {\textbf{CBOW}};
%-------------
\node[lay] (wp2) at (10,0) {$t_{k+2}$};
\node[lay] (wp1) at (10,1) {$t_{k+1}$};
\node[lay] (wm1) at (10,3) {$t_{k-1}$};
\node[lay] (wm2) at (10,4) {$t_{k-2}$};

\node (rr1) at (6,5) {Entrada};
\node (rr2) at (8,5) {Projeção};
\node (rr3) at (10,5) {Saída};

\node[lay,text=white] (soma2) at (8,2) {Soma};
\node[lay](w0) at (6,2) {$t_k$};

\path  (soma2) -> (wp2);
\path  (soma2) -> (wp1);
\path  (soma2) -> (wm1);
\path  (soma2) -> (wm2);
\path (w0) -> (soma2);


\node (cbow) at (8,-1) {\textbf{Skip-Gram}};


\end{tikzpicture}
\caption{word2vec}
\end{figure}



\end{document}


