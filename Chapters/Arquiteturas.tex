\chapter{Arquiteturas}


\begin{figure}
    \centering
\begin{tikzpicture}
[proc/.style = {rectangle,rounded corners=10,draw=blue,font=\sffamily,fill=blue!10,align=center,thick,minimum height = 32pt, minimum width = 6em},
app/.style = { rectangle,rounded corners=10,dashed,very thick,draw,inner sep=12pt,fill=blue!30},
seta/.style = {-{LaTeX},draw,thick,blue}]

\node[proc] (RH) at (0,0) {\textbf{Request} \\ \textbf{Handler}};

\node[right = of RH] (fake) {};

\node[proc,right = of fake] (RW)  
{\textbf{Response} \\ \textbf{Writer}};    

\node[proc,below = of fake,yshift=-12pt] (In)  
{ \textbf{Index}};    
\node[proc,above = of RH] (Q)  
{ \textbf{Query}};    
\node[proc,above = of RW] (R)  
{ \textbf{Response}};    
\node[proc,below = of In] (UH)  
{ \textbf{Update Handler}};  



\begin{pgfonlayer}{bm}
\node[app,fill=blue!20,draw,fit=(In)] (Lucene)  {};
\end{pgfonlayer}
\begin{pgfonlayer}{background}
\node[app,fit=(RH)(RW)(Lucene)(UH)] (Solr) {};
\end{pgfonlayer}

\node[text=white,fill=purple,anchor=west] (Name1) at (Solr.north west) {\textbf{Solr}};
\node[text=white,fill=purple,anchor=west] (Name2) at (Lucene.north west) {\textbf{Lucene}};


\node[proc, right = of UH,xshift=5em] (DS) {\textbf{Data Source}};

\path[seta](Q) -- (RH);
\path[seta](RW) -- (R);
\path[seta](RH) |- (In);
\path[seta](In) -| (RW);
\path[seta](UH) -- (In);
\path[seta](DS) -- (UH);
\end{tikzpicture}



    
    \caption{Solr Architecture}
    \label{fig:my_solrarh}
\end{figure}

\begin{figure}
    \centering
    \label{fig:qrocha}
    \begin{tikzpicture}
    \tikzmath{  \yd = 1.3 ; 
                \xd = 1.75 ;
                \x1 = 0 ; 
                \y1 = 0 ;
                \x2 = \x1 + \xd*1.5 ;
                \x3 = \x2 + 2*\xd ;
                \x4 = \x3 + \xd ;
                \x6 = \x4 + \xd ;
                \x5 = \x4 + 1.5*\xd ;
                \y2 = \y1 + \yd ;
                \y3 = \y2 + \yd ;
                \y4 = \y3 + \yd ;
                \y5 = \y4 + \yd ;
                \y6 = \y5 + \yd ;
                \y7 = \y6 + \yd ;         
                }
    \node[cloud,draw,aspect=2] (software) at (\x2,\y1) {Software};
    \node[ellipse,align=center,draw] (avaliacao) at (\x3,\y2) {Avaliação};
    \node[draw=black,rectangle] (medidas) at (\x2,\y3) {Medidas};
    \node[draw=black,rectangle] (crit) at (\x4,\y3) {Critérios};
    \node[rectangle,align=center,draw] (proc) at (\x5,\y2) {Processo de\\ Avaliação};
    \node[draw=black,ellipse,draw] (agre) at (\x2,\y4) {Agregar};
    \node[draw=black,rectangle] (medagre) at (\x2,\y5) {Medidas Agregadas};
    \node[draw=black,rectangle] (atrib) at (\x4,\y5)  {Atributos} edge [in=30,out=60,loop right] node[below=0.3]{são  compostos por} ();
    \node[draw=black,rectangle] (obj) at (\x4,\y6) {Objetivos};
    \node[align=center] (rq) at (\x2,\y7) {\textbf{Relações} \\  \textbf{Quantitativas}};
    \node[align=center] (rl) at (\x4,\y7) {\textbf{Relações} \\ \textbf{Lógicas}};
    \node[rectangle,draw,rotate=90] (fz) at (\x1,\y4) {Funções Fuzzy};
    
    \draw[thick,dashed] (avaliacao.north) -- (\x3,\y7) ;
    \draw[->] (software) -- (avaliacao);
    \draw[->] (avaliacao) -- (medidas.south east);
    \draw[->] (medidas) -- (agre);
    \draw[->] (agre) -- (medagre);
    \draw[->] (medagre) -- node [midway,above] {quantificam} (atrib);
    \draw[->] (medidas) --node [midway,above] {quantificam}  (crit);
    \draw[->] (crit) -- node [midway, above right] {determinam} (proc);
    \draw[->] (atrib) --node [midway,right] {são compostos por}  (crit);
\draw[->] (obj) -- node [midway,right] {são atingidos por}  (atrib);
    
    \draw[->] (fz.east) |- node [midway,above] {interpretam} (medagre.west);
    \draw[->] (fz.west) |- node [midway,below] {interpretam} (medidas.west);
\end{tikzpicture} 
    \caption{Modelo Fuzzy de Qualidade Rocha }
    
\end{figure}


\begin{figure}
    \centering
    \begin{tikzpicture}%
[every node/.style={%
draw,%
black,%
align=center,%
node distance=1cm and 1cm,%
minimum height = 1.5cm,%
minimum width = 3cm,
},
every path/.style={%
black,
Latex-Latex,
thick
}%
]

\node (CPU) at (0,0) {CPU};
\node (MP) [left = of CPU]  {Memória \\ Princial};
\node (MA) [right = of CPU]  {Memória \\ Auxiliar};
\node (ES) [above = of CPU]  {Dispositivo \\ de Entrada \\ e Saída};

\draw (CPU.east) -- (MA.west);
\draw (CPU.west) -- (MP.east);
\draw (CPU.north) -- (ES.south);

\end{tikzpicture}
    \caption{Computador Simples, usa estilos genéricos}
    \label{fig:comp1s}
\end{figure}



\autoref{fig:comp12s} uses different interesting commands, such as calculating point position by the intersections of a vertical and a a horizontal reference (using \verb!(h-|v)! or \verb!(v|-h)!, general and specific styles, style overwriting, y and x shifts in points, etc.

\begin{verbatim}
\begin{tikzpicture}%
[every node/.style={%
draw,%
black,%
align=center,%
node distance=1cm and 3cm,%
minimum height = 1.5cm,%
minimum width = 3cm,
},
registro/.style={%
minimum height = 18pt,%
minimum width = 4cm,
node distance=.5cm and 3cm,%
},
every path/.style={%
black,
Latex-Latex,
thick
}%
]

\node (UdC) at (0,0) {Unidade \\ de Controle};
\node (ALU) [above = of UdC]  {ALU};
\node (B) [above = of ALU]  {Buffer};

\node[registro] (Pilha) [right = of UdC] {Registro de Pilha};
\node[registro] (End) [above = of Pilha] {Registro de Endereço};
\node[registro] (Regn) [above = of End]  {Registro Geral N};
\node[registro,draw=none] (Regp) [above = of Regn]  {...};
\node[registro] (Reg2) [above = of Regp]  {Registro Geral 2};
\node[registro] (Reg1) [above = of Reg2]  {Registro Geral 1};
 
\node[draw=none] (ini) at  (2,-1) {};
\node[draw=none]  at (2,7) {};
 %($(UdC.south west)!.5!(Pilha.east)$)
\draw[-,ultra thick] ([yshift=-10pt]3,0|-Pilha.west) -- ([yshift=10pt]3,0|-Reg1.west);
\draw (Pilha.west) -- (3,0|-Pilha.west);
\draw (Reg1.west) -- (3,0|-Reg1.west); 
\draw (Reg2.west) -- (3,0|-Reg2.west);
\draw (Regn.west) -- (3,0|-Regn.west);
\draw (End.west) -- (3,0|-End.west);  
\draw (ALU.east) -- (3,0|-ALU.east);  
\draw (UdC.east) -- (3,0|-UdC.east);  
\draw (B.east) -- (3,0|-B.east);  

\draw[-,ultra thick] ([xshift=15pt,yshift=15pt]Reg1.north)
-- node [pos=0.5,draw=none,above] {Barramento 
de Dados} ([xshift=-15pt,yshift=15pt]Reg1.north-|B.north);
\draw ([yshift=15pt]Reg1.north)
-- (Reg1.north) ;
\draw ([yshift=15pt]Reg1.north-|B.north)
-- (B.north) ;

\node[draw=none] (T1) at ([yshift=-10pt]Pilha.south) {Barramento Interno de Dados};
\draw[Latex-,dotted] ([yshift=-10pt]3,0|-Pilha.west) -- (T1.west) ;

\draw[ultra thick,-] ([yshift=15pt,xshift=-15pt]B.west) --
([yshift=-15pt,xshift=-15pt]UdC.south west) --
node [draw=none,below,pos=0.5] {Barramento de Controle}
([yshift=-15pt,xshift=15pt]Pilha.south east |- 
UdC.south west) --
([yshift=15pt,xshift=15pt] Pilha.south east |- B.west) 
;

\draw (Pilha.east) -- ([xshift=15pt]Pilha.east -| Pilha.south east);
\draw (Pilha.east) -- ([xshift=15pt]Pilha.east -| Pilha.south east);
\draw (Pilha.east) -- ([xshift=15pt]Pilha.east -| Pilha.south east);
\draw (End.east) -- ([xshift=15pt]End.east -| Pilha.south east);

\draw (UdC.west) -- ([xshift=-15pt]UdC.west -| UdC.south west);
\draw (ALU.west) -- ([xshift=-15pt]ALU.west -| UdC.south west);
\draw (B.west) -- ([xshift=-15pt]B.west -| UdC.south west);

\end{tikzpicture}
\end{verbatim}




\begin{figure}
    \centering
\begin{tikzpicture}%
[every node/.style={%
draw,%
black,%
align=center,%
node distance=1cm and 3cm,%
minimum height = 1.5cm,%
minimum width = 3cm,
},
registro/.style={%
minimum height = 18pt,%
minimum width = 4cm,
node distance=.5cm and 3cm,%
},
every path/.style={%
black,
Latex-Latex,
thick
}%
]

\node (UdC) at (0,0) {Unidade \\ de Controle};
\node (ALU) [above = of UdC]  {ALU};
\node (B) [above = of ALU]  {Buffer};

\node[registro] (Pilha) [right = of UdC] {Registro de Pilha};
\node[registro] (End) [above = of Pilha] {Registro de Endereço};
\node[registro] (Regn) [above = of End]  {Registro Geral N};
\node[registro,draw=none] (Regp) [above = of Regn]  {...};
\node[registro] (Reg2) [above = of Regp]  {Registro Geral 2};
\node[registro] (Reg1) [above = of Reg2]  {Registro Geral 1};
 
\node[draw=none] (ini) at  (2,-1) {};
\node[draw=none]  at (2,7) {};
 %($(UdC.south west)!.5!(Pilha.east)$)
\draw[-,ultra thick] ([yshift=-10pt]3,0|-Pilha.west) -- ([yshift=10pt]3,0|-Reg1.west);
\draw (Pilha.west) -- (3,0|-Pilha.west);
\draw (Reg1.west) -- (3,0|-Reg1.west); 
\draw (Reg2.west) -- (3,0|-Reg2.west);
\draw (Regn.west) -- (3,0|-Regn.west);
\draw (End.west) -- (3,0|-End.west);  
\draw (ALU.east) -- (3,0|-ALU.east);  
\draw (UdC.east) -- (3,0|-UdC.east);  
\draw (B.east) -- (3,0|-B.east);  

\draw[-,ultra thick] ([xshift=15pt,yshift=15pt]Reg1.north)
-- node [pos=0.5,draw=none,above] {Barramento 
de Dados} ([xshift=-15pt,yshift=15pt]Reg1.north-|B.north);
\draw ([yshift=15pt]Reg1.north)
-- (Reg1.north) ;
\draw ([yshift=15pt]Reg1.north-|B.north)
-- (B.north) ;

\node[draw=none] (T1) at ([yshift=-10pt]Pilha.south) {Barramento Interno de Dados};
\draw[Latex-,dotted] ([yshift=-10pt]3,0|-Pilha.west) -- (T1.west) ;

\draw[ultra thick,-] ([yshift=15pt,xshift=-15pt]B.west) --
([yshift=-15pt,xshift=-15pt]UdC.south west) --
node [draw=none,below,pos=0.5] {Barramento de Controle}
([yshift=-15pt,xshift=15pt]Pilha.south east |- 
UdC.south west) --
([yshift=15pt,xshift=15pt] Pilha.south east |- B.west) 
;

\draw (Pilha.east) -- ([xshift=15pt]Pilha.east -| Pilha.south east);
\draw (Pilha.east) -- ([xshift=15pt]Pilha.east -| Pilha.south east);
\draw (Pilha.east) -- ([xshift=15pt]Pilha.east -| Pilha.south east);
\draw (End.east) -- ([xshift=15pt]End.east -| Pilha.south east);

\draw (UdC.west) -- ([xshift=-15pt]UdC.west -| UdC.south west);
\draw (ALU.west) -- ([xshift=-15pt]ALU.west -| UdC.south west);
\draw (B.west) -- ([xshift=-15pt]B.west -| UdC.south west);

\end{tikzpicture}    
    \caption{Abstração da CPU, usa referências com shift e cálculo de pontos por interseção de uma referência vertical com uma horizontal, estilos genéricos e específicos}
    \label{fig:comp12s}
\end{figure}


\begin{figure}[hbt]
    \centering
\begin{tikzpicture}[seta/.style = {-latex},
adf/.style = {minimum width = 2cm}]
  \node[doc,cascaded] (fet) at (0,0) {Fetch list};
  \node[draw=none,right = of fet] (int) {};
  \node[doc,cascaded, right = of int] (u) {Updates};
  \node[doc,cascaded, right = of u] (c) {Content};
  \node[doc,cascaded, right = of c] (i) {Indexes};

\node[server,adf,above = 2.0cm of int,label=below:{web db}] (wd) {};
\node[server,adf,below = 2.0cm of int,label=below:{fetchers}] (fs) {};
\node[server,adf,above = of i,label=above:{indexers}] (is) {};
\node[server,adf,below = of i,label=right:{searchers}] (ss) {};
\node[server,adf,below = of ss,label=below:{web servers}] (ws) {};

\draw[seta] (wd.west) to[bend right] (fet) ;
\draw[seta] (u) to[bend right] (wd.east);
\draw[seta] (fet) to[bend right] (fs.west);
\draw[seta] (fs.east) to[bend right] (u);
\draw[seta] (u) -- (c);
\draw[seta] (is) -- (i);
\draw[seta] (i) -- (ss);
\draw[seta] (c) to[bend left] (is.west);
\draw[seta] (c) to[bend right] (ss.west);
\draw[latex-latex] (ss) -- (ws);
\end{tikzpicture}
\caption{Lucene?}
\end{figure}

\section{Mecanismos de Busca}

\begin{figure}
    \centering
\resizebox{\textwidth}{!}{%
\begin{tikzpicture}[every node/.style = {rectangle,draw,%
minimum width = 15ex,
rounded corners=4,
minimum height = 24pt,
node distance= 18pt and 24pt,
align = center
},
every path/.style = {-latex},
db/.style = {cylinder, 
aspect=0.25,
rounded corners=0,
shape border rotate=90}]
\node[cloud] (web) at (0,0) {Web};
\node[right = of web] (crawler) {Crawler};
\node[db, right = of crawler] (paginas) {Páginas};
\node[right = of paginas] (prepro) {Pré-Processamento};
\node[ellipse,fill = yellow,rounded corners=0,, below = of web] (anu) {Anunciante};
\node[right = of anu] (sisanu) {Sistema de \\ Anúncios};
\node[below = of prepro] (outras) {Outras \\ Análises};
\node[right = of outras] (indexador) {Indexador};
\node[right = of indexador] (arede) {Análise \\ da Rede};
\node[below = of sisanu] (apre) {Aprendizado};
\node[db,right = of apre] (muser) {Modelo \\ Usuário};
\node[db, right = of muser] (outdad) {Outros \\ Dados};
\node[db, below = of indexador] (indice) {Índice};
\node[db, below = of arede] (mrede) {Modelo \\ da Rede};
\node[below left = of apre] (obs) {Observador};
\node[ellipse,fill = yellow,below = of obs] (user) {Usuário};
\node[right = of user] (interf) {Interface};
\node[right = of interf] (mod) {Modificador \\ de Consultas};
\node[right = of mod] (busca) {Buscador};

\draw (web) -- (crawler);
\draw (crawler) -- (paginas);
\draw (paginas) -- (prepro);
\draw[latex-latex] (anu) -- (sisanu);
\draw (sisanu) -- (outras);
\draw (prepro) -- (outras);
\draw ($(prepro.east)!.5!(prepro.south east)$) -| (indexador);
\draw (prepro.east) -| (arede);
\draw (arede) -- (mrede);
\draw (indexador) -- (indice);
\draw[latex-latex] (outras) -- (outdad);
\draw (mrede) |- ($(busca.east)!.5!(busca.south east)$);
\draw (indice) |- ($(busca.east)!.5!(busca.north east)$);
\draw (outdad) -- (busca);
\draw (outdad) -- (mod);
\draw (muser) -- (mod);
\draw (muser) -- (busca);
\draw[latex-latex] (mod) -- (busca);
\draw[latex-latex] (interf) -- (mod);
\draw[latex-latex] (user) -- (interf);
\draw (user) -- (obs);
\draw (obs) -- (apre);
\draw (apre) -- (muser);
\draw (obs) -- (sisanu);
\draw (obs.west) -| ($(anu.west)+
(-.5,0)$) |- ($(outras.west)!0.5!(outras.north west)$);
\draw (prepro.north) |- ($(paginas.north)+(0,.3)$) -| (crawler.north);
\draw (obs) -| ($(outdad.south west)!.2!(outdad.south)$);
\draw[latex-latex] (interf) -- (obs);
\end{tikzpicture}
}
    \caption{Modelo genérico de um mecanismo de busca moderno}
    \label{fig:mbm}
\end{figure}

\begin{figure}
    \centering
\resizebox{\textwidth}{!}{
\begin{tikzpicture}[every node/.style = {rectangle,draw,%
minimum width = 15ex,
rounded corners=4,
minimum height = 48pt,
node distance= 36pt and 32pt,
align = center
},
n/.style ={draw=none,minimum width = 0, minimum height = 0},
every path/.style = {-latex},
ruido/.style = {dashed},
ss/.style = {latex-latex},
db/.style = {cylinder, 
aspect=0.25,
rounded corners=0,
shape border rotate=90},
c1/.style={
%minimum height = 1cm, minimum width=4cm , 
aspect=1.5},
d1/.style = {minimum width = 1cm,
minimum height = 48pt}]

\node[cloud,c1] (io) at (0,0) {Ideia \\ Original};
\node[alice, minimum width=1cm,
below right = of io] (em) {Emissor};
\node[doc,d1,right = of em] (re) {Representação \\ Escrita};
\node[right = of re] (ir) {Indexador};
\node[db, right = of ir] (rp) {Repositório};
\node[below = of ir] (mb) {Mecanismo \\ de Busca};
\node[doc,d1,cascaded,left = of mb] (od) {Outros \\ Documentos};

\node[cloud,c1,below left = of em] (ni) {Necessidade \\ da \\ Informação};
\node[bob, minimum width=1cm,below right = of ni] (rc) {Receptor};
\node[cloud,c1, left = of rc,aspect=1.7 ] (ii) {Interpretação};
\node[doc,d1,right = of rc] (rn) {Representação \\ da Necessidade};

\draw[ruido,ss] (io) to[bend left]node[n,midway,above,yshift=4pt] {1} (em);
\draw[ruido,ss] (rc) to[bend left] node[n,midway,above] {g}  (ii);
\draw[ruido,ss] (rc) to[bend right] node[n,midway,above,xshift=4pt] {a} (ni);
\draw[ruido] (em) --node[n,midway,above] {2}   (re);
\draw (re) --node[n,midway,above] {3} (ir);
\draw[ruido] (ir) --node[n,midway,above] {4} (rp);

\draw[ruido] (rp) --node[n,midway,above] {d} (mb);
\draw (rn) --node[n,midway,above] {c} (mb);
\draw[ruido] (mb) --node[n,midway,above] {e} (re);
\draw[ruido] (re) to[bend right=10]node[n,midway,above] {f} (rc);
\draw[ruido] (rc) --node[n,midway,above] {b} (rn);
\draw[ruido] (mb) --node[n,midway,above] {e} (od);
\draw[ruido] (od) --node[n,midway,above] {f} (rc);
\end{tikzpicture}}

    \caption{Caption Identificada 1}
    \label{fig:my_labelsxg2}

    
\end{figure}

\begin{figure}
    \centering
    
\begin{tikzpicture}[every node/.style = {rectangle,draw,%
minimum width = 15ex,
rounded corners=4,
minimum height = 24pt,
node distance= 36pt and 32pt,
align = center
},
every path/.style = {-latex},
db/.style = {cylinder, 
aspect=0.25,
rounded corners=0,
shape border rotate=90}]
\node[doc,cascaded] (h) at (0,0) {HTML};
\node[doc,cascaded, right = of h] (p) {PDF};
\node[doc,cascaded, right = of p] (w) {Word};

\node[below = of h] (et1) {Extrair Texto};
\node[below = of p] (et2) {Extrair Texto};
\node[below = of w] (et3) {Extrair Texto};

\node[below = of et2] (a) {Análise};
\node[db, below = of a] (i) {Índice};

\draw (h) -- (et1);
\draw (p) -- (et2);
\draw (w) -- (et3);
\draw (et1) -- (a);
\draw (et2) -- (a);
\draw (et3) -- (a);
\draw (a) -- (i);
\end{tikzpicture}
    \caption{Indexar}
    \label{fig:my_label2sd312}
\end{figure}


