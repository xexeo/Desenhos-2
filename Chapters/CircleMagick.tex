\chapter{Circle Magic}

Pie charts are very easy!

\begin{verbatim}
\begin{tikzpicture}
\pie{24/SAP, 12/Oracle,
6/Sage, 6/Infor, 5/Microsoft,
47/Outros}
\end{tikzpicture}
\end{verbatim}

\begin{figure}[hbt]
    \centering
    \begin{tikzpicture}
\pie{24/SAP, 12/Oracle,
6/Sage, 6/Infor, 5/Microsoft,
47/Outros}
\end{tikzpicture}
    \caption{ERP Market in 2013 }
    \label{fig:ERPMarket}
\end{figure}

Using polar coordinates it is easy to draw a circle. The following code results in \autoref{fig:circle1}.

\begin{verbatim}
\begin{tikzpicture}
  \coordinate (center) at (1,2);
  \def\radius{2.5cm}
  % a circle
  \draw[dotted] (center) circle[radius=\radius];

  \fill[black] (center) ++(0:\radius)
  circle[radius=4pt] node[black,right] {1} ;

  \fill[red] (center) ++(36:\radius)
  circle[radius=2pt] node[right] {2 Malala's Call};

  \fill[red] (center) ++(2*36:\radius)
  circle[radius=2pt] node[above] {3};

  \fill[blue] (center) ++(3*36:\radius)
  circle[radius=2pt] node[above] {4};

  \fill[red] (center) ++(4*36:\radius)
  circle[radius=2pt] node[left] {5};

  \fill[red] (center) ++(5*36:\radius)
  circle[radius=2pt] node[left] {6};

  \fill[blue] (center) ++(6*36:\radius)
  circle[radius=2pt] node[left] {7};

  \fill[red] (center) ++(7*36:\radius)
  circle[radius=2pt] node[below] {8};

  \fill[red] (center) ++(8*36:\radius)
  circle[radius=2pt] node[below] {9};

  \fill[blue] (center) ++(9*36:\radius)
  circle[radius=2pt] node[right] {10};

   \draw[-{>[scale=2.5,
          length=2,
          width=3]}]  (center)+(4*36:\radius) --
          +(9*36:\radius) ;
   \draw[-{>[scale=2.5,
          length=2,
          width=3]}]  (center)+(2*36:\radius) --
          +(9*36:\radius) ;
   \draw[-{>[scale=2.5,
          length=2,
          width=3]}]  (center)+(0:\radius) --
          +(9*36:\radius) node [midway, right] {$w$};

   \draw[-{>[scale=2.5,
          length=2,
          width=3]}]  (center)+(7*36:\radius) --
          +(3*36:\radius) ;
   \draw[-{>[scale=2.5,
          length=2,
          width=3]}]  (center)+(5*36:\radius) --
          +(3*36:\radius) ;
   \draw[-{>[scale=2.5,
          length=2,
          width=3]}]  (center)+(0:\radius) --
          +(3*36:\radius) node [midway, below] {$w$};

  \draw[-{>[scale=2.5,
          length=2,
          width=3]}]  (center)+(1*36:\radius) --
          +(6*36:\radius) ;
   \draw[-{>[scale=2.5,
          length=2,
          width=3]}]  (center)+(8*36:\radius) --
          +(6*36:\radius) ;
   \draw[-{>[scale=2.5,
          length=2,
          width=3]}]  (center)+(0:\radius) --
          +(6*36:\radius) node [midway, below] {$w$};


\end{tikzpicture}
\end{verbatim}


\begin{figure}[hbt]
\begin{tikzpicture}
  \coordinate (center) at (1,2);
  \def\radius{2.5cm}
  % a circle
  \draw[dotted] (center) circle[radius=\radius];


  \fill[black] (center) ++(0:\radius)
  circle[radius=4pt] node[black,right] {1} ;

  \fill[red] (center) ++(36:\radius)
  circle[radius=2pt] node[right] {2 Malala's Call};

  \fill[red] (center) ++(2*36:\radius)
  circle[radius=2pt] node[above] {3};

  \fill[blue] (center) ++(3*36:\radius)
  circle[radius=2pt] node[above] {4};

  \fill[red] (center) ++(4*36:\radius)
  circle[radius=2pt] node[left] {5};

  \fill[red] (center) ++(5*36:\radius)
  circle[radius=2pt] node[left] {6};

  \fill[blue] (center) ++(6*36:\radius)
  circle[radius=2pt] node[left] {7};

  \fill[red] (center) ++(7*36:\radius)
  circle[radius=2pt] node[below] {8};

  \fill[red] (center) ++(8*36:\radius)
  circle[radius=2pt] node[below] {9};

  \fill[blue] (center) ++(9*36:\radius)
  circle[radius=2pt] node[right] {10};

   \draw[-{>[scale=2.5,
          length=2,
          width=3]}]  (center)+(4*36:\radius) --
          +(9*36:\radius) ;
   \draw[-{>[scale=2.5,
          length=2,
          width=3]}]  (center)+(2*36:\radius) --
          +(9*36:\radius) ;
   \draw[-{>[scale=2.5,
          length=2,
          width=3]}]  (center)+(0:\radius) --
          +(9*36:\radius) node [midway, right] {$w$};

   \draw[-{>[scale=2.5,
          length=2,
          width=3]}]  (center)+(7*36:\radius) --
          +(3*36:\radius) ;
   \draw[-{>[scale=2.5,
          length=2,
          width=3]}]  (center)+(5*36:\radius) --
          +(3*36:\radius) ;
   \draw[-{>[scale=2.5,
          length=2,
          width=3]}]  (center)+(0:\radius) --
          +(3*36:\radius) node [midway, below] {$w$};

  \draw[-{>[scale=2.5,
          length=2,
          width=3]}]  (center)+(1*36:\radius) --
          +(6*36:\radius) ;
   \draw[-{>[scale=2.5,
          length=2,
          width=3]}]  (center)+(8*36:\radius) --
          +(6*36:\radius) ;
   \draw[-{>[scale=2.5,
          length=2,
          width=3]}]  (center)+(0:\radius) --
          +(6*36:\radius) node [midway, below] {$w$};


\end{tikzpicture}
\caption{Points and chords in a circle.}
\label{fig:circle1}
\end{figure}

Another example of using polar coordinates to draw a circle results in \autoref{fig:malala}.

\begin{verbatim}
\begin{tikzpicture}
% posicao central do circulo
  \coordinate (center) at (1,2);
% coloca o nome aqui
\def\nome{Campbell's Hero Journey}
% nome fica no centro
  \node[align=center,text width=4cm,anchor=center] at (center) {\baselineskip=16pt \Huge{\nome}\par};
% raio do circulo
  \def\radius{4cm}
% numero de pontos
  \def\passos{10}
% tamanho em angulo graus do passo
  \def\passo{360/\passos}

% em vez de círculo, podemos usar um arco
% aqui tem um truque, que é usar o shift para
% o primeiro valor que você usar no nosso
% "loop aberto", no caso o 2*\passo
%

\draw[black] ([shift=(2*\passo:\radius)]center) arc (2*\passo:-7*\passo:\radius);

% cada ponto é um fill

% tem que acertar para cada ponto o multiplicador do passo
% isso deveria ser um for, mas é realmente melhor
% fazer na mão para controlar tudo

  \fill (center) ++(2*\passo:\radius)
   node[above,yshift=1em,xshift=2em] {\textbf{Primeiro Ato}};


  \fill[black] (center) ++(2*\passo:\radius)
  %circle[radius=4pt]
  node[regular polygon, regular polygon sides=3, fill,regular polygon rotate=-90,minimum width = 11pt,inner sep =0] {}
  node[left,yshift=-.7em ] {1} node[black, right,xshift=.5em,yshift=.3em] {Call to Action } ;

  \fill[black] (center) ++(1*\passo:\radius)
  circle[radius=2pt] node[left] {2} node[right] {Malala's Call};

  \fill[black] (center) ++(0*\passo:\radius)
  circle[radius=2pt] node[left] {3} node[right] {Malala's Call};

% SEGUNDO ATO

  \fill (center) ++(-1*\passo:\radius)
   node[above,yshift=.5em,xshift=4.3em] {\textbf{Segundo Ato}};
  \fill[black] (center) ++(-1*\passo:\radius)
  circle[radius=2pt] node[below,right] {Malala's Call}
  node[left] {4};

  \fill[black] (center) ++(-2*\passo:\radius)
  circle[radius=2pt]  node[above] {5} node[below,xshift=3.5em] { Malala's Call};

% TERCEIRO ATO

    \fill (center) ++(-3*\passo:\radius)
   node[below,yshift=-.5em,xshift=-3em] {\textbf{Terceiro Ato}};

  \fill[black] (center) ++(-3*\passo:\radius)
  circle[radius=2pt]  node[above] {6} node[left,xshift=-1em] {Malala's Call};



  \fill[black] (center) ++(-4*\passo:\radius)
  circle[radius=2pt] node[right] {7} node[left] {Malala's Call};

  \fill[black] (center) ++(-5*\passo:\radius)
  circle[radius=2pt] node[right] {8} node[left] {Malala's Call};

  \fill[black] (center) ++(-6*\passo:\radius)
  circle[radius=2pt] node[right] {9} node[left]  {Malala's Call};

  \fill[black] (center) ++(-7*\passo:\radius)
  node[shape=rectangle,fill] {} node[right,yshift=-.5em] {10}  node[above,xshift=-3.5em] {Malala's Call };

 \end{tikzpicture}
\end{verbatim}


\begin{figure}[hbt]
\begin{tikzpicture}
% posicao central do circulo
  \coordinate (center) at (1,2);
% coloca o nome aqui
\def\nome{Campbell's Hero Journey}
% nome fica no centro
  \node[align=center,text width=4cm,anchor=center] at (center) {\baselineskip=16pt \Huge{\nome}\par};
% raio do circulo
  \def\radius{4cm}
% numero de pontos
  \def\passos{10}
% tamanho em angulo graus do passo
  \def\passo{360/\passos}

% em vez de círculo, podemos usar um arco
% aqui tem um truque, que é usar o shift para
% o primeiro valor que você usar no nosso
% "loop aberto", no caso o 2*\passo
%

\draw[black] ([shift=(2*\passo:\radius)]center) arc (2*\passo:-7*\passo:\radius);

% cada ponto é um fill

% tem que acertar para cada ponto o multiplicador do passo
% isso deveria ser um for, mas é realmente melhor
% fazer na mão para controlar tudo

  \fill (center) ++(2*\passo:\radius)
   node[above,yshift=1em,xshift=2em] {\textbf{Primeiro Ato}};


  \fill[black] (center) ++(2*\passo:\radius)
  %circle[radius=4pt]
  node[regular polygon, regular polygon sides=3, fill,regular polygon rotate=-90,minimum width = 11pt,inner sep =0] {}
  node[left,yshift=-.7em ] {1} node[black, right,xshift=.5em,yshift=.3em] {Call to Action } ;

  \fill[black] (center) ++(1*\passo:\radius)
  circle[radius=2pt] node[left] {2} node[right] {Malala's Call};

  \fill[black] (center) ++(0*\passo:\radius)
  circle[radius=2pt] node[left] {3} node[right] {Malala's Call};

% SEGUNDO ATO

  \fill (center) ++(-1*\passo:\radius)
   node[above,yshift=.5em,xshift=4.3em] {\textbf{Segundo Ato}};
  \fill[black] (center) ++(-1*\passo:\radius)
  circle[radius=2pt] node[below,right] {Malala's Call}
  node[left] {4};

  \fill[black] (center) ++(-2*\passo:\radius)
  circle[radius=2pt]  node[above] {5} node[below,xshift=3.5em] { Malala's Call};

% TERCEIRO ATO

    \fill (center) ++(-3*\passo:\radius)
   node[below,yshift=-.5em,xshift=-3em] {\textbf{Terceiro Ato}};

  \fill[black] (center) ++(-3*\passo:\radius)
  circle[radius=2pt]  node[above] {6} node[left,xshift=-1em] {Malala's Call};



  \fill[black] (center) ++(-4*\passo:\radius)
  circle[radius=2pt] node[right] {7} node[left] {Malala's Call};

  \fill[black] (center) ++(-5*\passo:\radius)
  circle[radius=2pt] node[right] {8} node[left] {Malala's Call};

  \fill[black] (center) ++(-6*\passo:\radius)
  circle[radius=2pt] node[right] {9} node[left]  {Malala's Call};

  \fill[black] (center) ++(-7*\passo:\radius)
  node[shape=rectangle,fill] {} node[right,yshift=-.5em] {10}  node[above,xshift=-3.5em] {Malala's Call };

 \end{tikzpicture}
\caption{The Heroine Learner Journey}
\label{fig:malala}
\end{figure}





