\chapter{3D and Fake 3D}

Some times I had to build some 3D drawings based on boxes, for example, to describe a Data Warehouse Cube. There is an easy solution that is to develop a basic cube subroutine and use them to build more complex figures.

O próximo código construi uma caixa calculando os pontos, como uma projeção de 3D em 3D.

\begin{lstlisting}[style=myLateX,caption=Cubo azul em Fake 3D]
\newcommand{\drawbox}[5]{
    \pgfmathsetmacro \angle {30}
    \pgfmathsetmacro \xd {{2/3*cos(\angle)*#5}}
    \pgfmathsetmacro \yd {{2/3*sin(\angle)*#5}}
    \pgfmathsetmacro \x {{#1-#5+(#2-#5)*(\xd)*#5}}
    \pgfmathsetmacro \y {{#3-#5+(#2-#5)*(\yd)*#5}}

    \draw[fill=#4] (\x,\y) --
    (\x+#5,\y) -- (\x+#5,\y+#5) --
    (\x,\y+#5) -- cycle;

    \draw[fill=#4] (\x,\y+#5) --
    (\x+\xd,\y+#5+\yd) --
    (\x+#5+\xd,\y+#5+\yd) --
    (\x+#5,\y+#5) -- cycle;

    \draw[fill=#4] (\x+#5,\y+#5) --
    (\x+#5+\xd,\y+#5+\yd) --
    (\x+#5+\xd,\y+\yd) --
    (\x+#5,\y) -- cycle;

\drawbox{}{1}{1}{blue}{1}
}
\end{lstlisting}

\newcommand{\drawbox}[5]{
    \pgfmathsetmacro \angle {30}
    \pgfmathsetmacro \xd {{2/3*cos(\angle)*#5}}
    \pgfmathsetmacro \yd {{2/3*sin(\angle)*#5}}
    \pgfmathsetmacro \x {{#1-#5+(#2-#5)*(\xd)*#5}}
    \pgfmathsetmacro \y {{#3-#5+(#2-#5)*(\yd)*#5}}

    \draw[fill=#4] (\x,\y) -- (\x+#5,\y) -- (\x+#5,\y+#5) -- (\x,\y+#5) -- cycle;

    \draw[fill=#4] (\x,\y+#5) -- (\x+\xd,\y+#5+\yd) -- (\x+#5+\xd,\y+#5+\yd) -- (\x+#5,\y+#5) -- cycle;

    \draw[fill=#4] (\x+#5,\y+#5) -- (\x+#5+\xd,\y+#5+\yd) -- (\x+#5+\xd,\y+\yd) -- (\x+#5,\y) -- cycle;
}


A simple blue cube, \autoref{fig:box3dfake}, can be easily drawn with:



\begin{figure}[hbt]
    \centering
  \begin{tikzpicture}
  \drawbox{}{1}{1}{blue}{1}
  \end{tikzpicture}
    \caption{A fake 3D cube}
    \label{fig:box3dfake}
\end{figure}

And a composite figure can use the order of drawing to build a cube made of cubes, as in \autoref{fig:box3dfake1}.

\begin{lstlisting}[style=myLaTeX,caption=A cube made of cubes]

\pgfmathsetmacro{\profX}{5}

  \drawbox{1}{\profX}{1}{green}{.5}
  \drawbox{1.5}{\profX}{1}{green}{.5}
  \drawbox{2}{\profX}{1}{green}{.5}

  \drawbox{1}{\profX}{1.5}{green!50}{.5}
  \drawbox{1.5}{\profX}{1.5}{green!50}{.5}
  \drawbox{2}{\profX}{1.5}{green!50}{.5}

  \drawbox{1}{\profX}{2}{green!25}{.5}
  \drawbox{1.5}{\profX}{2}{green!25}{.5}
  \drawbox{2}{\profX}{2}{green!25}{.5}

  \pgfmathsetmacro{\profX}{3}

  \drawbox{1}{\profX}{1}{red}{.5}
  \drawbox{1.5}{\profX}{1}{red}{.5}
  \drawbox{2}{\profX}{1}{red}{.5}

  \drawbox{1}{\profX}{1.5}{red!50}{.5}
  \drawbox{1.5}{\profX}{1.5}{red!50}{.5}
  \drawbox{2}{\profX}{1.5}{red!50}{.5}

  \drawbox{1}{\profX}{2}{red!25}{.5}
  \drawbox{1.5}{\profX}{2}{red!25}{.5}
  \drawbox{2}{\profX}{2}{red!25}{.5}

  \drawbox{1}{1}{1}{blue}{.5}
  \drawbox{1.5}{1}{1}{blue}{.5}
  \drawbox{2}{1}{1}{blue}{.5}

  \drawbox{1}{1}{1.5}{blue!50}{.5}
  \drawbox{1.5}{1}{1.5}{blue!50}{.5}
  \drawbox{2}{1}{1.5}{blue!50}{.5}

  \drawbox{1}{1}{2}{blue!25}{.5}
  \drawbox{1.5}{1}{2}{blue!25}{.5}
  \drawbox{2}{1}{2}{blue!25}{.5}

\end{lstlisting}

\begin{figure}[hbt]
    \centering
\begin{tikzpicture}
\pgfmathsetmacro{\profX}{5}

  \drawbox{1}{\profX}{1}{green}{.5}
  \drawbox{1.5}{\profX}{1}{green}{.5}
  \drawbox{2}{\profX}{1}{green}{.5}

  \drawbox{1}{\profX}{1.5}{green!50}{.5}
  \drawbox{1.5}{\profX}{1.5}{green!50}{.5}
  \drawbox{2}{\profX}{1.5}{green!50}{.5}

  \drawbox{1}{\profX}{2}{green!25}{.5}
  \drawbox{1.5}{\profX}{2}{green!25}{.5}
  \drawbox{2}{\profX}{2}{green!25}{.5}

  \pgfmathsetmacro{\profX}{3}

  \drawbox{1}{\profX}{1}{red}{.5}
  \drawbox{1.5}{\profX}{1}{red}{.5}
  \drawbox{2}{\profX}{1}{red}{.5}

  \drawbox{1}{\profX}{1.5}{red!50}{.5}
  \drawbox{1.5}{\profX}{1.5}{red!50}{.5}
  \drawbox{2}{\profX}{1.5}{red!50}{.5}

  \drawbox{1}{\profX}{2}{red!25}{.5}
  \drawbox{1.5}{\profX}{2}{red!25}{.5}
  \drawbox{2}{\profX}{2}{red!25}{.5}

  \drawbox{1}{1}{1}{blue}{.5}
  \drawbox{1.5}{1}{1}{blue}{.5}
  \drawbox{2}{1}{1}{blue}{.5}

  \drawbox{1}{1}{1.5}{blue!50}{.5}
  \drawbox{1.5}{1}{1.5}{blue!50}{.5}
  \drawbox{2}{1}{1.5}{blue!50}{.5}

  \drawbox{1}{1}{2}{blue!25}{.5}
  \drawbox{1.5}{1}{2}{blue!25}{.5}
  \drawbox{2}{1}{2}{blue!25}{.5}

  \end{tikzpicture}
    \caption{A fake 3D cube of cubes}
    \label{fig:box3dfake1}
\end{figure}

It is also possible to use fixed coordinates, such as the following code, that result in \autoref{fig:2cubes}.

\begin{lstlisting}[style=myLaTeX,caption=A cube with fixed 2d Coordinates]


\begin{tikzpicture}[scale=0.33] %[x={10.0pt},y={10.0pt}]
\draw[line width=2pt] (0,0) -- (0,10);
\draw[line width=2pt] (0,0) -- (10,0);
\draw[line width=2pt] (0,10) -- (10,10);
\draw[line width=2pt] (0,10) -- (10,10);
\draw[line width=2pt] (10,0) -- (10,10);
\draw[line width=2pt] (0,0) -- (3,5);
\draw[line width=2pt] (10,0) -- (13,5);
\draw[line width=2pt] (0,10) -- (3,15);
\draw[line width=2pt] (10,10) -- (13,15);
\draw[line width=2pt] (3,5) -- (13,5);
\draw[line width=2pt] (3,5) -- (3,15);
\draw[line width=2pt] (13,5) -- (13,15);
\draw[line width=2pt] (3,15) -- (13,15);
\draw[line width=1pt] (5,0) -- (5,10);
\draw[line width=1pt] (5,10) -- (8,15);
\draw[line width=1pt] (5,0) -- (8,5);
\draw[line width=1pt] (8,5) -- (8,15);
\draw[black, fill=blue,fill opacity=0.5] (5,0) -- (5,10) -- (8,15) -- (8,5) -- cycle;
\end{tikzpicture}% pic 1
\qquad % <----------------- SPACE BETWEEN PICTURES
% ou
%\hspace{3cm}
\begin{tikzpicture}[scale=0.33] %[x={10.0pt},y={10.0pt}]
\draw[line width=2pt] (0,0) -- (0,10);
\draw[line width=2pt] (0,0) -- (10,0);
\draw[line width=2pt] (0,10) -- (10,10);
\draw[line width=2pt] (0,10) -- (10,10);
\draw[line width=2pt] (10,0) -- (10,10);
\draw[line width=2pt] (0,0) -- (3,5);
\draw[line width=2pt] (10,0) -- (13,5);
\draw[line width=2pt] (0,10) -- (3,15);
\draw[line width=2pt] (10,10) -- (13,15);
\draw[line width=2pt] (3,5) -- (13,5);
\draw[line width=2pt] (3,5) -- (3,15);
\draw[line width=2pt] (13,5) -- (13,15);
\draw[line width=2pt] (3,15) -- (13,15);
\draw[line  width=1pt] (0,10) -- (10,0);
\draw[line  width=1pt] (0,10) -- (3,5);
\draw[line  width=1pt] (3,5) -- (10,0);
\draw[black, fill=blue,fill opacity=0.5] (0,10) -- (10,0) -- (3,5) -- cycle;
\draw[line  width=1pt] (3,15) -- (10,10);
\draw[line  width=1pt] (3,15) -- (13,5);
\draw[line  width=1pt] (10,10) -- (13,5);
\draw[black, fill=blue,fill opacity=0.5] (3,15) -- (10,10) -- (13,5) -- cycle;
\end{tikzpicture}% pic 2
\end{lstlisting}


\begin{figure}[hbt]
\begin{center}
\begin{tikzpicture}[scale=0.33] %[x={10.0pt},y={10.0pt}]
\draw[line width=2pt] (0,0) -- (0,10);
\draw[line width=2pt] (0,0) -- (10,0);
\draw[line width=2pt] (0,10) -- (10,10);
\draw[line width=2pt] (0,10) -- (10,10);
\draw[line width=2pt] (10,0) -- (10,10);
\draw[line width=2pt] (0,0) -- (3,5);
\draw[line width=2pt] (10,0) -- (13,5);
\draw[line width=2pt] (0,10) -- (3,15);
\draw[line width=2pt] (10,10) -- (13,15);
\draw[line width=2pt] (3,5) -- (13,5);
\draw[line width=2pt] (3,5) -- (3,15);
\draw[line width=2pt] (13,5) -- (13,15);
\draw[line width=2pt] (3,15) -- (13,15);
\draw[line width=1pt] (5,0) -- (5,10);
\draw[line width=1pt] (5,10) -- (8,15);
\draw[line width=1pt] (5,0) -- (8,5);
\draw[line width=1pt] (8,5) -- (8,15);
\draw[black, fill=blue,fill opacity=0.5] (5,0) -- (5,10) -- (8,15) -- (8,5) -- cycle;
\end{tikzpicture}% pic 1
\qquad % <----------------- SPACE BETWEEN PICTURES
% ou
%\hspace{3cm}
\begin{tikzpicture}[scale=0.33] %[x={10.0pt},y={10.0pt}]
\draw[line width=2pt] (0,0) -- (0,10);
\draw[line width=2pt] (0,0) -- (10,0);
\draw[line width=2pt] (0,10) -- (10,10);
\draw[line width=2pt] (0,10) -- (10,10);
\draw[line width=2pt] (10,0) -- (10,10);
\draw[line width=2pt] (0,0) -- (3,5);
\draw[line width=2pt] (10,0) -- (13,5);
\draw[line width=2pt] (0,10) -- (3,15);
\draw[line width=2pt] (10,10) -- (13,15);
\draw[line width=2pt] (3,5) -- (13,5);
\draw[line width=2pt] (3,5) -- (3,15);
\draw[line width=2pt] (13,5) -- (13,15);
\draw[line width=2pt] (3,15) -- (13,15);
\draw[line  width=1pt] (0,10) -- (10,0);
\draw[line  width=1pt] (0,10) -- (3,5);
\draw[line  width=1pt] (3,5) -- (10,0);
\draw[black, fill=blue,fill opacity=0.5] (0,10) -- (10,0) -- (3,5) -- cycle;
\draw[line  width=1pt] (3,15) -- (10,10);
\draw[line  width=1pt] (3,15) -- (13,5);
\draw[line  width=1pt] (10,10) -- (13,5);
\draw[black, fill=blue,fill opacity=0.5] (3,15) -- (10,10) -- (13,5) -- cycle;
\end{tikzpicture}% pic 2
\end{center}
    \caption{Two cubes and 3 planes.}
    \label{fig:2cubes}
\end{figure}

\section{Trying to draw a Ellisoid}

\begin{figure}[htb]
\begin{tikzpicture}[x={(1cm,0cm)},y={(0cm,1cm)},z={(0.8cm,0.6cm)}]
% ellipsoid
\draw[gray] (0,0,0) arc (90:270:1 and 2);
\draw[dashed,gray] (0,0,0) arc (90:-90:1 and 2);
\draw[gray] (0,0,0) arc (270:450:1 and 2);
\draw[dashed,gray] (0,0,0) arc (270:90:1 and 2);
\draw[gray] (1,0,0) arc (0:360:1 and 0.5);
% axes
\draw[->] (0,0,0) -- (3,0,0) node[anchor=north east]{$x$};
\draw[->] (0,0,0) -- (0,2,0) node[anchor=north west]{$y$};
\draw[->] (0,0,0) -- (0,0,3) node[anchor=south]{$z$};
\end{tikzpicture}
\caption{Doido que eu devia arrumar}
\end{figure}

\begin{figure}[htb]
\centering
\begin{tikzpicture}[baseline=(current bounding box.center),x=0.5cm,y=0.5cm,z=0.3cm]
  \coordinate (O) at (0,0,0);
  \coordinate (A) at (6,0,0);
  \coordinate (B) at (0,3,0);
  \coordinate (C) at (0,0,4);

  \draw[-Latex] (O) -- (A) node[above] {$x$};
  \draw[-Latex] (O) -- (B) node[above] {$y$};
  \draw[-Latex] (O) -- (C) node[right] {$z$};

  \pgfmathsetmacro{\a}{4}
  \pgfmathsetmacro{\b}{2}
  \pgfmathsetmacro{\c}{1}

  \draw[ball color=blue!40,opacity=0.5] (O) ellipse ({\a} and {\b});

  \draw[thick,blue,-latex] (O) -- ({\a},0,0) node[anchor=north east]{$a$};
  \draw[thick,blue,-latex] (O) -- (0,{\b},0) node[anchor=north west]{$b$};
  \draw[thick,blue,-latex] (O) -- (0,0,{\c}) node[anchor=south]{$c$};

\begin{scope}[canvas is xz plane at y=0]
        \draw[dashed] (0,0) circle[x radius=\a, y radius=\c];
    \end{scope}

    \begin{scope}[canvas is xy plane at z=0]
        \draw[dashed] (0,0) circle[x radius=\a, y radius=\b];
    \end{scope}

    \begin{scope}[canvas is yz plane at x=0]
        \draw[dashed] (0,0) circle[x radius=\b, y radius=\c];
    \end{scope}

\end{tikzpicture}
\caption{Usa ball e fica bonitinho}
\end{figure}

\begin{figure}[htb]
\centering

\begin{tikzpicture}[scale=2, x={(1cm,0cm)}, y={(0cm,1cm)}, z={(0.5cm,0.5cm)}]
    \pgfmathsetmacro\a{1}
    \pgfmathsetmacro\b{0.7}
    \pgfmathsetmacro\c{0.5}

    \begin{scope}[canvas is xz plane at y=0]
        \draw[dashed] (0,0) circle[x radius=\a, y radius=\c];
    \end{scope}

    \begin{scope}[canvas is xy plane at z=0]
        \draw[dashed] (0,0) circle[x radius=\a, y radius=\b];
    \end{scope}

    \begin{scope}[canvas is yz plane at x=0]
        \draw[dashed] (0,0) circle[x radius=\b, y radius=\c];
    \end{scope}

    \draw[->] (-1.5,0,0) -- (1.5,0,0) node[right]{$x$};
    \draw[->] (0,-1.5,0) -- (0,1.5,0) node[above]{$y$};
    \draw[->] (0,0,-1.5) -- (0,0,1.5) node[below left]{$z$};

    \pgfmathsetmacro\anglex{30}
    \pgfmathsetmacro\angley{45}

    \begin{scope}[rotate around x=\anglex]
        \begin{scope}[rotate around y=\angley]
            \draw[fill=blue!50, opacity=0.5] (0,0,0) plot[variable=\t, domain=0:2*pi] ({\a*cos(deg(\t))},{\b*sin(deg(\t))},0) plot[variable=\t, domain=0:2*pi] ({\a*cos(deg(\t))},0,{\c*sin(deg(\t))}) plot[variable=\t, domain=0:2*pi] (0,{\b*cos(deg(\t))},{\c*sin(deg(\t))});
        \end{scope}
    \end{scope}
\end{tikzpicture}
\caption{Consertado por mim}
\end{figure}

\begin{figure}[htb]
\centering
\begin{tikzpicture}
\begin{axis}
[view={135}{20},%colormap/blackwhite,
axis lines=center, axis on top,ticks=none,
set layers=default,axis equal,
xlabel={$x$}, ylabel={$y$}, zlabel={$z$},
xlabel style={anchor=south east},
ylabel style={anchor=south west},
zlabel style={anchor=south west},
enlargelimits,
tick align=inside,
domain=0:2.00,
samples=20,
z buffer=sort,
]
\addplot3 [surf,opacity=0.4,domain=-1:0,
domain y=0:360] ({sin(y)*sqrt(1-x^2)},{2*cos(y)*sqrt(1-x^2)},{x});
\addplot3 [surf,opacity=0.4,domain=0:1,
domain y=0:360,on layer=axis foreground] ({sin(y)*sqrt(1-x^2)},{2*cos(y)*sqrt(1-x^2)},{x});
\end{axis}
\end{tikzpicture}
\caption{Quente que eu achei em um site}
% https://tex.stackexchange.com/questions/417491/draw-an-ellipsoid}
\end{figure}

\begin{figure}[htb]
\centering
\begin{tikzpicture}[scale=2]
  \pgfmathsetmacro\a{2}
  \pgfmathsetmacro\b{1.5}
  \pgfmathsetmacro\c{1}
  \pgfmathsetmacro\x{3} % x-axis scale factor

  % ellipsoid coordinates
  \foreach \theta in {0,10,...,170}
    \foreach \phi in {0,10,...,360}
    {
      \pgfmathsetmacro\x{\a*sin(\theta)*cos(\phi)}
      \pgfmathsetmacro\y{\b*sin(\theta)*sin(\phi)}
      \pgfmathsetmacro\z{\c*cos(\theta)}
      \coordinate (e) at (\x,\y,\z);
      \pgfmathsetmacro\rotX{30} % rotation angle in x
      \pgfmathsetmacro\rotY{45} % rotation angle in y
      \path (e) [rotate around x=\rotX, rotate around y=\rotY] coordinate (e');
      \draw[black!50] (e')--++(0,0.02,0) (e')--++(0,-0.02,0) (e')--++(0,0,0.02) (e')--++(0,0,-0.02);
    }

  % x-axis
  \draw[red, very thick, -latex] (0,0,0) -- (\a*\x,0,0);
  \node[red, above] at (\a*\x/2,0,0) {$x$};

  % y-axis
  \draw[green!70!black, very thick, -latex] (0,0,0) -- (0,\b,0);
  \node[green!70!black, right] at (0,\b/2,0) {$y$};

  % z-axis
  \draw[blue, very thick, -latex] (0,0,0) -- (0,0,\c);
  \node[blue, above] at (0,0,\c/2) {$z$};
\end{tikzpicture}
\caption{Não sei como o Chat GPT gerou isso}
\end{figure}

\begin{figure}[htb]
\centering
\begin{tikzpicture}
\begin{axis}
[view={135}{20},colormap={blue}{
            color=(blue) color=(blue)
        },
axis lines=center,axis on top,
ticks=none,set layers=default,axis equal,
xlabel={$x$}, ylabel={$y$}, zlabel={$z$},
xlabel style={anchor=south east},
ylabel style={anchor=south west},
zlabel style={anchor=south west},
enlargelimits,
tick align=inside,
domain=0:2.00,
samples=20,
z buffer=sort,
]
\addplot3 [surf,draw=blue!30,fill=white,domain=-1:1,samples=20,
domain y=00:180] ({x},{-2*cos(y)*sqrt(1-x^2)},{-sin(y)*sqrt(1-x^2)});
\addplot3 [surf,draw=blue!30,fill=white,domain=-1:1,
domain y=0:180,on layer=axis foreground] ({x},{2*cos(y)*sqrt(1-x^2)},{sin(y)*sqrt(1-x^2)});
\end{axis}
\end{tikzpicture}
\caption{Bem parecido com que eu quero}
\end{figure}

\begin{figure}[htb]
\centering
\begin{tikzpicture}[remember picture]
%\pgfsetlayers{pre main,main,axis foregound}
\begin{axis}
[view={135}{20},colormap={blue}{
            color=(cyan) color=(cyan)
        },axis lines=none,axis equal,set layers=standard,
enlargelimits,domain=0:2,samples=20, z buffer=sort,
]

\pgfonlayer{axis background}
\draw[-] (axis cs:0,0,-1.5)--(axis cs:0,0,-1);
\draw[-] (axis cs:0,-3,0)--(axis cs:0,-2,0);
\draw[-] (axis cs:-2,0,0)--(axis cs:-1,0,0);
\addplot3[draw=none] coordinates{(0,0,-2) (0,0,2)};
\addplot3[draw=none] coordinates{(0,-2.5,0) (0,2.8,0)};
\endpgfonlayer
\pgfonlayer{main}
\addplot3 [surf,draw=cyan,fill=white,domain=0:1,samples=20,
domain y=00:180] ({x},{-2*cos(y)*sqrt(1-x^2)},{-sin(y)*sqrt(1-x^2)});
\addplot3 [surf,draw=cyan,fill=white,domain=0:1,
domain y=0:180,on layer=axis foreground] ({x},{2*cos(y)*sqrt(1-x^2)},{sin(y)*sqrt(1-x^2)});
\coordinate(x1) at (axis cs:1,0,0);
\coordinate(x2) at (axis cs:1.5,0,0);
\coordinate(y1) at (axis cs:0,1.9,0);
\coordinate(y2) at (axis cs:0,2.5,0);
\coordinate(z1) at (axis cs:0,0,0.9);
\coordinate(z2) at (axis cs:0,0,1.5);
\endpgfonlayer
\end{axis}
\end{tikzpicture}
\begin{tikzpicture}[remember picture,overlay]
\draw[->] (x1)--(x2)node[left]{$x$};
\draw[->] (y1)--(y2)node[right]{$y$};
\draw[->] (z1)--(z2)node[right]{$z$};
\end{tikzpicture}
\caption{Novamente parecido com que eu quero}
\end{figure}

\begin{figure}

\begin{center}
\begin{tikzpicture}[baseline=(current bounding box.center),x=0.5cm,y=0.5cm,z=0.3cm]
  \coordinate (O) at (0,0,0);
  \coordinate (A) at (6,0,0);
  \coordinate (B) at (0,3,0);
  \coordinate (C) at (0,0,4);

  \draw[-Latex] (O) -- (A) node[above] {$x$};
  \draw[-Latex] (O) -- (B) node[above] {$y$};
  \draw[-Latex] (O) -- (C) node[right] {$z$};

\begin{scope}[rotate around y=30,rotate around z=-30]

  \pgfmathsetmacro{\a}{4}
  \pgfmathsetmacro{\b}{2}
  \pgfmathsetmacro{\c}{1}



  %ellipse ({\a} and {\b});

  \draw[thick,blue,-latex] (O) -- ({\a},0,0) node[anchor=north east]{$a$};
  \draw[thick,blue,-latex] (O) -- (0,{\b},0) node[anchor=north west]{$b$};
  \draw[thick,blue,-latex] (O) -- (0,0,{\c}) node[anchor=south]{$c$};

    \begin{scope}[canvas is xz plane at y=0]
    \draw[dashed] (O) circle[x radius=\a, y radius=\c];
    \end{scope}
    \begin{scope}[canvas is xy plane at z=0]
      \draw[ball color=blue!40,opacity=0.5] (O)  circle[x radius=\a, y radius=\b];
        \draw[dashed] (O) circle[x radius=\a, y radius=\b];
    \end{scope}
    \begin{scope}[canvas is yz plane at x=0]
        \draw[dashed] (O) circle[x radius=\b, y radius=\c];
    \end{scope}

\end{scope}

\end{tikzpicture}
\end{center}

\caption{Rotação por escopo!}
\end{figure}

